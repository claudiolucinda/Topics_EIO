\documentclass{beamer}
\usepackage{beamerthemesplit}
\usepackage[brazil]{babel}
\usepackage{epsfig}
\usepackage[utf8x]{inputenc}
\usepackage{pgf}
%\usepackage{tikz}
%\usetikzlibrary{snakes}
\usepackage{nicefrac}
\usepackage{amsfonts}
\usepackage{amsmath}
\usepackage{amssymb}
\usepackage{amsthm}
%\usepackage{float}
\usetheme{Frankfurt}
\usepackage{epstopdf}
\usepackage{comment}
\usepackage{natbib}
\usepackage{float}
\usepackage{graphicx}
\usepackage{booktabs}
\usepackage{array}
\title{Aula 03-b}

\subtitle{Identificação de Modelos de Demanda}

\author{Claudio R. Lucinda}


\institute{FEA/USP}

\date{}
\logo{\includegraphics[scale=.1]{logousp.png}}
\beamertemplatenavigationsymbolsempty
\begin{document}

\frame{\titlepage}
\begin{frame}{Agenda}
  \tableofcontents[pausesections]
\end{frame}


\section{Identificação de Modelos de Demanda}
\begin{frame}{\insertsection}
\begin{itemize}
    \item Vamos agora falar sobre a identificação dos modelos de demanda.
    \item Alguns destes resultados são úteis para modelos de demanda neoclássicos e outros são úteis para modelos de escolha discreta.
    \item Este aspecto é especialmente importante para organização industrial, uma vez que estamos especialmente preocupados com estimarmos parâmetros estruturais.
    \item E, em cima destes parâmetros estruturais, vamos construir os nossos contrafactuais
    \item Vou dividir estas abordagens em três categorias:
    \begin{itemize}
        \item Abordagens ``tradicionais''
        \item Abordagem de Hausman
        \item Abordagem BLP
    \end{itemize}
    \item Um texto legal sobre o assunto é o de \citet{Bresnahan1997a}
\end{itemize}

    
\end{frame}

\subsection{Abordagens Tradicionais}

\begin{frame}{Abordagens Tradicionais}
\begin{itemize}
    \item \textbf{Deslocadores de Custo}: Preços de fatores, coisas do gênero. São bons, mas:
    \begin{itemize}
        \item Nem sempre são fáceis de se conseguir, e mais difícil ainda de conseguir em quantidade suficiente para um monte de coeficientes estimáveis.
        \item Se você tiver menos deslocadores de custos que coeficientes estimáveis, tem jeito de resolver (interações com dummies de produtos, etc e tal...). Não é ideal.
    \end{itemize}
    \item \textbf{Dados Individuais de Consumidores}: Você pode assumir que os preços são exógenos às escolhas dos consumidores e a variação das características dos consumidores permitem a identificação da curva de demanda. Precisa de dados.
\end{itemize}
    
\end{frame}

\subsection{Abordagem de Hausman}

\begin{frame}{Abordagem de Hausman}
\begin{itemize}
    \item A abordagem de Hausman é baseada em um argumento de componentes de variância.
    \item Vamos entender isso assumindo a seguinte forma reduzida para o preço do produto $i$, no instante $t$ na região $r$:
\end{itemize}
\[
p_{itm}=\mathbf{W}\gamma+\epsilon_{itm}
\]
\begin{itemize}
    \item Vamos assumir que este termo erro $\epsilon$ tenha uma estrutura de componentes da seguinte forma:
\end{itemize}

\[
\epsilon_{itm}=f_{i\in K}+g_{t}+h_{m}+\varepsilon_{itm}
\]

\begin{itemize}
    \item Em que $f_{i\in K},g_{t}$ e $h_{m}$ são componentes comuns de empresa, instante de tempo e mercado
\end{itemize}
\end{frame}

\begin{frame}{Abordagem de Hausman II}
\begin{itemize}
    \item Neste caso, podemos dizer que:
    \[
    Cov(p_{itm},p_{it(-m)}) \neq 0
    \]
    \item Porque o componente $f_{i \in K}$ e $g_{t}$ são  comuns aos dois produtos.
    \item \textbf{Instrumentos de Hausman}: Para o preço do produto $i$ no mercado/instante do tempo  $m$, usar o preço do produto $i$ em um mercado/instante do tempo DIFERENTE.
    \item Isso presumivelmente garantiria a condição de força dos instrumentos.
    \item Esta abordagem foi aplicada com em \citet{HLZ94}, daí o nome de ``Instrumentos de Hausman''
\end{itemize}

    
\end{frame}

\begin{frame}{Abordagem de Hausman III}
\begin{itemize}
    \item O problema aqui não é a restrição de força dos instrumentos.
    \item O problema aqui é sobre a restrição de exclusão dos instrumentos. 
    \item Será que não existe um componente comum para o termo erro da equação demanda desse jeito aqui?
    \[
    q_{itm}=f(\mathbf{p}_{tm})+\varepsilon_{itm}
    \]
    \item Se existir componente comum na demanda similar à estrutura de componentes comuns impostas na forma reduzida mais acima, o instrumento não passa na condição de exclusão (instrumento não correlacionado com o termo erro).
    \item Este é um ponto que foi levantado por \citet{Bresnahan1997a}, e que precisa ser checada em qualquer aplicação.
\end{itemize}
    
\end{frame}

\subsection{Modelos de Escolha Discreta}

\begin{frame}{Modelos de Escolha Discreta}
\begin{itemize}
    \item Com modelos de escolha discreta e com dados agregados, a coisa fica um pouco mais complicada.
    \item Inicialmente vamos revisitar a estratégia adotada no paper de \citet{Berry1995}, para depois falarmos do que veio depois.
    \item No paper original de \citet{Berry1995}, a identificação vinha de três lugares:
    \begin{itemize}
        \item Os ``Instrumentos BLP'': Somas de características de outros produtos da mesma marca e Somas de características dos produtos de outras marcas.
        \item Condições de Momento Adicionais do lado da oferta.
        \item Um deslocador de custo.
        
    \end{itemize}
\end{itemize}
    
\end{frame}

\begin{frame}{BLP -- O lado da Oferta}
\begin{itemize}
    \item Para entendermos isso, vamos supor que tenhamos empresas multiproduto competindo à la Bertrand com produtos diferenciados.
    \[
    \pi_{f}=\sum_{i \in \mathcal{J}_{f}}(p_{i}-c_{i})q_{i}(\mathbf{p})
    \]
    \item Temos as seguintes condições de primeira ordem:
    \[
    s_{j}(\mathbf{p},\mathbf{x},\beta,\xi)+\sum_{i \in \mathcal{J}_{f}}(p_{i}-c_{i})\frac{\partial s_{j}(\mathbf{p}, \mathbf{x},\beta,\xi)}{\partial p_{i}}=0
    \]
\end{itemize}
    
\end{frame}


\begin{frame}{BLP -- O lado da Oferta}
\begin{itemize}
    \item Podemos representar esse sistema de condições de primeira ordem como sendo:
    \[
    \mathbf{s}(\mathbf{p},\mathbf{x},\beta,\xi)+\Delta (\mathbf{p}-\mathbf{c})=\mathbf{0}
    \]
    \item Reorganizando, temos a seguinte expressão para o custo marginal:
    \[
    \mathbf{c}=\mathbf{p}-\Delta^{-1}\mathbf{s}
    \]
    \item Podemos construir um conjunto de condições de momento do lado da oferta do seguinte tipo:
    \[
    \mathbf{p}-\Delta^{-1}\mathbf{s}-\gamma \mathbf{w}=\xi_{supply}
    \]
    \item Estas condições de momento adicionais implicam restrições entre equações dos coeficientes e ajudam (bastante!) na identificação dos coeficientes do lado da oferta.
\end{itemize}
    
\end{frame}
\begin{frame}{``Instrumentos BLP'' e ``Instrumentos BST''}
\begin{itemize}
    \item A premissa básica por trás dos Instrumentos BLP é a ideia que as características não observadas pelo econometrista tem média zero em relação às características observadas.
    \[
    E(\xi|\mathbf{x})=0
    \]
    \item Intuição: características que não os preços são determinadas previamente à competição de Bertrand do slide anterior.
    \item Com isso, temos que os instrumentos para os preços são somas das características (não o preço) dos outros produtos da mesma empresa, ou os produtos de outras empresas no mesmo mercado.
    \item \citet{Bresnahan1997}: Extensão dos ``Instrumentos BLP'' por ninhos.
\end{itemize}
    
\end{frame}

\begin{frame}{Problemas com Instrumentos BLP}
\begin{itemize}
    \item Podemos ter pouca variação nos dados para identificar as coisas
    \item Os instrumentos BLP podem ser fracos
    \item \citet{Berry2014}: mesmo que não tivéssemos preço, teríamos endogeneidade por conta dos RC.
    \item Ou seja, para isso funcionar, precisamos de \textbf{Instrumentos Fortes}.
    \item A literatura trabalha com a definição de instrumentos ótimos (paper clássico é o de \citet{Chamberlain1987})
\end{itemize}
    
\end{frame}

\begin{frame}{Instrumentos ótimos}
\begin{itemize}
    \item Problemas comuns em estatística e econometria lidam com modelos de momentos condicionais, que satisfaçam relações de regressão da forma $E[Y|X]=\mu (X,\beta_{0})$, $E[Y|X]=\mu (X,\beta_{0})$ em que  $E[\varepsilon |X]=0$.
    \item Esta premissa de média zero $E[\varepsilon |X]=0$ implica que podemos estimar estes modelos usando um conjunto infinito dimensional de restrições não condicionais de momento, pois ela implica que o seguinte também vale:
    \[
    E(h(X)(Y-X\beta_{0}))=0
    \]
    \item Existem infinitas funções que satisfazem estas restrições condicionais de momento.
    \item Só que vamos buscar funções $h(X)$ que sejam ótimas no sentido de permitir identificação dos parâmetros.

\end{itemize}
    
\end{frame}

\begin{frame}{Gandhi e Houde}
\begin{itemize}
    \item Em um paper bastante influente, \citet{Gandhi2015} mostram que podemos construir instrumentos ótimos que são parecidos com os de BLP
    \item \textit{Funções Base}
\end{itemize}
\begin{align}
    \sum_{j^{\prime} \neq j} \mathbf{x}_{jt}\\
    \sum_{j^{\prime} \neq j} (\mathbf{d}_{j^{\prime} t j})^{2}\\
    \sum_{j^{\prime} \neq j} \mathbf{1}(|\mathbf{d}_{j^{\prime} t j}|<\kappa)(\mathbf{d}_{j^{\prime} t j})
\end{align}
    
\end{frame}

\begin{frame}{Reynaert e Verboven}
\begin{itemize}
    \item Uma proposta interessante, ainda na mesma linha de ação de instrumentos ótimos de \citet{Chamberlain1987}, é a de \citet{Reynaert2014a}.
    \item Eles propõem instrumentos ótimos baseados nas seguintes condições de momento:
    
\end{itemize}
    \begin{align*}
        E\left[\frac{\partial \xi}{\partial \beta} | \mathbf{z} \right] \\
        E\left[\frac{\partial \xi_{supply}}{\partial \beta} | \mathbf{z} \right]
    \end{align*}
\begin{itemize}
    \item Note-se que estas são funções bem complicadas dos RC.
\end{itemize}
    
\end{frame}

\begin{frame}{Micromomentos}
\begin{itemize}
    \item Um outro caminho é a adição dos chamados ``Micromomentos''
    \item Alguma outra função dos dados e dos RC que possa ser mapeada nos dados.
    \item Exemplos são \citet{Berry2004}, e \citet{Petrin2002}.
    \item No primeiro dos papers, os autores conseguem microdados sobre quais seriam as segundas opções de carro para os entrevistados.
    \item Com as probabilidades estimadas, é possível calcular qual seria a proporção de consumidores que escolheriam o carro $Y$ como segunda opção.
\end{itemize}
    
\end{frame}



%\begin{comment}
\begin{frame}[allowframebreaks]
\bibliographystyle{aea}
\bibliography{references}

\end{frame}


%\end{comment}




\end{document}
