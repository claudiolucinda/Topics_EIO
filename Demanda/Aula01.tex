\documentclass{beamer}
\usepackage{beamerthemesplit}
\usepackage[brazil]{babel}
\usepackage{epsfig}
\usepackage[utf8x]{inputenc}
\usepackage{pgf}
%\usepackage{tikz}
%\usetikzlibrary{snakes}
\usepackage{nicefrac}
\usepackage{amsfonts}
\usepackage{amsmath}
\usepackage{amssymb}
\usepackage{amsthm}
%\usepackage{float}
\usetheme{Frankfurt}
\usepackage{epstopdf}
\usepackage{comment}
\usepackage{natbib}
\usepackage{float}
\usepackage{graphicx}
\usepackage{booktabs}
\usepackage{array}
\title{Aula 01}

\subtitle{Introdução ao Curso e Modelos Neoclássicos de Demanda}

\author{Claudio R. Lucinda}


\institute{FEA-RP/USP}

\date{}
\logo{\includegraphics[scale=.1]{logousp.png}}
\beamertemplatenavigationsymbolsempty
\begin{document}


\frame{\titlepage}
\begin{frame}{Agenda}
	\tableofcontents[pausesections]
\end{frame}

\section{Introdução ao Curso e Econometria Estrutural}

\subsection{Modelagem Econométrica Estrutural}

\begin{frame}[fragile]\frametitle{Abordagens à Modelagem Econométrica}
\begin{itemize}
  \item Podemos dividir a modelagem econométrica em duas linhas: descritiva e estrutural. 
  \item Vamos entender a diferença entre as duas imaginando a distribuição conjunta entre as variáveis cuja relação se busca entender, $f(x,y)$.
  \item Coisas específicas que se desejam caracterizar:
  \begin{itemize}
    \item A distribuição condicional de $y$ dado $x$, $f(y|x)$;
    \item A Esperança condicional de $y$ dado $x$, $E(y|x)$;
    \item A Correlação (ou covariância) condicional de $y$ dado $x$, $Corr(y|x)$ ou $Cov(y|x)$
    \item Um quantil específico $\alpha$ da distribuição de $y$ dado $x$ $Q_{\alpha}(y|x)$;
    \item O Melhor Preditor Linear de $y$ dado um valor para $x$ $BLP(y|x)$
  \end{itemize}
\end{itemize}
\end{frame}

\begin{frame}[fragile]\frametitle{Econometria Estrutural e Econometria Descritiva}
\small
  \begin{itemize}
      \item Nos modelos descritivos a idéia principal é caracterizar simplesmente a distribuição conjunta. 
      \item Na abordagem econométrica estrutural buscam-se parêmetros ou primitivas econômicas da distribuição conjunta
      \item Note-se que \textbf{a busca destas primitivas ou parâmetros da distribuição conjunta é sempre dependente destas
premissas que limitam a distribuição conjunta.} 
      \item Os elementos essenciais de um modelo estrutural são as hipóteses econômicas e estatísticas, as quais deveriam ser, pelo menos, razoáveis econômica e estatisticamente. 
      \item Note-se: mesmo que você não derive explicitamente um modelo estrutural, \textbf{qualquer conclusão de ordem causal ou
comportamental está implicitamente se baseando em um modelo estrutural}.
    \end{itemize}    
\end{frame}

\section{Modelos Neoclássicos de Demanda}

\begin{frame}[fragile]\frametitle{Abordagem Neoclássica de Demanda}
\begin{itemize}
  \item Modelagem de demanda baseada na escolha de uma cesta de bens.
  \item Usualmente se estipula uma forma funcional flexível que permite que os coeficientes estimados gerem um sistema de equações que caracteriza os ``parâmetros estruturais'' (usualmente derivadas) que você busca saber.
  \item Dois caminhos
  \begin{itemize}
    \item Abordagem Diferencial
    \item Abordagem da Função Dispêndio/Função Utilidade Indireta
  \end{itemize}
  \item De qualquer maneira, existem algumas propriedades que o sistema estimado precisa atender para que ele seja conforme com a teoria econômica
\end{itemize}
    


\end{frame}

\begin{frame}\frametitle{Propriedades Desejáveis de um Sistema de Demanda}
\small
\begin{itemize}
\item Em geral, a base para este tipo de características é o conceito de um sistema de demanda ``teoricamente plausível'' -- consistente com o processo de maximização da utilidade do consumidor. 
\item Em especial, este conceito pode ser operacionalizado verificando-se
as seguintes condições:

\begin{itemize}
\item \emph{Adding-up} (ou exaustão da restrição orçamentária): supõe-se
que o valor das demandas por todos os bens exaure o valor da restrição
orçamentária. 
\[
\sum_{k}p_{k}h_{k}=\sum_{k}p_{k}x_{k}=w
\]
\item Homogeneidade: as demandas hicksianas são homogêneas de grau zero
nos preços, e as demandas Marshallianas no gasto total e nos preços,
ou seja, para escalar $\theta>0$, 
\[
h_{i}(u,\theta\mathbf{p})=h_{i}(u,\mathbf{p})=x_{i}(\theta w,\theta\mathbf{p})=x_{i}(w,\mathbf{p})
\]
\end{itemize}
\end{itemize}
\end{frame}

\begin{frame}\frametitle{Propriedades Desejáveis (cont.)}
\begin{itemize}
\item Simetria: As derivadas cruzadas das demandas Hicksianas são simétricas.
\item Negatividade: A matriz de derivadas $\nabla_{p}h(u,\mathbf{p})$ das demandas hicksianas com relação aos preços tem que ser negativa semidefinida. Esta propriedade pode ser testada por meio da nossa querida Equação de Slutsky.
\item Nem todos os sistemas geralmente utilizados pela literatura são consistentes
com estas hipóteses.
\item Exemplo: Sistema de demanda log-linear supondo $i\in N$ produtos:
\[
\ln q_{i}=\alpha_{i}+e_{i}\ln w+\sum_{k}e_{ik}\ln p_{k}+u_{i}
\]
\end{itemize}
\end{frame}

\begin{frame}\frametitle{Sistema de demanda log-linear:}

\begin{itemize}
\item Esta função duplo log é muito comumente utilizada porque os coeficientes
estimados nos dão diretamente as elasticidades. 
\item No entanto, ela coloca problemas nos valores das elasticidades e da
exaustão da restrição orçamentária. Para entender isso melhor, vamos
definir o logaritmo da participação no gasto como sendo $\ln s_{i}=\ln q_{i}+\ln p_{i}-\ln w$.
Substituindo isso na equação acima, temos que:
\[
\ln q_{i}=\alpha_{i}+(e_{i}-1)\ln w+(e_{ii}+1)\ln p_{i}+\sum_{k\neq i}e_{ik}\ln p_{k}
\]
\item Pela restrição de exaustão da restrição orçamentária, mencionada acima,
temos que $\sum_{k}w_{k}e_{k}=1$, o que indica que \emph{ou tereemos
todos as elasticidades renda iguais a um ou pelo menos uma delas tem
que ser maior do que um}. 
\end{itemize}
\end{frame}

\subsection{Abordagem Diferencial e o Modelo de Rotterdam}
\begin{frame}\frametitle{Modelo de Rotterdam}

\begin{itemize}
\item Uma alternativa de modelagem empírica de demanda envolve aproximar diretamente a função demanda resultante do processo de maximização da utilidade do consumidor, que é o resultado do trabalho de Theil, como apresentado por \citet{Barnett2008}.
\item Desta forma, a equação fica sendo:
\[
s_{l}d\log x_{l}=\theta_{l}d\log\mathbf{Q}+\sum_{j=1}^{L}v_{ij}\left(d\log p_{j}-d\log P^{f}\right)
\]
\item Uma versão alternativa chamada versão em preços relativos desta equação é dada por:
\[
s_{l}d\log x_{l}=\theta_{l}d\log\mathbf{Q}+\sum_{j=1}^{L}\pi_{ij}d\log p_{j}
\]
\end{itemize}
\end{frame}

\begin{frame}\frametitle{Modelo de Rotterdam (II):}

Para impormos as restrições tradicionais, precisamos que os coeficientes
atendam às seguintes restrições:
\begin{itemize}
\item \emph{Adding-Up}: $\sum_{j}\theta_{j}=1$ e $\sum_{l}\pi_{lj}=0$,
para todos os $l$
\item Homogeneidade: $\sum_{j}\pi_{lj}=0$, em uma mesma equação
\item Simetria da matriz de Slutsky: $\pi_{ij}=\pi_{ji}$
\item Concavidade: a matriz de Slutsky precisa ser negativa semidefinida
com posto $L-1$.
\end{itemize}
\end{frame}

\begin{frame}\frametitle{Modelo de Rotterdam (III):}

As elasticidades preço compensadas e elasticidade renda são:
\begin{eqnarray*}
\epsilon_{ij} & = & \frac{\pi_{ij}}{s_{i}}\\
\epsilon_{w} & = & \frac{\theta_{l}}{s_{l}}
\end{eqnarray*}

\end{frame}

\subsection{Linear Expenditure System}
\selectlanguage{brazil}%
\begin{frame}\frametitle{LES}

\begin{itemize}
\small
\item Começaremos pelo \emph{Linear Expenditure System}. Este modelo é de
\citet{Klein1947}, e começa com a seguinte
função de utilidade indireta:
\[
v(\mathbf{P},w)=\frac{w-\sum p_{k}b_{k}}{\Pi_{k}p_{k}^{a_{k}}}
\]
\item Usando a Identidade de Roy, chegamos às seguintes formas funcionais
para as equações:
\[
s_{i}=\frac{p_{i}b_{i}}{w}+a_{i}\left[1-\frac{\sum_{k}p_{k}b_{k}}{w}\right]
\]
\item O legal deste modelo é que os parâmetros possuem interpretações comportamentais.
Uma família cujo sistema de demanda é LES começa comprando quantidades
``comprometidas'' de cada um dos bens $(b_{1},b_{2},\cdots,b_{n})$,
e depois dividindo o excedente, $w-\sum_{k}p_{k}b_{k}$ entre os bens
em proporções fixas $(a_{1},a_{2},\cdots,a_{n})$. 
\end{itemize}
\end{frame}

\begin{frame}\frametitle{LES - Elasticidades}

\begin{itemize}
\item As elasticidades deste sistema de equações são dadas por:
\begin{eqnarray*}
e_{ii} & = & \frac{p_{i}b_{i}(1-a_{i})}{p_{i}b_{i}+a_{i}\left(w-\sum_{k}p_{k}b_{k}\right)}-1\\
e_{ij} & = & \frac{-a_{i}b_{j}p_{j}}{p_{i}b_{i}+a_{i}\left(w-\sum_{k}p_{k}b_{k}\right)}\\
e_{w} & = & \frac{a_{i}w}{p_{i}b_{i}+a_{i}\left(w-\sum_{k}p_{k}b_{k}\right)}
\end{eqnarray*}
\end{itemize}
\end{frame}

\subsection{Translog}

\begin{frame}\frametitle{Translog}

\begin{itemize}
\item O paper de \citet{ChristensenTLUtil75} partem da seguinte função de utilidade indireta:
\[
\ln(v(\mathbf{P},w))=\alpha_{0}+\sum\alpha_{i}\ln\frac{p_{i}}{w}+\frac{1}{2}\sum\sum\beta_{ij}\ln\frac{p_{i}}{w}\ln\frac{p_{j}}{w}
\]
\item A vantagem desta função de utilidade indireta é que ela aproxima os valores das primeiras e segundas derivadas da função ``verdadeira'' de utilidade indireta em torno da média amostral dos dados.
\item Usando a nossa querida Identidade de Roy, eles chegam no seguinte sistema:
\begin{eqnarray*}
s_{i} & = & \frac{\alpha_{j}+\sum\beta_{ji}\ln\frac{p_{i}}{w}}{\alpha_{M}+\sum\beta_{Mi}\ln\frac{p_{i}}{w}}\\
\alpha_{M} & = & \sum\alpha_{k}\\
\beta_{Mi} & = & \sum\beta_{ki}
\end{eqnarray*}
\end{itemize}
\end{frame}

\begin{frame}\frametitle{Translog Continuação}

\begin{itemize}
\item Normalizando $\alpha_{M}$ para ser igual a -1. Vamos calcular as elasticidades, e para isso iremos fazer a seguinte definição:
\begin{eqnarray*}
\mathbf{A} & = & \alpha_{j}+\sum\beta_{ji}\ln\frac{p_{j}}{w}\\
\mathbf{B} & = & \alpha_{M}+\sum\beta_{Mi}\ln\frac{p_{j}}{w}
\end{eqnarray*}
\item Com isto, podemos definir a quantidade demandada como sendo:
\[
x_{i}=\frac{w}{p_{i}}\left[\frac{\mathbf{A}}{\mathbf{B}}\right]
\]
\end{itemize}
\end{frame}

\begin{frame}\frametitle{Translog -- Elasticidades}

\begin{itemize}
\item Elasticidade-Cruzada:
\[
e_{ij}=\left[\frac{\beta_{ji}}{\mathbf{A}}-\frac{\beta_{Mj}}{\mathbf{B}}\right]
\]
\item Elasticidade-Preço:
\[
e_{ii}=\left[\frac{\beta_{ji}}{\mathbf{A}}-\frac{\beta_{Mj}}{\mathbf{B}}\right]-1
\]
\item Elasticidade-Renda:
\[
e_{w}=\left[-\frac{\sum\beta_{ji}}{\mathbf{A}}+\frac{\sum\beta_{Mj}}{\mathbf{B}}\right]+1
\]
\end{itemize}
\end{frame}

\subsection{AIDS}

\begin{frame}\frametitle{AIDS (Almost Ideal Demand System)}

\begin{itemize}
\item Este modelo foi apresentado por \citet{Deaton1980} se baseia na seguinte função utilidade indireta:
\[
v(\mathbf{P},w)=G(\mathbf{P})[\ln w-\ln g(\mathbf{P})]
\]
\item Sendo que a função $G(\mathbf{P})$ é homogênea de grau zero nos preços,
e a $g(\mathbf{P})$ é homogênea de grau 1. 
\item A classe geral deste tipo de função de utilidade indireta é denominada
PIGLOG (``Price Independent Generalized Linearity'', PIGL em forma
Logaritmica). 
\item Na verdade, esta condição se relaciona com a relação entre os preços
relativos e a curva de Engel. No caso específico da demanda AIDS,
temos que:
\begin{eqnarray*}
G(\mathbf{P}) & = & \Pi_{k}p_{k}^{-\gamma_{k}}\\
\ln g(\mathbf{P}) & = & \alpha_{0}+\sum\alpha_{k}\ln p_{K}+\frac{1}{2}\sum_{k}\sum_{j}\beta_{kj}\ln p_{k}\ln p_{j}
\end{eqnarray*}
\end{itemize}
\end{frame}

\begin{frame}\frametitle{AIDS -- Continuação:}
\small
\begin{itemize}
\item Aplicando a nossa amiga, a Identidade de Roy, nesta função de utilidade
indireta e cozinhando vigorosamente, temos a seguinte forma para a
equação demanda pelo produto na forma de \emph{share de consumo}:
\[
s_{i}=\alpha_{i}+\sum_{i}\beta_{ki}\ln p_{k}+\gamma_{i}\ln\left(\frac{w}{g(\mathbf{P})}\right)
\]
\item Deaton e Muellbauer, no seu paper da \emph{AER}, mencionam que uma
alternativa quando os preços dos diferentes produtos são muito colineares,
é a utilização do seguinte índice de preços de \citet{stone1954measurement} no lugar
da função $g(\mathbf{P})$:
\[
\mathbf{P}^{*}=\sum_{k}\bar{s}_{k}\ln p_{k}
\]
\item Em que $\bar{s}$ seria a média das participações de
mercado. Outra vantagem desta aproximação (conhecida por LA-AIDS)
é que a estimação do sistema de equações envolve apenas equações lineares,
o que facilita a implementação computacional do modelo. 
\end{itemize}
\end{frame}

\begin{frame}\frametitle{AIDS -- Elasticidades:}
\small
\begin{itemize}
\item As elasticidades preço e cruzadas do modelo são da seguinte forma:
\begin{eqnarray*}
e_{ii} & = & \frac{\beta_{ii}-\gamma_{i}s_{i}}{s_{i}}-1\\
e_{ij} & = & \frac{\beta_{ij}-\gamma_{i}s_{j}}{s_{i}}\\
e_{w} & = & 1+\frac{\gamma_{i}}{s_{i}}
\end{eqnarray*}
\item Caso não seja adotada a linearização do índice de preços, as elasticidades-preço
assumem uma forma um pouco mais complexa.
\begin{eqnarray*}
e_{ii} & = & \frac{\beta_{ii}-\gamma_{i}s_{i}+\gamma_{i}^{2}\ln\left(\frac{w}{g(\mathbf{P})}\right)}{s_{i}}-1\\
e_{ij} & = & \frac{\beta_{ii}-\gamma_{i}s_{i}+\gamma_{i}\gamma_{j}\ln\left(\frac{w}{g(\mathbf{P})}\right)}{s_{i}}
\end{eqnarray*}
\end{itemize}
\end{frame}

\begin{frame}[fragile]\frametitle{Restrições}
\small
\begin{itemize}
\item Simetria: Precisamos que os termos $\beta$ cruzados sejam iguais, ou:
\[
\beta_{ij}=\beta_{ji}
\]
\item Adding-up: Esta premissa também permite que recuperarmos os coeficientes da última equação, mesmo ela não
estimada pelo fato das participações no gasto necessariamente somarem 1.:
\begin{eqnarray*}
\sum_{i}\alpha_{i} & = & 1\\
\sum_{i}\beta_{ij} & = & 0\\
\sum_{i}\gamma_{i} & = & 0
\end{eqnarray*}
\item Homogeneidade:
\[
\sum_{j}\beta_{ij}=0
\]
 \end{itemize}   


\end{frame}

\subsection{QUAIDS}
\begin{frame}[fragile]\frametitle{QUAIDS}
\begin{itemize}
  \item Este modelo foi proposto por \citet{Banks2007}, permitindo que as curvas de Engel (relacionando share do bem e log da renda) não sejam lineares.
  \item Basicamente eles fazem isso por meio de uma extensão quadrática do modelo AIDS:
\end{itemize}
\begin{eqnarray*}
ln(a(\mathbf{p})) &= \alpha_{0}+\sum_{i=1}^{n} \alpha_{i} lnp_{i}+\frac{1}{2}\sum_{i=1}^{n} \sum_{j=1}^{n} \gamma_{ij} lnp_{i} lnp_{j}\\
b(\mathbf{p})&=\prod_{i=1}^{n}p_{i}^{\beta_{i}}\\
s_{i}&=\alpha_{i}+\sum_{j=1}^{n}\gamma_{ij}lnp_{j}+\beta_{i}ln \left[ \frac{w}{a(\mathbf{p})}\right]+\frac{\lambda_{i}}{b(\mathbf{p})}\left[ ln \left[\frac{w}{a(\mathbf{p})}  \right] \right] ^{2}
\end{eqnarray*}
\end{frame}

\begin{frame}[fragile]\frametitle{QUAIDS -- Elasticidades}
\begin{align*}    
\mu_{i}&=\frac{\partial s_{i}}{\partial ln w}=\beta_{i}+\frac{2\lambda_{i}}{b(\mathbf{p})} \left[ln \left[ \frac{w}{a(\mathbf{p})}  \right]   \right]\\
\mu_{ij}&=\frac{\partial s_{i}}{\partial ln p_{j}}=\gamma_{ij}-\mu_{i}\left(\alpha_{j} +\sum_{k} \gamma_{jk} lnp_{k}  \right) - \frac{\lambda_{i} \beta_{j}}{b(\mathbf{p})} \left[ ln \left( \frac{w}{a(\mathbf{p})} \right) \right]^{2}\\
e_{w}^{u}&=\frac{\mu_{i}}{s_{i}}+1\\
e_{ij}^{u}&=\frac{\mu_{ij}}{s{i}}+\mathbf{1}(i=j)
\end{align*}
\end{frame}

\subsection{EASI}
\begin{frame}[fragile]\frametitle{\textit{Exact Affine Stone Index}}
\begin{itemize}
  \item O objetivo deste modelo, apresentado no artigo de \citet{Lewbel2009}, é construir sistemas de demanda que:
  \begin{itemize}
    \item Possuem respostas flexíveis aos preços (ou seja, deixa os coeficientes estimados aproximarem as derivadas relevantes);
    \item Têm curvas de Engel de qualquer forma (não linear que nem o AIDS ou quadrática que nem o QUAIDS)
    \item E os erros da equação são parâmetros aleatórios da utilidade que podem ser incorporados na parte da utilidade do consumidor.
  \end{itemize}
  \item Notação:
  \begin{itemize}
    \item $x$ - Log de $w$
    \item $\mathbf{p}$ - Vetor de Log de preços, de dimensão $J\times 1$
    \item $\mathbf{z}$ - Vetor de Demographics, de dimensão coluna $L$
  \end{itemize}
   
\end{itemize}
    


\end{frame}

\begin{frame}[fragile]\frametitle{EASI}
\begin{itemize}
  \item Versão aproximada:
  \item Seja $\tilde{y}=x-\mathbf{p}^{\prime} \bar{\mathbf{s}}$
  \item Temos então um sistema de equações, com $\mathbf{b}^{r}$, $\mathbf{C}$, $\mathbf{D}$, $\mathbf{B}$ e $\mathbf{A_{l}}$ sendo matrizes de coeficientes:
\end{itemize}
\[
\mathbf{s}\approx \sum_{r=1}^{5} \mathbf{b}^{r} \tilde{y}^{r}+\mathbf{Cz}+\mathbf{Dz}\tilde{y} + \sum_{l=1}^{L} z_{l} \mathbf{A_{l}p}+ \mathbf{Bp}\tilde{y}+\varepsilon
\]
\begin{itemize}
  \item Versão Completa
  \item Seja $y=\frac{x-\mathbf{p}^{\prime}\mathbf{s}+\sum_{l=1}^{L}z_{l}\mathbf{p^{\prime}A_{l}p}/2}{1-\mathbf{p^{\prime}Bp}}$
\end{itemize}
\[
\mathbf{s}= \sum_{r=1}^{5} \mathbf{b}^{r} y^{r}+\mathbf{Cz}+\mathbf{Dz}y + \sum_{l=1}^{L} z_{l} \mathbf{A_{l}p}+ \mathbf{Bp}y+\varepsilon
\]
    


\end{frame}

\begin{frame}[fragile]\frametitle{Elasticidades}
\begin{itemize}
  \item Semielasticidade-preço compensadas:
  \[
  \nabla_{\mathbf{p}}\mathbf{s}=\sum_{l=1}^{L}z_{l}\mathbf{A_{l}} + \mathbf{B}y
  \]
  \item Semielasticidade-renda compensada:
  \[
  \nabla_{\mathbf{y}}\mathbf{s}=\sum_{r=1}^{5} \mathbf{b}^{r} y^{r-1} r + \mathbf{Dz} + \mathbf{Bp}
  \]
  \item As elasticidades-preço são obtidas por meio da divisão desses valores pelos shares (e no caso das elasticidades-preço próprias, subtraindo um).
\end{itemize}
    


\end{frame}


%\begin{comment}
\begin{frame}[allowframebreaks]
\bibliographystyle{aea}
\bibliography{C:/Bibliog/library}

\end{frame}

%\end{comment}


\end{document}


