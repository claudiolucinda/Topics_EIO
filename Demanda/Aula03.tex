\documentclass{beamer}
\usepackage{beamerthemesplit}
\usepackage[brazil]{babel}
\usepackage{epsfig}
\usepackage[utf8x]{inputenc}
\usepackage{pgf}
%\usepackage{tikz}
%\usetikzlibrary{snakes}
\usepackage{nicefrac}
\usepackage{amsfonts}
\usepackage{amsmath}
\usepackage{amssymb}
\usepackage{amsthm}
%\usepackage{float}
\usetheme{Frankfurt}
\usepackage{epstopdf}
\usepackage{comment}
\usepackage{natbib}
\usepackage{float}
\usepackage{graphicx}
\usepackage{booktabs}
\usepackage{array}
\title{Aula 03}

\subtitle{Modelos de Escolha Discreta com Dados Agregados}

\author{Claudio R. Lucinda}


\institute{FEA-RP/USP}

\date{}
\logo{\includegraphics[scale=.1]{logousp.png}}
\beamertemplatenavigationsymbolsempty
\begin{document}

\frame{\titlepage}
\begin{frame}{Agenda}
  \tableofcontents[pausesections]
\end{frame}


\section{Modelos de Escolha Discreta com Dados Agregados}
\begin{frame}{Modelos de Escolha Discreta com Dados Agregados:}
\begin{itemize}
\item Agora iremos discutir um pouco mais como podemos utilizar os modelos
de escolha discreta, quando apenas temos dados agregados de mercado
-- ou seja, não temos microdados com as escolhas dos indivíduos.
\item Supondo que os indivíduos ajam de acordo com os modelos de escolha
discreta colocados anteriormente, temos que a quantidade vendida em
um determinado mercado é dada por:
\[
q_{j}=M\times s_{j}
\]
\end{itemize}
\end{frame}

\begin{frame}\frametitle{Escolha Discreta com Dados Agregados: O bem externo}

\begin{itemize}
\item Além disso, tem um ponto adicional, que é a existência de um bem ``externo'',
cujo preço não é estabelecido em resposta aos preços dos $J$ produtos. 
\item Caso não fizéssemos isto, os consumidores seriam forçados a escolher
apenas entre os bens existentes e a demanda dependeria somente das
diferenças de preços. Portanto, um aumento generalizado dos preços
não reduzirá a produção agregada. 
\item A colocação deste produto externo torna necessário que estimemos o
tamanho do mercado, uma vez que $s_{0}=M\times(1-\sum^{J}s_{j})$
\end{itemize}
\end{frame}

\begin{frame}\frametitle{Derivando as Especificações -- Modelo Logit}

\begin{itemize}
\item Supondo que os consumidores funcionem como no modelo Logit Multinomial
tradicional, temos que
\[
s_{j}=M\times\frac{e^{V_{j}}}{1+\sum_{k}e^{V_{k}}}
\]
\item Passando o Log dos dois lados:
\[
\ln s_{j}=\ln M+V_{j}+\ln(1+\sum_{k}e^{V_{k}})
\]
\item O bem externo, por sua vez, tem a sua participação de mercado dada
por:
\[
s_{0}=M\times\frac{1}{1+\sum_{k}e^{V_{k}}}
\]
\item Passando o log também, temos que:
\[
\ln s_{0}=\ln M+\ln(1+\sum_{k}e^{V_{k}})
\]
\end{itemize}
\end{frame}

\begin{frame}\frametitle{Especificações -- Modelo Logit (II):}

\begin{itemize}
\item Tirando a diferença das duas equações, temos que:
\[
\ln s_{j}-\ln s_{0}=V_{j}+\varepsilon_{j}
\]
\item Em que $\varepsilon_{j}$ seria o termo erro econométrico.
\item Esta especificação poderia ser estimada -- mas existem
alguns problemas que iremos discutir mais adiante:

\begin{itemize}
\item IIA
\item Características não observadas dos consumidores
\item Endogeneidade
\end{itemize}
\item Com relação ao IIA, o slide seguinte mostra uma possível solução --
o modelo logit aninhado
\end{itemize}
\end{frame}

\begin{frame}\frametitle{Especificações -- Modelo Nested Logit}

\begin{itemize}
\item Seguindo \citet{Berry1994}, temos que a especificação de teste
é a seguinte:
\[
\ln s_{j}-\ln s_{0}=V_{j}+(1-\sigma)\ln s_{j|K}+\varepsilon_{j}
\]
\item Mais uma vez, o termo $\varepsilon_{j}$ representa o termo erro econométrico.
\item CUIDADO!!! Na demonstração do Berry, o $\sigma$ dele é exatamente
igual ao meu $1-\sigma$.
\item Caso $\sigma\rightarrow1$, temos que esta especificação colapsa para
o Logit Multinomial tradicional.
\item Note que o ``aninhamento'' das alternativas é algo intrinsecamente
determinado pelo analista; é ele quem coloca os produtos nos diferentes
ninhos.
\end{itemize}
\end{frame}

\begin{frame}\frametitle{Identificação nestes Modelos}

\begin{itemize}
\item Quando estamos falando de dados agregados, uma questão fundamental
é a identificação.
\item Diferentemente do caso em que tínhamos microdados, em que poderia
se assumir que as características dos produtos eram exógenas à escolha
de cada um dos produtos, quando estamos falando em dados agregados,
com certeza teremos alguma endogeneidade entre as características
dos produtos presentes no $V_{j}$ e o termo $\varepsilon_{j}$.
\item Para resolver isto, a melhor forma é o uso de Variáveis Instrumentais.
\item \citet{Berry1994} sugere os seguintes instrumentos:

\begin{itemize}
\item Variáveis de custos dos produtos que não entram em $V_{j}$, tais
como preço de insumos
\item A soma das características dos outros produtos da mesma empresa
\item A soma das características dos produtos das outras empresas.
\end{itemize}
\end{itemize}

\end{frame}

\section{Modelo de Escolha Discreta com Características Não-Observáveis}
\begin{frame}\frametitle{Escolha Discreta e Características Não Observáveis}

\begin{itemize}
\item Vamos aqui discutir uma outra forma de driblar o problema decorrente
da IIA.
\item Neste caso, a sensibilidade da utilidade à atributos das alternativas
depende de características individuais.
\item A vantagem é que prescinde de imposições do analista acerca do padrão
de substituição entre as alternativas.
\item No Logit Aninhado, quem tem que escrever a árvore de escolhas é o
analista.
\item Exemplos bons dessa abordagem são encontrados em \citet{Verboven1996} e \citet{Goldberg1998}
\end{itemize}
\end{frame}

\begin{frame}\frametitle{Escolha Discreta...(II):}

\begin{itemize}
\item Essa abordagem é apresentada no artigo clássico de \citet{Berry1995}	
\item Uma vez que é baseada em uma microfundamentação bem robusta, permite
que estendamos esta metodologia para um conjunto bem amplo de situações.
Vamos então começar introduzindo o modelo microeconômico subjacente:
\[
U_{ij}=\sum_{k}x_{jk}\beta_{ik}+\xi_{j}+\epsilon_{ij}
\]
\item O termo $\xi_{j}$ captura as avaliações que os consumidores fazem
das características não observaas da alternativa $j$. 
\item A diferença em relação ao que vimos na aula anterior é que voltamos
a supor que a sensibilidade do consumidor $i$ em relação ao atributo
$k$ do produto $j$ pode ser diferente da mesma sensibilidade no
caso de outro consumidor, ou seja $\beta_{ik}\neq\beta_{dk},\forall i,d$. 
\end{itemize}
\end{frame}

\begin{frame}\frametitle{Escolha Discreta...(III):}

\begin{itemize}
\item A idéia aqui é modelar a heterogeneidade dos coeficientes dos indivíduos
da seguinte forma:
\[
\beta_{ik}=\bar{\beta}_{k}+\beta_{k}^{o}\mathbf{z}_{i}+\beta_{k}^{u}\mathbf{v}_{i}
\]
\item Ou seja, este coeficiente é composto por uma sensibilidade ``média'',
$\bar{\beta}_{k}$, mais alguns efeitos:

\begin{itemize}
\item As características individuais observadas possui sobre ele , que denotaremos
$\mathbf{z}_{i}$
\item As características individuais \textbf{não observadas}, denominadas
$\mathbf{v}_{i}$
\end{itemize}
\item Substituindo esta definição dos coeficientes, temos que:
\[
U_{ij}=\sum_{k}x_{jk}\bar{\beta}_{k}+\sum_{k}x_{jk}\beta_{k}^{o}\mathbf{z}_{i}+\sum_{k}x_{jk}\beta_{k}^{u}\mathbf{v}_{i}+\xi_{j}+\epsilon_{ij}
\]
\end{itemize}
\end{frame}

\begin{frame}\frametitle{Escolha Discreta...(IV):}

\begin{itemize}
\item Podemos reescrever de forma similar à da aula passada, separando esta
especificação em uma parte correspondente á utilidade ``média''
da alternativa, e a parte aleatória:
\begin{eqnarray*}
U_{ij} & = & V_{j}+\sum_{k}x_{jk}\beta_{k}^{o}\mathbf{z}_{i}+\sum_{k}x_{jk}\beta_{k}^{u}\mathbf{v}_{i}+\epsilon_{ij}\\
V_{j} & = & \sum_{k}x_{jk}\bar{\beta}_{k}+\xi_{j}
\end{eqnarray*}
\item Na verdade, não temos microdados, somente as participações de mercado
e algumas características dos produtos.
\item Como fazer neste caso? 
\end{itemize}
\end{frame}

\begin{frame}\frametitle{Escolha Discreta...(V):}
\small
\begin{itemize}
\item Temos que a probabilidade de escolha de uma alternativa é, na verdade,
a seguinte integral:
\[
P_{ij}=\int\int_{\varepsilon}I(\epsilon_{ik}-\epsilon_{ij}<V_{ij}-V_{ik},\forall k\neq j)f(\varepsilon)d(\varepsilon)f(\mathbf{v})d(\mathbf{v})
\]
\item O problema não é a integral de dentro -- supondo uma distribuição
valores extermos para $\varepsilon$, temos uma forma analítica --
mas sim a integral de fora.
\item Na maior parte dos casos, não teremos dados observados sobre as características
dos consumidores que compraram cada uma das alternativas, de forma
que apenas trabalharemos com características não observadas dos consumidores
-- ou seja, apenas os termos $\mathbf{v}_{i}$. A análise
subseqüente será em três etapas.

\begin{enumerate}
\item Especificar as participações de mercado em função dos coeficientes;
\item Recuperar os sinais dos termos $\xi_{j}$ a partir dos resultados
da etapa anterior;
\item Estimar os coeficientes por GMM
\end{enumerate}
\end{itemize}
\end{frame}

\subsection{Estimando as Participações de Mercado}
\begin{frame}\frametitle{Estimando as Participações de Mercado}

\begin{itemize}
\item A primeira parte é recuperar as participações de mercado em função
da utilidade média, $V_{j}$, e dos coeficientes $\beta_{k}^{u}$.
Uma vez que os termos $\mathbf{v}_{i}$ não são observados, iremos
impor uma premissa sobre eles, que é a que eles seguem uma distribuição
qualquer, enquanto que os termos $\epsilon_{ij}$ seguem a já tradicional
distribuição de Valores Extremos I. Desta forma, temos que:
\[
s_{j}(V_{j},\beta)=\int\left(\frac{\exp[V_{j}+\sum_{k}x_{jk}\beta_{k}^{u}\mathbf{v}_{i}]}{1+\sum_{q>0}\exp[V_{q}+\sum_{k}x_{qk}\beta_{k}^{u}\mathbf{v}_{i}]}\right)f(\mathbf{v})d\mathbf{v}
\]
\item Para resolvermos esta integral, precisamos de uma forma funcional
para $f(\mathbf{v})$. 
\item Geralmente, utiliza-se dados obtidos a partir de microdados --
por exemplo, se a renda dos consumidores não é observada, usa-se distribuição
da renda buscada na PNAD.
\end{itemize}
\end{frame}

\begin{frame}\frametitle{Calculando Integrais por Simulação}

\begin{itemize}
\item Não existe uma forma analítica de resolução desta integral, de forma
que teremos que usar ou métodos numéricos (por exemplo, método da
quadratura) ou métodos de simulação para a obtenção do valor desta
integral -- e, consequentemente, o valor desta participação
de mercado. 
\item Vamos falar um pouco sobre os métodos de simulação.

\begin{itemize}
\item Em geral simulação consiste em sortear valores aleatórios de uma distribuição,
calcular alguma coisa com cada um destas valores sorteados e depois
tirar uma média destes cálculos. Em todos estes casos, o pesquisador
quer calcular uma média da forma $\tilde{t}=\int t(\varepsilon)f(\varepsilon)d\varepsilon$.
\end{itemize}
\item Vamos supor que você tenha sorteado $ns$ valores desta distribuição
$f(\mathbf{v})$. Vamos supor que $\mathbf{z}=\varnothing$.
\end{itemize}
\end{frame}

\begin{frame}\frametitle{Calculando Integrais por Simulação}

\begin{itemize}
\item Com estes valores, podemos calcular a participação de mercado da alternativa
$j$ com a seguinte fórmula:
\[
\hat{s}^{ns}(\delta_{j},\beta)=\frac{1}{ns}\sum_{r=1}^{ns}\left[\frac{\exp[V_{j}+\sum_{k}x_{jk}\beta_{k}^{u}\mathbf{v}_{ir}]}{1+\sum_{q>0}\exp[V_{q}+\sum_{k}x_{qk}\beta_{k}^{u}\mathbf{v}_{ir}]}\right]
\]
\item Ou seja, pegamos os valores de $\mathbf{v}_{i}$ associados com os
sorteios de cada característica, e calculamos a fórmula do \emph{share}
com cada um deles. Depois disso, somamos e tiramos a média. 
\item Evidentemente, a utilização de métodos de simulação aumenta a imprecisão
das estimativas, que pode ser reduzida quanto maior for o valor de
$ns$. 
\end{itemize}
\end{frame}

\subsection{Calculando a Utilidade Média}
\begin{frame}\frametitle{Calculando a Utilidade Média}

\begin{itemize}
\item O passo seguinte é calcular os efeitos das características dos produtos
que não são observadas pelo analista. Para isso, precisamos obter
estimativas do termo $V_{j}$. 
\item Tendo este negócio, podemos obter estimativas do $\xi_{j}$, do $\bar{\beta}_{j}$
e do $\beta_{k}^{u}$. 
\item Em primeiro lugar, no paper BLP, os autores notam que a seguinte relação:
\[
V_{j}^{h}=V_{j}^{h-1}+\ln[s_{j}]-\ln[\hat{s}_{j}^{ns}]
\]
\item É um chamado \emph{contraction mapping}, que possui um ponto fixo.
Ou seja, se fizermos um processo sequencial, começando com um valor
inicial para o $V_{j}$, e em cada iteração ajustando o valor de $V_{j}$
no valor igual à diferença entre os logs da participação de mercado
e a participação de mercado observada, acabaremos em um ponto fixo,
em que não há alterações adicionais em $V_{j}$. 
\end{itemize}
\end{frame}

\begin{frame}\frametitle{Calculando a Utilidade Média (II):}

\begin{itemize}
\item Um bom valor inicial para fazer a recursão pode ser a participação
de mercado obtida com a estimação de um modelo LOGIT multinomial. 
\item Na verdade, na programação é utilizada a seguinte versão não-linear
do estimador:
\[
\exp[V_{j}^{h}]=\exp[V_{j}^{h-1}]\times\frac{s_{j}}{\hat{s}_{j}^{ns}}
\]
\item Desta forma, obtemos uma estimativa para o termo $V_{j}$. A partir
desta estimativa, podemos construir as condições de momento, lembrando
que o termo $\xi_{j}$ pode ser entendida como a diferença entre a
utilidade média e os valores dos coeficientes médios. 
\end{itemize}
\end{frame}

\subsection{Construindo as Condições de Momento}
\begin{frame}\frametitle{Construindo as Condições de Momento}

\begin{itemize}
\item Tendo este valor para $V_{j}$, podemos definir os erros da seguinte
forma:
\[
\xi_{j}=V_{j}-\sum_{k}\bar{\beta}_{k}x_{jk}
\]
\selectlanguage{brazil}%
\item O problema é que aqui temos que estimar os $\bar{\beta}_{\mathbf{j}}$.
Isto é análogo a concentrar a função objetivo para ficar apenas em
termos dos $\mathbf{\beta^{u}}$. Vamos fazer isso projetando a covariância
entre os $\mathbf{x_{j}}$ e as variáveis instrumentais, $\mathbf{Z}$,
na covariância entre o $V_{j}$ e os instrumentos, ou seja:
\[
\bar{\beta}_{j}=\mathbf{(K(Z^{T}Z)K^{T})^{-1}K(Z^{T}Z)L}
\]
\end{itemize}
\selectlanguage{brazil}%

\note[item]{Em que $\mathbf{K=x_{j}^{T}Z}$ e $\mathbf{L}=V_{j}^{T}\mathbf{Z}$.
Implicitamente, esta conta significa que estamos expressando o $\bar{\beta}_{j}$
como função dos $\beta^{u}$. }

\end{frame}

\begin{frame}\frametitle{Construindo as Condições de Momento (II)}

\begin{itemize}
\item Supondo que tenhamos variáveis exógenas e instrumentos para os preços,
podemos construir as condições de momento da seguinte forma:
\[
m(\theta)=\Xi\mathbf{Z}
\]
\item Em que $\Xi$ é o empilhamento dos $\xi_{j}$. Com isto, podemos construir
uma função objetivo da seguinte forma:
\[
q=\Xi^{T}\mathbf{Z\Phi^{-1}Z}^{T}\Xi
\]
\item Em que $\Phi$ representa uma estimativa da matriz variância-covariância
dos momentos das equações, ou $\Phi=E(\mathbf{Z}^{T}\Xi\Xi^{T}\mathbf{Z})$. 
\end{itemize}

\note[item]{Um caminho razoavelmente adequado envolve, na primeira iteração,
calcular $\Phi$ como sendo homocedástica $\Phi=\mathbf{Z}^{T}\mathbf{Z}$. }

\note[item]{Com isto, são calculados os valores dos parâmetros e, com os valores
dos parâmetros, calcular aí uma estimativa completa da matriz de variância
covariância das condições de momento, $\Phi=\mathbf{Z}^{T}\Xi\Xi^{T}\mathbf{Z}$. }

\end{frame}

\begin{frame}\frametitle{Iterando até Convergir}

\begin{itemize}
\item Neste ponto, perdemos de vista os coeficientes $\beta_{k}^{u}$. Para
isto, precisamos ter em mente que as etapas anteriores são repetidas
todas ao longo de cada iteração. Na verdade, uma vez que, para os
valores de $\beta_{k}^{u}$ temos forma analítica para os coeficientes
$\bar{\beta_{k}}$, podemos fazer o seguinte; em cada iteração apenas
usar métodos numéricos para obter o mínimo valor da função critério
na iteração. Ou seja, em cada iteração temos o seguinte roteiro:

\begin{enumerate}
\item Dados os valores novos dos parâmetros $\beta_{k}^{u}$, podemos recalcular
todo o processo e calcular a função objetivo;
\item Pelo algoritmo numérico, temos novas sugestões para os valores $\beta_{k}^{u}$.
\end{enumerate}
\end{itemize}
\end{frame}

%\begin{comment}
\begin{frame}[allowframebreaks]
\bibliographystyle{aea}
\bibliography{C:/Bibliog/library}

\end{frame}

%\end{comment}




\end{document}


