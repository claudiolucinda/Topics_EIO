\documentclass{beamer}
\usepackage{beamerthemesplit}
\usepackage[brazil]{babel}
\usepackage{epsfig}
\usepackage[utf8x]{inputenc}
\usepackage{pgf}
%\usepackage{tikz}
%\usetikzlibrary{snakes}
\usepackage{nicefrac}
\usepackage{amsfonts}
\usepackage{amsmath}
\usepackage{amssymb}
\usepackage{amsthm}
%\usepackage{float}
\usetheme{Frankfurt}
\usepackage{epstopdf}
\usepackage{comment}
\usepackage{natbib}
\usepackage{float}
\usepackage{graphicx}
\usepackage{booktabs}
\usepackage{array}
\title{Aula 02}

\subtitle{Modelos de Escolha Discreta com Dados Desagregados}

\author{Claudio R. Lucinda}


\institute{FEA-RP/USP}

\date{}
\logo{\includegraphics[scale=.1]{logousp.png}}
\beamertemplatenavigationsymbolsempty
\begin{document}

\frame{\titlepage}
\begin{frame}{Agenda}
  \tableofcontents[pausesections]
\end{frame}


\section{Modelos de Escolha Discreta}

\begin{frame}\frametitle{Modelos de Escolha Discreta}


\begin{itemize}
\item Agora, iremos discutir os modelos em que a escolha se dá sobre o ``espaço
de características''; os produtos derivam utilidade apenas na medida
em que eles são agregados de características. 
\item Esta escolha no espaço de característica possui um elemento inerentemente
idiosincrático; este lado idiosincrático é o que permite a estimação
dos parâmetros.
\item Inicialmente começaremos analisando o processo de estimação quando
o analista possui dados sobre a escolha individual dos consumidores;
depois discutiremos as situações em que apenas possuimos dados agregados.
\item Uma bibliografia muito boa sobre esse assunto é 
\end{itemize}
\end{frame}

\begin{frame}\frametitle{Modelos de Escolha Discreta}


\begin{itemize}
\item O analista começará postulando uma função que relaciona estes dados
observados com a escolha do consumidor, que chamaremos de $V(x_{nj},s_{nj})$,
sendo que $x_{nj}$ representa as características observadas do produto
e $s_{nj}$ as características não observadas.
\item Uma vez que alguns aspectos da utilidade do consumidor não são observados,
em geral $V\neq U$, em que $U$ é a ``verdadeira'' utilidade do
consumidor. Desta forma, podemos fazer o seguinte ajuste:
\[
U_{ij}=V_{ij}+\epsilon_{ij}
\]
\item Em que $i$ denota o consumidor e $j$ a alternativa. A idéia é que
o termo $\epsilon_{ij}$ capture os aspectos do produto ou do indivíduo
que não são observados pelo econometrista. 
\item Dada esta definição, as características deste termo dependem fundamentalmente
de como o mesmo especifica $V_{ij}$. 
\end{itemize}
\end{frame}

\begin{frame}\frametitle{Modelos de Escolha Discreta II}


\begin{itemize}
\item No entanto, para que possamos estimar os componentes de $V_{ij}$,
precisamos do termo $\epsilon_{ij}$, e de uma distribuição conjunta
para os $\epsilon_{ij}$ de todos os $j$. Denominando a distribuição
conjunta de $\varepsilon=<\epsilon_{i1},\epsilon_{i2},\cdots,\epsilon_{iN}>$,
temos:
\begin{eqnarray*}
P_{ij} & = & Prob(U_{ij}>U_{ik},\forall k\neq j)\\
 & = & Prob(V_{ij}+\epsilon_{ij}>V_{ik}+\epsilon_{ik},\forall k\neq j)\\
 & = & Prob(V_{ij}-V_{ik}>\epsilon_{ik}-\epsilon_{ij},\forall k\neq j)\\
 & = & Prob(\epsilon_{ik}-\epsilon_{ij}<V_{ij}-V_{ik},\forall k\neq j)
\end{eqnarray*}
\end{itemize}
\end{frame}

\begin{frame}\frametitle{Modelos de Escolha Discreta III}
\small
\begin{itemize}
\item Esta última igualdade é uma distribuição acumulada, que nos diz a
probabilidade que cada um dos termos aleatórios $\epsilon_{ik}-\epsilon_{ij}$
está abaixo das diferenças entre as avaliações observadas $V_{ij}-V_{ik}$.
Podemos calcular este negócio, usando a distribuição conjunta dos
$\varepsilon$, com a seguinte integral multidimensional:\textrm{
\[
P_{ij}=\int_{\varepsilon}I(\epsilon_{ik}-\epsilon_{ij}<V_{ij}-V_{ik},\forall k\neq j)f(\varepsilon)d(\varepsilon)
\]
}
\item Em português, esta integral nos dá a área da distribuição conjunta
de $\varepsilon$ tal que as diferenças nos componentes idiosincráticos
sejam menores do que as diferenças nos componentes determinísticos.
\item Diferentes especificações de modelos de escolha discreta surgem em
resposta a diferentes especificações da variável aleatória multidimensional
$\varepsilon$. Por exemplo, se $\varepsilon$ for uma distribuição
$N(0,\Omega)$, isso nos dá o modelo probit multinomial.
\end{itemize}
\end{frame}


\subsection{Modelo LOGIT Multinomial}

\begin{frame}\frametitle{Modelos LOGIT Multinomial:}
\begin{itemize}
\item Se $\varepsilon$ seguir uma distribuição de valores extremos I:
\begin{eqnarray*}
f(\epsilon_{ij}) & = & e^{-\epsilon_{ij}}e^{-e^{-\epsilon_{ij}}}\\
F(\epsilon_{ij}) & = & e^{e^{-\epsilon_{ij}}}
\end{eqnarray*}
\item Temos o modelo LOGIT Multinomial. É importante notar que, para o caso
dos modelos LOGIT, a integral multidimensional que fizemos anteriormente
pode ser resolvida analiticamente.
\item O primeiro passo para entendermos isso é uma regrinha que diz que
as diferenças entre duas variáveis aleatórias que seguem esta distribuição
valores extremos I têm distribuição logística:
\begin{eqnarray*}
\epsilon_{ij}^{*} & = & \epsilon_{ik}-\epsilon_{ij}\\
F(\epsilon_{ij}^{*}) & = & \frac{\epsilon_{ij}^{*}}{1+\epsilon_{ij}^{*}}
\end{eqnarray*}
\end{itemize}
\end{frame}

\begin{frame}\frametitle{Modelo LOGIT Multinomial:}


\begin{itemize}
\item A segunda parada é que os componentes idiosincráticos das utilidades
são i.i.d.; mas antes, vamos reescrever a última das probabilidades
antes da integral da seguinte forma:
\[
P_{ij}=Prob(\epsilon_{ik}<\epsilon_{ij}+V_{ij}-V_{ik},\forall k\neq j)
\]
\item Se o $\epsilon_{ij}$ é considerado como dado, esta função nos dá
a função de distribuição acumulada para cada $\epsilon_{ik}$ avaliada
em $\epsilon_{ij}+V_{ij}-V_{ik}$, o que, de acordo com a distribuição
valores extremos I é igual a $\exp[-\exp[-(\epsilon_{ij}+V_{ij}-V_{ik})]]$.
Como os elementos do vetor $\varepsilon$ são independentes, isto
significa que esta probabilidade conjunta -- afinal de
contas, vale para todos os elementos de $\varepsilon$ exceto $j$
-- é igual a um produto das distribuições individuais:
\[
P_{ij}|\epsilon_{ij}=\prod_{k\neq j}e^{-e^{-(\epsilon_{ij}+V_{ij}-V_{ik})}}
\]
\end{itemize}
\end{frame}

\begin{frame}\frametitle{Modelo LOGIT Multinomial (II):}


\begin{itemize}
\item Evidentemente, $\epsilon_{ij}$ não é dado, desta forma a probabilidade
conjunta é a integral desta parada com respeito a todos os valores
de $\epsilon_{ij}$:
\[
P_{ij}=\int_{\epsilon_{ij}=-\infty}^{\infty}\left(\prod_{k\neq j}e^{-e^{-(\epsilon_{ij}+V_{ij}-V_{ik})}}\right)e^{-\epsilon_{ij}}e^{-e^{-\epsilon_{ij}}}d\epsilon_{ij}
\]
\item Vamos cozinhar um pouco esta equação; lembrando que, para o produto
$j$ , $V_{ij}-V_{ij}=0$, temos que a integral acima pode ser reconstruída
da seguinte forma:
\[
P_{ij}=\int_{\epsilon_{ij}=-\infty}^{\infty}\left(\prod_{k}e^{-e^{-(\epsilon_{ij}+V_{ij}-V_{ik})}}\right)e^{-\epsilon_{ij}}d\epsilon_{ij}
\]
\end{itemize}
\end{frame}

\begin{frame}\frametitle{Modelo LOGIT Multinomial (III):}


\begin{itemize}
\item Podemos transformar este produtório em soma, uma vez que as bases
são iguais:
\begin{eqnarray*}
P_{ij} & = & \int_{\epsilon_{ij}=-\infty}^{\infty}\exp\left(-\sum_{k}e^{-(\epsilon_{ij}+V_{ij}-V_{ik})}\right)e^{-\epsilon_{ij}}d\epsilon_{ij}\\
 & = & \int_{\epsilon_{ij}=-\infty}^{\infty}\exp\left(-e^{-\epsilon_{ij}}\sum_{k}e^{-(V_{ij}-V_{ik})}\right)e^{-\epsilon_{ij}}d\epsilon_{ij}
\end{eqnarray*}
\item Redefinindo as variáveis de integração, tal que $e^{-\epsilon_{ij}}=t$,
tal que $dt=-e^{-\epsilon_{ij}}d\epsilon_{ij}$. Note que, quando
$\epsilon_{ij}\rightarrow\infty$, $t\rightarrow0$, e quando $\epsilon_{ij}\rightarrow-\infty$,
$t\rightarrow-\infty$, o que faz com que os limites de integração
agora sejam 0 e $\infty$. 
\end{itemize}
\end{frame}

\begin{frame}\frametitle{Modelo LOGIT Multinomial}
\begin{itemize}
\item Usando este novo termo:
\begin{eqnarray*}
P_{ij} & = & \int_{t=\infty}^{0}\exp\left(-t\sum_{k}e^{-(V_{ij}-V_{ik})}\right)(-dt)\\
 & = & \int_{t=0}^{\infty}\exp\left(-t\sum_{k}e^{-(V_{ij}-V_{ik})}\right)dt\\
 & = & \left.\frac{\exp\left(-t\sum_{k}e^{-(V_{ij}-V_{ik})}\right)}{\sum_{k}e^{-(V_{ij}-V_{ik})}}\right|_{0}^{\infty}\\
 & = & \frac{1}{\sum_{k}e^{-(V_{ij}-V_{ik})}}=\frac{e^{V_{ij}}}{\sum_{k}e^{V_{ik}}}
\end{eqnarray*}
\item Podemos resumir os cuidados que temos na estimação dos modelos de
escolha discreta em duas afirmações, ``apenas diferenças
de utilidade são importantes'' e ``a escala da
utilidade é arbitrária''. 
\end{itemize}

\end{frame}

\begin{frame}[fragile]\frametitle{A Escala da Utilidade é Arbitrária}
    \begin{itemize}
    	\item Se somarmos uma constante a cada um dos termos $V_{ik}$, a fórmula da probabilidade do slide anterior não se altera.
    	\item Isso implica que os únicos parâmetros que podem ser estimados são aqueles que capturam diferenças entre as alternativas.
    	\item Como fazer com variáveis que são constantes entre as alternativas:
    	\begin{itemize}
    		\item Assumir diferentes coeficientes para cada alternativa
    	\end{itemize}
    	\item Como só as diferenças de utilidade importam, na verdade o modelo de utilidade aleatória é expresso em termo de $J-1$ diferenças.
    \end{itemize}


\end{frame}

\begin{frame}[fragile]\frametitle{Apenas diferenças de utilidade são importantes}
\begin{itemize}
	\item Podemos notar que se multiplicarmos todos os termos $V_{ik}$ por uma constante, a fórmula do slide anterior não se altera.
	\item Para lidar com isso, você precisa normalizar a escala dos termos erro, usualmente normalizando a variância dos $\varepsilon$
	\item No caso do Logit, a variância é $\frac{\pi^{2}}{6}$, ou aproximadamente 1.6. No Probit, a variância é 1.
	\item Por isso tem que tomar o cuidado em comparar os coeficientes do Probit e do Logit (os do logit são mais ou menos $\sqrt{1.6}$ o do Probit).
	\item Os coeficientes são $\frac{\beta}{\sigma}$, com $\sigma$ sendo o DP do $\varepsilon$.

\end{itemize}
    
\end{frame}

\begin{frame}\frametitle{Estimação dos Modelos de Escolha Discreta:}
\begin{itemize}
\item Em geral, os procedimentos de estimação do modelo Logit Multinomial
está baseado no princípio da Máxima Verossimilhança. Inicialmente,
vamos supor que a amostra seja aleatória, e que tenhamos dados sobre
$N$ tomadores de decisão. A probabilidade de um indivíduo $i$ escolher
a alternativa que ele efetivamente escolheu é igual a:
\[
\prod_{j\in J}(P_{ij})^{y_{ij}}
\]
\item Em que $y_{ij}=1$ se o indivíduo $i$ escolheu o produto e $y_{ij}=0$,
caso contrário. 
\end{itemize}
\end{frame}

\begin{frame}\frametitle{Estimação dos Modelos de Escolha Discreta (II):}

\begin{itemize}
\item Supondo independência das escolhas dos indivíduos, a probabilidade
de observação de uma amostra igual à que temos é:
\[
L(\beta)=\prod_{i\in N}\prod_{j\in J}(P_{ij})^{y_{ij}}
\]
\item Em geral, os algoritmos numéricos maximizam o logaritmo desta probabilidade
conjunta, o que dá:
\[
\ln(L(\beta))=LL(\beta)=\sum_{i\in N}\sum_{j\in J}y_{ij}\ln P_{ij}
\]
\end{itemize}
\end{frame}


\begin{frame}\frametitle{Estimação dos Modelos de Escolha Discreta (III):}
\begin{itemize}
\item Em geral, também podemos dar uma interpretação de GMM ao método de
estimação utilizado da seguinte forma. O vetor de parâmetros que minimiza
esta função deve atender à seguinte condição de primeira ordem:
\[
\frac{\partial LL(\beta)}{\partial\beta}=0
\]
\item Para facilitar, vamos supor que $V_{ij}=x_{ij}\beta$. Neste caso,
temos:
\[
\sum_{i\in N}\sum_{j\in J}(y_{ij}-P_{ij})x_{ij}=0
\]
\end{itemize}
\end{frame}




\subsection{Elasticidades e o Problema da IIA}
\begin{frame}[fragile]\frametitle{Efeitos Marginais}
    \small
    \begin{itemize}
    	\item Efeito Marginal sobre a Probabilidade de escolha do Produto $j$ de uma alteração no atributo $r$ do produto $j$ (efeito marginal próprio):
	\begin{eqnarray*}
	\frac{\partial P_{j}}{\partial x_{j}^{r}} & = & \frac{\partial(e^{Vij}/\sum_{k\in J}e^{V_{ik}})}{\partial x_{j}^{r}}\\
	 & = & \frac{\partial V_{j}}{\partial x_{j}^{r}}P_{j}(1-P_{j})
	\end{eqnarray*}
		\item Efeito Marginal sobre a Probabilidade de escolha do Produto $j$ de uma alteração no atributo $r$ de um produto $k\neq j$ (efeito marginal Cruzado):
	\begin{eqnarray*}
	\frac{\partial P_{j}}{\partial x_{k}^{r}} & = & \frac{\partial(e^{Vj}/\sum_{c\in J}e^{V_{ic}})}{\partial x_{k}^{r}}\\
	 & =- & \frac{\partial V_{ik}}{\partial x_{k}^{r}}P_{j}P_{k}
	\end{eqnarray*}
		
    \end{itemize}


\end{frame}

\begin{frame}[fragile]\frametitle{Elasticidades}
    \begin{itemize}
    	\item Elasticidade Própria:
		\[
		\epsilon = \frac{\partial P_{j}}{\partial x_{j}^{r}} \times \frac{x_{j}^{r}}{P_{j}}= \frac{\partial V_{j}}{\partial x_{j}^{r}}(1-P_{j})x_{j}^ {r}
		\]
		\item Elasticidade Cruzada:
		\[
		\epsilon_{ikr}=\frac{\partial P_{j}}{\partial x_{k}^{r}}\times\frac{x_{k}^{r}}{P_{j}}=- \frac{\partial V_{ik}}{\partial x_{k}^{r}}P_{k}x_{k}^{r}
		\]
		\item Esse último termo só depende de uma derivada parcial e de coisas relacionadas a $k$ -- e não a $j$
		\item Isso é chamada ``Independência das Alternativas Irrelevantes''
    \end{itemize}


\end{frame}

\section{Soluções para o Problema da IIA}
\begin{frame}[fragile]\frametitle{\insertsection}
	\begin{enumerate}
    	\item Estipular ``ex ante'' que os termos $\varepsilon$ sejam correlacionados -- a saída do Logit Aninhado
    	\item Permitir variação aleatória entre os gostos, o que induz a correlação entre os $\varepsilon$ -- a saída do Random Coefficient Logit (ou Mixed Logit)
    	\item Algumas literaturas consideram interagir $x_{k}^{r}$ com case specific variables pra resolver isso.
    \end{enumerate}    


\end{frame}


\begin{comment}
\begin{frame}[allowframebreaks]
\bibliographystyle{aea}
\bibliography{C:/Bibliog/library}

\end{frame}

\end{comment}




\end{document}


