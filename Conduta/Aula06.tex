\documentclass{beamer}
\usepackage{beamerthemesplit}
\usepackage[brazil]{babel}
\usepackage{epsfig}
\usepackage[utf8x]{inputenc}
\usepackage{pgf}
\usepackage{tikz}
%\usetikzlibrary{snakes}
\usepackage{nicefrac}
\usepackage{amsfonts}
\usepackage{amsmath}
\usepackage{amssymb}
\usepackage{amsthm}
%\usepackage{float}
\usetheme{Frankfurt}
\usepackage{epstopdf}
\usepackage{comment}
\usepackage{natbib}
\usepackage{float}
\usepackage{graphicx}
\usepackage{booktabs}
\usepackage{array}
\usepackage{bookmark}
\usepackage[normalem]{ulem}

\title{Aula 06}

\subtitle{Conduta}

\author{Claudio R. Lucinda}


\institute{FEA-RP/USP}

\date{}
\logo{\includegraphics[scale=.1]{logousp.png}}
\beamertemplatenavigationsymbolsempty
\begin{document}

\frame{\titlepage}
\begin{frame}\frametitle{Agenda}
  \tableofcontents[pausesections]
\end{frame}

\section{O Parâmetro de Conduta}
\begin{frame}\frametitle{Receita Marginal=Custo Marginal, revisitados}
\begin{itemize}
\item Seria possível determinar, a partir dos dados observados, se os casos
polares de competição e monopólio são discerníveis?
\begin{itemize}
\item Basicamente, o autor pressupõe que as quantidades e preços observados
são o resultado do equilíbrio entre uma curva de demanda e uma \uline{relação}
de oferta. 
\item A forma mais geral desta relação de oferta é $RMg_{p}=CMg$, em que
$RMg_{p}$ representa a receita marginal percebida pela empresa. 
\end{itemize}
\end{itemize}
\end{frame}

\begin{frame}\frametitle{Receita Marginal Percebida}

\begin{itemize}
\item A Receita Marginal associada, por sua vez, é dada por:
\[
RMg_{j}=\frac{\partial(P\times q_{j})}{\partial q_{j}}=P+q_{j}\frac{\partial P}{\partial Q}\frac{\partial Q}{\partial q_{j}}
\]
\item O termo $\frac{\partial Q}{\partial q_{j}}$ é o que dá a ``percebida''
para a Receita Marginal percebida.
\item Esta fórmula abrange todos os casos tradicionais de Microeconomia.
\end{itemize}
\end{frame}

\begin{frame}\frametitle{Receita Marginal Percebida (II)}

\begin{itemize}
\item Vamos desempacotar um pouco mais esta parte ``percebida'' da Receita
Marginal:
\[
\frac{\partial Q}{\partial q_{j}}=\sum_{k=1}^{J}\frac{\partial q_{k}}{\partial q_{j}}=\left(1+\sum_{k\neq j}\frac{\partial q_{k}}{\partial q_{j}}\right)=\lambda
\]
\item O que implica na seguinte expressão para a Receita Marginal Percebida
- se assumirmos uma demanda linear:
\end{itemize}
\[
RMg_{j}=P+q_{j}\frac{\lambda}{\alpha_{1}}
\]

\begin{itemize}
\item As estimativas para o parâmetro $\lambda$ podem ser obtidas a partir
da estimação de demanda - para obtermos uma estimativa para o $\alpha_{1}$-
e uma relação de oferta, que nada mais é do que igualarmos isso à
receita marginal percebida ao custo marginal.
\end{itemize}
\end{frame}

\begin{frame}\frametitle{O Papel dos Custos}
\begin{itemize}
\item Aqui precisamos trazer o outro lado -- da tecnologia -- para derivarmos a nossa relação de oferta. Existem várias
opções aqui:
\begin{itemize}
\item Custos Marginais determinados externamente -- Por exemplo, \citet{Genesove1998a}  fazem isso para o mercado de açúcar, e \citet{Wolfram1999a} para o mercado de eletricidade da Califórnia.
\item Imposição de uma forma funcional linear nos parâmetros e dependendo dos custos de fatores, mas \uline{não} dependendo da quantidade produzida 
\item Imposição de uma forma funcional linear nos parâmetros e dependendo dos custos de fatores \uline{e} da quantidade produzida da empresa. O Caso de \citet{Bresnahan1982a} e \citet{Lau1982a}.
\end{itemize}
\end{itemize}
\end{frame}

\begin{frame}\frametitle{O Papel dos Custos (II):}

\begin{itemize}
\item Iremos falar aqui do segundo dos casos, em que:
\end{itemize}
\[
CMg_{j}=\phi_{0}+\phi_{1}W_{j}+\xi_{j}
\]
\begin{itemize}
\item Igualando o Custo Marginal com a Receita Marginal Percebida,
temos:
\begin{eqnarray*}
RMg_{j} & = & CMg_{j}\\
P+q_{j}\frac{\lambda}{\alpha_{1}} & = & \phi_{0}+\phi_{1}W_{j}+\xi_{j}
\end{eqnarray*}
\end{itemize}
\end{frame}

\begin{frame}\frametitle{Modelo Estimável}

\begin{itemize}
\item Com tudo isso, chegamos às seguintes equações estimáveis:
\end{itemize}
\begin{eqnarray*}
P & = & \frac{-\alpha_{0}}{\alpha_{1}}-\frac{\alpha_{2}}{\alpha_{1}}Y+\frac{Q}{\alpha_{1}}-\epsilon\\
P & = & -q_{j}\frac{\lambda}{\alpha_{1}}+\phi_{0}+\phi_{1}W_{j}+\xi_{j}
\end{eqnarray*}

\begin{itemize}
\item Basicamente, qual é o problema econométrico aqui? 
\item Na equação de baixo, temos uma razão entre dois coeficientes, $\lambda$
e $\alpha_{1}$. Todavia, conseguimos estimar o $\alpha_{1}$ a partir
da equação de demanda (basicamente a gente usa $W_{j}$ como instrumento
para $Q$ na equação de demanda), e aí consegue desamarrar o $\lambda$.
\end{itemize}
\end{frame}

\begin{frame}\frametitle{Valores de $\lambda$}
\small
\begin{itemize}
\item E como este $\lambda$ pode ser usado para identificarmos
o sobrepreço de cartel? Em primeiro lugar, como checagem de consistência
da metodologia, deveríamos checar se o $\lambda$ resultante está
de acordo com alguma estrutura de mercado:
\begin{itemize}
\item Para Oligopólio competindo à la Cournot: $\lambda=1$
\item Para Competição Perfeita: $\lambda=0$
\item Para Cartel/Solução de Monopólio: $\lambda=\nicefrac{1}{s_{j}}$
\end{itemize}
\item Independentemente do que se considera sobre a modelagem do
parâmetro de conduta, podemos obter qual seria o preço contrafactual.
\begin{itemize}
\item Basicamente, substituiríamos no valor de $\lambda$ o valor
correspondente à estrutura de mercado que consideraríamos que melhor
representasse o comportamento das empresas no cartel e utilizamos
os valores das variáveis e outros coeficientes para estimar uma trejetória
contrafactual para a empresa. 
\end{itemize}
\end{itemize}
\end{frame}

\begin{frame}\frametitle{Mas....(Sempre tem um mas, não é?)}

\begin{itemize}
\item No entanto, não podemos perder de vista que isso é o resultado das
seguintes premissas:

\begin{itemize}
\item Produto Homogêneo
\item Demanda Linear
\item Custo Marginal independente da quantidade produzida
\item Termos quantidades das empresas específicas
\end{itemize}
\item Cada uma destas premissas pode -- ou não --
ser adequuada para o contexto em que se faz necessário o cálculo da
margem do cartel. Mas temos que saber o que fazer quando estas premissas
\uline{não} são adequadas para o contexto e, mesmo assim, desejamos
manter a abordagem estrutural para estimar a margem de sobrepreço
do cartel. 
\end{itemize}
\end{frame}


\section{Extensões}

\subsection{Demanda Não Linear}
\begin{frame}\frametitle{Demanda Não Linear}
\small
\begin{itemize}
\item Neste caso, o que precisamos fazer é adaptar a função
Receita Marginal Percebida para o contexto em questão. Vamos, por
simplicidade, supor uma demanda duplo log:
\[
\ln P=\alpha_{0}+\alpha_{1}\ln Q+\alpha_{2}\ln Y+\epsilon
\]
\item Se continuarmos supondo uma curva de custo marginal
horizontal em relação à quantidade produzida, poderíamos representar
isso da seguinte forma:
\[
P\left(1+s_{j}\alpha_{1}\lambda\right)=\phi_{0}+\phi_{1}W_{j}+\xi_{j}
\]
\item O jeito mais simples de estimar este $\lambda$ é
por meio de GMM. O resíduo do GMM é dado por:
\[
\xi_{j}=P\left(1+s_{j}\alpha_{1}\lambda\right)-\phi_{0}+-\phi W_{j}
\]
\item Como a gente novamente pode estimar o $\alpha_{1}$
consistentemente por GMM da demanda com o $W_{j}$ sendo instrumento
para $\ln Q$, a gente poderia estimar essa relação de oferta usando
$\ln Y,P,s_{j},W_{j}$ como instrumentos.
\end{itemize}
\end{frame}


\subsection{Dados Agregados}
\begin{frame}\frametitle{Dados Agregados}

\begin{itemize}
\item E se não tivermos dados por empresa, apenas de mercado?
\item Dá pra fazer a mesma análise, apenas a interpretação dos valores correspondentes
às estruturas de mercado muda um pouco
\item Vamos multiplicar a Receita Marginal percebida pela empresa $j$,
bem como o Custo Marginal, pelas participações de mercado e somar
para todas as empresas. Note que ainda não estamos levando esta equação
aos dados:
\[
\sum_{j}s_{j}RMg_{j}=P\sum_{j}s_{j}+\frac{\partial P}{\partial Q}\sum_{j}q_{j}s_{j}\frac{\partial Q}{\partial q_{j}}=\sum_{j}s_{j}CMg_{j}=\bar{CMg}
\]
\item Reorganizando:
\[
P+\frac{\partial P}{\partial Q}Q\sum_{j}s_{j}\frac{\partial Q}{\partial q_{j}}\frac{q_{j}}{Q}=\bar{CMg}
\]
\end{itemize}
\end{frame}

\begin{frame}\frametitle{Dados Agregados (II):}
\small
\begin{itemize}
\item Não iremos conseguir desempacotar mais a somatória,
então a representaremos por $\lambda^{A}$. Utilizando as mesmas funções
de antes:
\[
P+Q\frac{\lambda^{A}}{\alpha_{1}}=\phi_{0}+\phi_{1}W_{j}+\xi_{j}
\]
\item O caminho ali descrito para identificarmos os parâmetros
continua inalterado. No entanto, a interpretação dos valores de $\lambda^{A}$
é diferente. Lembrando que:
\[
\lambda^{A}=\sum_{j}s_{j}^{2}\frac{\partial Q}{\partial q_{j}}
\]
\item Temos que, se as empresas funcionam em um mercado
competindo à la Cournot, e $\nicefrac{\partial Q}{\partial q_{j}}=1$,
temos que este $\lambda^{A}$ precisaria ser igual ao Índice Herfindahl-Hirschman
(a soma dos quadrados das participações de mercado, com os shares
expressos em forma decimal). Se estivermos em concorrência perfeita,
temos que $\lambda^{A}=0$ e, no caso de uma solução de cartel, teríamos
que $\lambda^{A}=1$.
\end{itemize}
\end{frame}

\begin{frame}\frametitle{Custos Dependendo da Quantidade da Empresa}

\begin{itemize}
\item Essa foi a contribuição de \citet{Bresnahan1982a} e \citet{Lau1982a},
que mostram que uma condição suficiente para a identificação de $\lambda$
neste caso é que a variável endógena na equação de demanda não seja
aditivamente separável em relação aos outros deslocadores de demanda.
\item Em outras palavras, pelo menos um deslocador de demanda
tem que interagir com a variável preço.
\item Caso não tenhamos isso, a relação de oferta fica sendo:
\begin{eqnarray*}
CMg_{j} & = & RMg_{j}\\
\beta_{0}+\beta_{1}q_{j}+\beta_{2}W & = & P+\lambda\frac{q_{j}}{\alpha_{1}}\\
P & = & \beta_{0}+q_{j}\left(\beta_{1}-\frac{\lambda}{\alpha_{1}}\right)+\beta_{2}W
\end{eqnarray*}
\end{itemize}
\end{frame}

\begin{frame}\frametitle{Custos Dependendo da Quantidade da Empresa (II):}

\begin{itemize}
\item No slide anterior, a gente teria três (na verdade dois, porque $\alpha_{1}$
você estima da equação de demanda) coeficientes para identificar,
e apenas uma variável - o $q_{j}$. Neste sentido, não dá pra identificar
as coisas.
\item Agora, se tivermos um deslocador de demanda interagindo com a variável
endógena na equação de demanda, temos:
\end{itemize}
\[
P=\beta_{0}+q_{j}\left(\beta_{1}-\frac{\lambda}{\alpha_{1}+\alpha_{3}Y}\right)+\beta_{2}W
\]

\begin{itemize}
\item Neste caso, conseguimos identificar adequadamente os coeficientes
envolvidos, pois o pacote de três coeficientes, $\alpha_{1}$, $\alpha_{3}$
e $\lambda$ pode ser obtido por duas variáveis diferentes --
$q_{j}$ e $Y$. 
\end{itemize}
\end{frame}


\subsection{Produtos Diferenciados}
\begin{frame}\frametitle{Produtos Diferenciados}

\begin{itemize}
\small
\item Uma outra crítica foi levantada por \citet{Nevo1998a} à abordagem de Breshanan
(1982) e Lau (1982), especialmente relevante quando consideramos mercados
em que os produtos são fortemente diferenciados. 

\begin{itemize}
\item Iremos entender aqui produtos diferenciados como sendo aqueles em
que a elasticidade cruzada é bem definida -- ou seja, se
um produtor reduz unilateralmente o preço do seu bem ele não rouba
toda a demanda das outras empresas.
\end{itemize}
\item Neste caso, a demanda pelo produto de uma empresa $j$ é uma função
não apenas do preço de mercado do próprio produto mas também dos preços
dos produtos que são seus substitutos imperfeitos.
\item O problema é que, para identificar os $\lambda$'s e os $\alpha$'s
teríamos que ter a dimensão de variáveis exógenas nas equações de
demanda igual ao número de produtos. Quando temos dois produtos, encontrar
dois instrumentos é fácil. 
\end{itemize}
\end{frame}

\begin{frame}\frametitle{Produtos Diferenciados (II):}

\begin{itemize}
\item As Condições de Primeira Ordem de uma empresa competindo em preços
são:
\end{itemize}
\[
q_{i}^{j}(\mathbf{p})+\sum_{i\in I_{j}}(p_{i}-CMg_{i})\times\frac{\partial q_{i}^{j}}{\partial p_{i}}=0
\]

\begin{itemize}
\item Para deixar a relação de oferta mais clara, iremos representar este
sistema de condições de primeira ordem em forma matricial, definindo
duas matrizes. 
\item A primeira delas será a que coleta todos os efeitos marginais e tem
elemento típico$S_{jr}=-\frac{\partial q_{r}}{\partial p_{j}},j,r=1,\cdots,J$,
que é o negativo do vetor de sensibilidades das demandas aos preços
e a segunda é o que seria uma matriz seletora de propriedade
\end{itemize}
\end{frame}

\begin{frame}\frametitle{Produtos Diferenciados (III):}

\begin{itemize}
\item Matriz Seletora de Propriedade
\[
\Theta_{jr}=\begin{cases}
1, & \exists f:\{r,j\}\in I_{j}\\
0 & c.c.
\end{cases}
\]
\item Adicionalmente, podemos definir $\Omega_{jr}=\Theta_{jr}S_{jr}$,
o que permite representar em notação vetorial o sistema de condições
de primeira ordem da seguinte forma:
\[
\mathbf{q(p)}-\Omega(\mathbf{p}-\mathbf{CMg})=\mathbf{0}
\]
\item Reorganizando:
\[
\mathbf{p=CMg+\Omega^{-1}q(p)}
\]
\end{itemize}
\end{frame}

\begin{frame}\frametitle{Produtos Diferenciados (IV):}

\begin{itemize}
\item Nevo (1998)afirma que a modelagem de parâmetros de conduta como a
que vimos equivaleria a substituir a matriz $\Theta$ por uma matriz
de parâmetros a serem estimados. 

\begin{itemize}
\item O problema que o autor levanta é que, no caso de produtos diferenciados,
é extremamente difícil identificar os parâmetros constantes nesta
matriz $\Theta$, além da identificação dos efeitos cruzados.
\end{itemize}
\item Para resolver este problema, os autores sugerem que, para a identificação
da conduta do setor, sejam estimados diferentes modelos e depois fazendo
um teste estatístico formal. 
\item Eles envolvem as chamadas ``hipóteses não-aninhadas''. 
\end{itemize}
\end{frame}


%\begin{comment}
\begin{frame}[allowframebreaks]
\bibliographystyle{aea}
\bibliography{C:/Bibliog/library}

\end{frame}

%\end{comment}




\end{document}


