\documentclass{beamer}
\usepackage{beamerthemesplit}
\usepackage[brazil]{babel}
\usepackage{epsfig}
\usepackage[utf8x]{inputenc}
\usepackage{pgf}
%\usepackage{tikz}
%\usetikzlibrary{snakes}
\usepackage{nicefrac}
\usepackage{amsfonts}
\usepackage{amsmath}
\usepackage{amssymb}
\usepackage{amsthm}
%\usepackage{float}
\usetheme{Frankfurt}
\usepackage{epstopdf}
\usepackage{comment}
\usepackage{natbib}
\usepackage{float}
\usepackage{graphicx}
\usepackage{booktabs}
\usepackage{array}
\usepackage{bookmark}
\usepackage[normalem]{ulem}

\title{Aula 06}

\subtitle{Conduta -- Outras Abordagens}

\author{Claudio R. Lucinda}


\institute{FEA-RP/USP}

\date{}
\logo{\includegraphics[scale=.1]{logousp.png}}
\beamertemplatenavigationsymbolsempty
\begin{document}

\frame{\titlepage}
\begin{frame}\frametitle{Agenda}
  \tableofcontents[pausesections]
\end{frame}

\section{Funcoes de Producao}
\begin{frame}[fragile]\frametitle{Funções de Produção e Conduta}
    \begin{itemize}
        \item O ponto de partida é a tradicional função de produção homogênea
        \begin{equation*}
            Y=F(K,L)
        \end{equation*}
        \item Passando o Logaritmo dos dois Lados:
        \begin{align*}
            \log{Y}&=\log{F(K,L)}  \\
            d\log{Y}&=\left [\frac{\partial F}{\partial K}\frac{Y}{K}d\log{K} + \frac{\partial F}{\partial L}\frac{Y}{L} d\log{L} \right]
        \end{align*}
        \item O crescimento do progresso técnico é a diferença entre essas duas coisas
        \begin{equation}
            d\log{Z}=d\log{Y}-\left [\frac{\partial F}{\partial K}\frac{K}{Y}d\log{K} + \frac{\partial F}{\partial L}\frac{L}{Y} d\log{L} \right]
        \end{equation}
    \end{itemize}
    

\end{frame}
\begin{frame}[fragile]\frametitle{Resíduo de Solow -- Continuação}
    \begin{itemize}
        \item Os termos $\nicefrac{\partial F}{\partial K} \times \nicefrac{K}{Y}$ e $\nicefrac{\partial F}{\partial L} \times \nicefrac{L}{Y}$ são as elasticidades da produção com respeito aos fatores de produção
        \item Caso tenhamos
        \begin{itemize}
            \item Retornos Constantes de Escala
            \item Competição Perfeita nos mercados de fatores e de produtos
        \end{itemize}
        \item Esses negócios são iguais às participações dos fatores de produção no valor da produção
        \item Isso pode fazer sentido na Macroeconomia, mas será que faz na microeconomia?
    \end{itemize}


\end{frame}

\subsection{Caso 1: Retornos de Escala}
\begin{frame}[fragile]\frametitle{Revisitando o lado dos Custos}
    \begin{itemize}
        \item Vamos lembrar que os Custos podem ser definidos como:
        \begin{equation*}
            C=wL+rK
        \end{equation*}
        \item E vamos construir uma medida de economias de escala como a razão entre custo médio e custo marginal:
        \begin{align*}
            \gamma&=\frac{\nicefrac{C}{Y}}{\nicefrac{\partial C}{\partial Y}} \\
            C&=\gamma \times \frac{\partial C}{\partial Y} \times Y
        \end{align*}
    \end{itemize}


\end{frame}
\begin{frame}[fragile]\frametitle{Custos -- Parte II}
    \begin{itemize}
        \item Agora vamos lembrar de outra coisa: se temos competição no mercado de fatores, o uso do fator vai ser até o ponto em que a remuneração do mesmo seja igual ao Valor do Produto Marginal do mesmo
        \begin{align*}
            \frac{\partial F}{\partial K}\times P&=r \\
            \frac{\partial F}{\partial L}\times P&=w
        \end{align*}
        \item Mas temos competição perfeita, então $P=\nicefrac{\partial C}{\partial Y}$, o que implica:
        \begin{align*}
            \frac{\partial F}{\partial K}\times \frac{\partial C}{\partial Y}&=r \\
            \frac{\partial F}{\partial L}\times \frac{\partial C}{\partial Y}&=w
        \end{align*}
    \end{itemize}


\end{frame}

\begin{frame}[fragile]\frametitle{Custos -- Parte III}
    \begin{itemize}
        \item Reorganizando, podemos escrever:
        \begin{align*}
            \frac{\partial F}{\partial K}\frac{K}{Y}&=\gamma \frac{Y}{C} \frac{K}{Y} r= \gamma  \frac{rK}{C}\\
            \frac{\partial F}{\partial L}\frac{L}{Y}&=\gamma \frac{Y}{C} \frac{L}{Y} w=\gamma \frac{wL}{C} \\
        \end{align*}
        \item Ou seja, neste caso, o resíduo de Solow é dado por
        \begin{equation}
            d\log{Z}=d\log{Y}-\gamma \left [s_K d\log{K} + s_L d\log{L} \right]
        \end{equation}
        \item Em que $s_K$ e $s_L$ são as participações dos fatores de produção nos custos (e nas receitas)
    \end{itemize}


\end{frame}

\subsection{Caso 2: Competição Imperfeita}
\begin{frame}[fragile]\frametitle{Competição Imperfeita}
    \begin{itemize}
        \item E quando temos Competição Imperfeita?
        \item Neste caso, temos que $P \neq CMg$, mas podemos escrever $P=\mu \times CMg$, em que $\mu$ é a razão Preço/Custo Marginal.
        \item Reorganizando:
        \begin{align*}
            C&=\frac{\gamma}{\mu}\times P \times Y \\
            \frac{wL}{C}&=\frac{\mu}{\gamma}\frac{wL}{PY} \\
            \frac{rK}{C}&=\frac{\mu}{\gamma}\frac{rK}{PY}
        \end{align*}
    \end{itemize}


\end{frame}
\begin{frame}[fragile]\frametitle{Competição Imperfeita -- II}
    \begin{itemize}
        \item Neste caso, a equação dos resíduos de Solow fica sendo:
        \begin{equation}
            d\log{Z}=d\log{Y}-\mu \left [s_K d\log{K} + s_L d\log{L} \right]
        \end{equation}
        \item Em que $s_K$ e $s_L$ são as participações dos fatores de produção NAS RECEITAS (que são diferentes dos custos)
        \item Ou seja:
            \begin{itemize}
                \item Só com Retornos Constantes de Escala E Competição Perfeita temos que as elasticidades dos fatores com relação à produção são iguais aos shares de fatores
                \item Quando UMA destas coisas não acontece, temos que as participações das remunerações dos fatores na receita verdadeiras não somam 1, mas sim $\nicefrac{\gamma}{\mu}$ 
            \end{itemize}
    \end{itemize}


\end{frame}
\section{A Abordagem de Hall}
\begin{frame}[fragile]\frametitle{Implicações Empíricas deste Monte de Coisa}
\scriptsize
    \begin{itemize}
        \item A partir dessa volta pelo Resíduo de Solow, podemos tirar a seguinte definição \citet{Hall1991a}:
    \begin{definition}    
        O Crescimento da Produtividade não deve ser correlacionado com nenhuma variável que (a) afete a produção e que (b) não seja argumento da função de produção
    \end{definition}
        \item Essa é a base para a primeira parte do teste de \citet{Hall1988} -- Calcular o Resíduo de Solow da forma tradicional e checar se ele é correlacionado com alguma coisa que a gente sabe que afeta o produto.
        \item Se afetar, temos evidência de Retornos Não Constantes de Escala e/ou Competição Imperfeita:
        \item Outra Implicação importante depois: $\mu=\frac{PY}{C}=\frac{\nicefrac{\partial ln(F)}{\partial ln(L)}}{\nicefrac{wL}{PY}}$
    \end{itemize}

\end{frame}

\subsection{Os Testes de Hall}
\begin{frame}[fragile]\frametitle{Abordagem de Hall}
    \begin{itemize}
        \item Duas Etapas:
        \begin{itemize}
            \item Na primeira, é testado o conjunto (Retornos Constantes de Escala E Competição perfeita)
            \begin{itemize}
                \item Neste caso, o Resíduo de Solow calculado da forma tradicional é regredido contra uma variável que espera-se que não afete a produtividade mas sim o PIB. 
                \item Pelo slide anterior, esta variável não deveria afetar a produtividade.
            \end{itemize}
            \item Na segunda, é estimada uma regressão do tipo $ln(\nicefrac{Y}{K})=\mu [s_L ln(\nicefrac{L}{K})]$ e estima-se o $\mu$
        \end{itemize}
    \end{itemize}

\end{frame}

\subsection{A Abordagem de De Loecker}
\begin{frame}[fragile]\frametitle{De Loecker}
    \begin{itemize}
        \item Em 2006, apareceu um paper que buscou revisitar essa abordagem, \citet{Loecker2009a}
        \item A essência da abordagem reside em que podemos calcular a razão preço-custo marginal como a razão entre duas coisas:
        \begin{itemize}
            \item A participação de cada insumo na receita
            \item A elasticidade produto de cada fator de produção
        \end{itemize}
        \item Isso tem a vantagem que você não precisa de informações detalhadas sobre volume produzido dos diferentes produtos, preços de cada um deles...
    \end{itemize}


\end{frame}

\section{Estatística Panzar-Rosse}
\begin{frame}[fragile]\frametitle{Estatística Panzar-Rosse}
\begin{itemize}
	\item No final dos anos 70, Panzar e Rosse (\citet{Panzar1987}) propuseram uma forma de se determinar o comportamento das empresas sem precisar de informações de preços e quantidades. 
	\item A ``Estatística Panzar-Rosse'' nada mais é do que a soma das elasticidades da receita total com relação aos preços dos fatores de produção:
	\begin{align*}
	\ln{R}&=\alpha_{0}+\sum_{i} \beta_{i}\ln{w_{i}}+\varepsilon\\
	h&=\sum_{i} \beta_{i}
	\end{align*}
	\item Diferentes estruturas de competição podem ser associadas a diferentes valores de $h$
\end{itemize}
    


\end{frame}

\begin{frame}[fragile]\frametitle{Estatística Panzar-Rosse}
\footnotesize
\begin{itemize}
	\item Monopólio: $h<0$. Intuitivamente, em resposta a uma elevação nos custos em $h$\%, as receitas totais se reduzem.
	\item Competição Monopolística: $h\leq 1$.
	\item Competição Perfeita: $h=1$
	\item Oligopólio: Como mostrado em \citet{Shaffer1982}, podemos mostrar que, para elasticidade-preço da demanda constante e custo linear na quantidade produzida, temos que:
	\begin{eqnarray*}
	h & = & 1-\frac{1}{m}\\
	m & = & \frac{p-CMg}{p}=\frac{1}{\varepsilon}
	\end{eqnarray*}

	\item No contexto de modelos específicos de oligopólio, coisas adicionais podem ser feitas. 
	\item Supondo, custo marginal constante na quantidade produzida, no caso de conjecturas equivalentes (ou seja, as empresas seguem um comportamento oligopolizado), temos que $h\leq0$. 
\end{itemize}
    


\end{frame}

\section{Demanda Residual}
\begin{frame}\frametitle{Demanda Residual}
\small
\begin{itemize}
\item A idéia desta abordagem é propôr procedimentos econométricos para
a estimação de um sistema de demanda que será enfrentado pelas partes
que serão objeto de fusão, baseada somente em dados pré-fusão. 
\item O elemento chave para isto é a agregação dos efeitos de todas as outras
firmas em um único parâmetro. 
\item Formalmente, começaremos com um sistema de demanda de produtos diferenciados
com $n$ empresas. 
\item Após manipulação do sistema, eliminamos os preços de todas as empresas,
exceto as duas diretamente envolvidas em uma fusão. 
\item Isto reduz a dimensionalidade do sistema, para um tamanho que o torna
gerenciável. 
\item Passemos então à descrição da metodologia propriamente dita, apresentada originalmente por \citet{Baker1985a} e \citet{Baker1988a}
\end{itemize}
\end{frame}

\begin{frame}\frametitle{Demanda Residual (II):}

\begin{itemize}
\item Vamos supor que duas empresas estejam contemplando a fusão. Podemos
representar a função demanda inversa de cada uma das empresas da seguinte
forma:
\begin{eqnarray*}
P_{1} & = & h_{1}(q_{1},q_{2},\tilde{q},Y,\eta_{1})\\
P_{2} & = & h_{2}(q_{1},q_{2},\tilde{q},Y,\eta_{2})
\end{eqnarray*}

\item Em que $\tilde{q}$ representa a produção das $n-2$ empresas não
diretamente envolvidas, $Y$ um vetor de deslocadores de demanda,
$\eta_{1}$ e $\eta_{2}$ são vetores de parâmetros da função demanda
das duas empresas.
\item Podemos, adicionalmente, representar a função demanda inversa das
$n-2$ empresas remanescentes do mercado da seguinte forma:
\[
\tilde{P}=\tilde{h}(q_{1},q_{2},\tilde{q},Y,\tilde{\eta})
\]

\end{itemize}
\end{frame}

\begin{frame}\frametitle{Demanda Residual (III):}

\begin{itemize}
\item Agora, vamos especificar o comportamento das $n-2$ empresas remanescentes. 
\item Independentemente da hipótese comportamental, podemos assumir que
elas agirão de forma a igualar a receita marginal \uline{percebida}
(ou seja, envolvendo as conjecturas das empresas sobre o comportamento
das outras) igual ao seu custo marginal. 
\item Ou seja, para estas $n-2$ empresas a receita marginal é da seguinte
forma:
\[
\tilde{MR}=\tilde{P}+\tilde{g}(q_{1},q_{2},\tilde{q},Y,\tilde{\eta})\tilde{q}
\]

\item Sendo que a função $\tilde{g}$ é uma representação da inclinação
da demanda $\tilde{h}$ percebida pelas $n-2$ empresas. 
\end{itemize}
\end{frame}

\begin{frame}\frametitle{Demanda Residual (IV):}

\begin{itemize}
\item Da mesma forma, podemos escrever a função custo marginal da seguinte
forma:
\[
\tilde{CMg}=\tilde{j}(\tilde{q},W,\tilde{\beta)}
\]

\item Ou seja, a relação de preços é dada pela seguinte igualdade:
\[
\tilde{MR}=\tilde{P}+\tilde{g}(q_{1},q_{2},\tilde{q},Y,\tilde{\eta})\tilde{q}=\tilde{CMg}=\tilde{j}(\tilde{q},W,\tilde{\beta)}
\]

\item Resolvendo o sistema composto pela equação de demanda e a igualdade
entre receita marginal e custo marginal, podemos expressar a quantidade
produzida pelas $n-2$ firmas da seguinte forma:
\[
\tilde{q}=\tilde{e}(q_{1},q_{2},Y,W,\tilde{\eta},\tilde{\beta})
\]

\end{itemize}
\end{frame}

\begin{frame}\frametitle{Demanda Residual (V):}

\begin{itemize}
\item Com esta equação, podemos escrever uma forma reduzida do sistema para
as duas equações que estão se unindo, da seguinte forma;
\begin{eqnarray*}
P_{1} & = & h_{1}(q_{1},q_{2},\tilde{e}(q_{1},q_{2},Y,W,\tilde{\eta},\tilde{\beta}),Y,\eta_{1})\\
P_{2} & = & h_{2}(q_{1},q_{2},\tilde{e}(q_{1},q_{2},Y,W,\tilde{\eta},\tilde{\beta}),Y,\eta_{2})\\
P_{1} & = & r_{1}(q_{1},q_{2},Y,W,\eta_{1},\tilde{\eta},\tilde{\beta})\\
P_{2} & = & r_{2}(q_{1},q_{2},Y,W,\eta_{2},\tilde{\eta},\tilde{\beta})
\end{eqnarray*}

\item A análise empírica se baseia na estimação do sistema composto pelas
duas equações do sistema. 
\end{itemize}
\end{frame}


%\begin{comment}
\begin{frame}[allowframebreaks]
\bibliographystyle{aea}
\bibliography{C:/Bibliog/library}

\end{frame}

%\end{comment}




\end{document}


