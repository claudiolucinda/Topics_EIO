\documentclass{beamer}
\usepackage{beamerthemesplit}
\usepackage[brazil]{babel}
\usepackage{epsfig}
\usepackage[utf8x]{inputenc}
\usepackage{pgf}
\usepackage{tikz}
%\usetikzlibrary{snakes}
\usepackage{nicefrac}
\usepackage{amsfonts}
\usepackage{amsmath}
\usepackage{amssymb}
\usepackage{amsthm}
%\usepackage{float}
\usetheme{Frankfurt}
\usepackage{epstopdf}
\usepackage{comment}
\usepackage{natbib}
\usepackage{float}
\usepackage{graphicx}
\usepackage{booktabs}
\usepackage{array}
\usepackage{bookmark}

\title{Aula 05}

\subtitle{Funcoes de Producao II}

\author{Claudio R. Lucinda}


\institute{FEA-RP/USP}

\date{}
\logo{\includegraphics[scale=.1]{logousp.png}}
\beamertemplatenavigationsymbolsempty
\begin{document}

\frame{\titlepage}
\begin{frame}\frametitle{Agenda}
  \tableofcontents[pausesections]
\end{frame}

\section{Funcoes de Producao}

\subsection{Olley e Pakes}

\begin{frame}\frametitle{Olley e Pakes (1996)}
\begin{itemize}
\item \citet{Olley1996} começam o seu estudo afirmando que, para a obtenção
de estimativas consistentes dos parâmetros de funções de produção
dois problemas inter-relacionados precisam ser resolvidos: um problema
de seleção gerado pela relação entre a variável não observada de produtividade
e a decisão de saída do mercado, e um problema de simultaneidade gerado
pela relação entre produtividade e demandas de fatores. 
\item Neste caso, os autores propõem um algoritmo, cujo primeiro estágio
se baseia na seguinte regra de acumulação para o capital:
\begin{eqnarray*}
k_{t+1} & = & (1-\delta)k_{t}+i_{t}\\
a_{t+1} & = & a_{t}+1
\end{eqnarray*}
\item Sendo $k_{t}$ o capital, $\delta$ a depreciação, e $i_{t}$ o investimento
em capital, e $a_{t}$ a idade da empresa. 
\end{itemize}
\end{frame}

\begin{frame}\frametitle{Olley e Pakes (II)}

\begin{itemize}
\item Além disso, os autores assumem uma regra de evolução para a produtividade,
denominada $\omega_{t}$, que seguiria um processo Markoviano de ordem
1. 
\item Também assume-se que a empresa maximiza o valor presente dos seus
lucros, o que nos dá a seguinte equação de Bellman:{\small{}
\[
V_{t}(\omega_{t},a_{t},k_{t})=\max\{\Phi,\sup\pi_{t}(\omega_{t},a_{t},k_{t})-c(i_{t})+\beta E[V_{t+1}(\omega_{t+1},a_{t+1},k_{t+1})|J_{t}\}
\]
}{\small \par}
\item Em que $\Phi$ representa o valor terminal da planta, $\pi_{t}$ representa
os lucros da empresa como função das variáveis de estado (produtividade,
idade e capital - supõe-se que o trabalho se ajuste instantaneamente
às mudanças nestas variáveis), e $c(i_{t})$ representa o custo do
ajustamento do estoque de capital -- denominado investimento. 
\item Enquanto isso,$\beta$ é o coeficiente de desconto intertemporal da
empresa e $J_{t}$ é o conjunto de informações disponível no instante
$t$.
\end{itemize}
\end{frame}

\begin{frame}\frametitle{Olley e Pakes (III):}

\begin{itemize}
\item O que esta relação quer dizer é que, se o valor presente dos lucros
da empresa, descontados adequadamente, for menor do que o valor terminal
dos ativos, ela decidirá sair do mercado e fechar a empresa. 
\item Caso isso não seja verdade, e ela decida manter-se no mercado, ela
acabará por estabelecer um nível de investimento consistente com a
operação continuada da empresa. 
\item Dadas as hipóteses sobre o comportamento do termo de produtividade,
podemos gerar uma regra de saída de mercado e uma demanda por investimentos.
A regra de saída é:
\[
\chi_{t}=\begin{cases}
1 & se\,\omega_{t}\geq\underline{\omega}_{t}(a_{t},k_{t})\\
0 & c.c.
\end{cases}
\]
\item E podemos construir uma função investimentos da seguinte forma:
\[
i_{t}=i_{t}(\omega_{t},a_{t},k_{t})
\]
\end{itemize}
\end{frame}

\begin{frame}\frametitle{Olley e Pakes (IV):}

\begin{itemize}
\item Segundo Pakes (1994, Teorema 27), temos que esta função, para quaisquer
valores do par $(a_{t},k_{t})$ é crescente em $\omega_{t}$. Desta
forma, definindo $h=i^{-1}$ a função inversa da anterior, podemos
escrever:
\[
\omega_{t}=h_{t}(i_{t},a_{t},k_{t})
\]
\item Ou seja, podemos escrever a produtividade como uma função do investimento,
da idade e do estoque de capital, o que nos permite fazer o seguinte:
\[
\ln(y_{it})=\beta_{0}+\beta_{a}\ln(a_{it})+\beta_{k}\ln(k_{it})+\beta_{l}\ln(l_{it})+\omega_{it}+\eta_{it}
\]
\item Substituindo a função produtividade em termos de $i_{t},a_{t},k_{t}$
na função de produção acima, temos que:
\[
\ln(y_{it})=\beta_{l}\ln(l_{it})+\phi_{t}(i_{it},a_{it},k_{it})+\eta_{it}
\]
\item Uma vez que a colocação da função $\omega_{t}$ vai absorver os logs
da idade, do investimento e do estoque de capital. 
\end{itemize}
\end{frame}

\begin{frame}\frametitle{Olley e Pakes (V):}

\begin{itemize}
\item Podemos reescrever a função $\phi(\cdot)$ como:
\[
\phi_{t}(i_{t},a_{t},k_{t})=\beta_{0}+\beta_{a}\ln(a_{it})+\beta_{k}\ln(k_{it})+h(i_{t},a_{t},k_{t})
\]
\item Aqui podemos fazer \textbf{o primeiro passo da metodologia de estimação
de Olley-Pakes}. 
\item Em especial, esta metodologia consiste em fazer uma regressão de $\ln(y_{it})$
contra $\ln(l_{it})$ e um polinômio de terceira ou quarta ordem em
$(i_{t},a_{t},k_{t})$. 
\item Ou seja, estimamos a seguinte regressão:{\tiny{}
\begin{eqnarray*}
\ln(y_{it}) & = & \beta_{0}+\beta_{l}\ln(l_{it})+\sum_{x=1}^{3}\gamma_{1x}(\ln(k_{it}))^{x}+\sum_{x=1}^{3}\gamma_{2x}(\ln(i_{it}))^{x}+\sum_{x=1}^{3}\gamma_{3x}(\ln(a_{it}))^{x}+\\
 &  & +\sum_{x=1}^{3}\gamma_{4x}(\ln(k_{it})\times\ln(a_{it}))^{x}+\sum_{x=1}^{3}\gamma_{5x}(\ln(k_{it})\times\ln(i_{it}))^{x}+\sum_{x=1}^{3}\gamma_{6x}(\ln(i_{it})\times\ln(a_{it}))^{x}
\end{eqnarray*}
}{\tiny \par}
\item Com isto aqui, conseguimos estimar consistentemente o coeficiente
$\beta_{l}$, que é o de ajustamento imediato.
\end{itemize}
\end{frame}

\begin{frame}\frametitle{Olley e Pakes (VI):}

\begin{itemize}
\item No entanto, precisamos ainda dos coeficientes dos outros elementos.
Para estimarmos os coeficientes de $\beta_{a}$ e $\beta_{k}$, precisamos
das estimativas de $\beta_{l}$, bem como de $\hat{\phi}$ e das probabilidades
de sobrevivência da empresa (ou seja, estar no mercado, dado que esteve
no período anterior). 
\item Definindo esta variável por $\chi_{t+1}$, podemos escrever um probit
da seguinte forma:{\tiny{}
\begin{eqnarray*}
Prob(\chi_{t+1}=1) & = & \sum_{x=1}^{3}\gamma_{1x}(\ln(k_{it}))^{x}+\sum_{x=1}^{3}\gamma_{2x}(\ln(i_{it}))^{x}+\sum_{x=1}^{3}\gamma_{3x}(\ln(a_{it}))^{x}+\\
 &  & +\sum_{x=1}^{3}\gamma_{4x}(\ln(k_{it})\times\ln(a_{it}))^{x}+\sum_{x=1}^{3}\gamma_{5x}(\ln(k_{it})\times\ln(i_{it}))^{x}+\sum_{x=1}^{3}\gamma_{6x}(\ln(i_{it})\times\ln(a_{it}))^{x}
\end{eqnarray*}
}{\tiny \par}
\item A partir daí, são calculados os valores previstos da probabilidade
de sobrevivência. Caso tenhamos uma base de dados balanceada, este
estágio não seria necessário. Com este modelo, são estimadas as probabilidades
de ocorrência daquele evento, o que denominaremos $\hat{P}_{t}$. 
\end{itemize}
\end{frame}

\begin{frame}\frametitle{Olley e Pakes (VII):}

\begin{itemize}
\item Finalmente, podemos estimar os coeficientes do capital e da idade. 
\item Em especial, isto é conseguido por meio de estimação não-linear do
seguinte modelo:
\tiny
\begin{eqnarray*}
\ln(y_{it})-\hat{\beta}_{l}\ln(l_{it}) & = & c+\beta_{a}\ln(a_{it})+\beta_{k}\ln(k_{it})+\sum_{j=0}^{3-m}\sum_{m=0}^{3}\beta_{mj}\hat{h}_{t-1}^{m}\hat{P}_{t-1}^{j}+e_{it}\\
\hat{h}_{t-1} & = & \hat{\phi}_{t-1}+\beta_{a}\ln(a_{it-1})+\beta_{k}\ln(k_{it-1})
\end{eqnarray*}
\normalsize
\item Finalmente, os autores calculam a produtividade da seguinte forma:
\[
p_{t}=\exp[\ln(y_{it})-\beta_{l}\ln(l_{it})-\beta_{k}\ln(k_{it})-\beta_{a}\ln(a_{it})]
\]
\end{itemize}
\end{frame}

\subsection{Levinsohn e Petrin}

\begin{frame}\frametitle{Levinsohn e Petrin (2000)}
\begin{itemize}
\item \citet{Levinsohn2000} levantam um ponto importante sobre os resultados
de Olley e Pakes (1996): nem sempre o investimento responde integralmente
aos choques de produtividade das empresas. 
\item Uma vez que o investimento é uma variável de controle sobre uma variável
de estado (o estoque de capital), em geral ela é custosa de ajustar. 
\item Estes custos de ajustamento podem ser de tal ordem que tornam a inversão
proposta por Olley e Pakes (1996) para a obtenção da produtividade
como função dos investimentos e do capital e da idade inviável. 
\end{itemize}
\end{frame}

\begin{frame}\frametitle{Levinsohn e Petrin (2000)}

\begin{itemize}
\item Em primeiro lugar, a produtividade pode possuir um componente previsível
e um componente não previsível. Se parte do componente previsível,
o ajustamento nas variáveis de estado (o capital) se dará específicamente
sobre ela. Neste caso, o investimento somente responderia à parte
não esperada do choque de produtividade. Além disso, como o trabalho
é suposto que se ajuste instantaneamente, provavelmente ele se ajusta
aos dois componentes, criando assim simultaneidade.
\item Outro cenário em que isso pode ocorrer é quando a produtividade possui
um componente i.i.d. Neste caso, as expectativas sobre o futuro não
são ajustadas, ainda que afetam os valores dos fatores variáveis.
Além disso, como os choques de produtividades são i.i.d., os investimentos
não se alteram em resposta a estes componentes da produtividade.
\end{itemize}
\end{frame}

\begin{frame}\frametitle{Levinsohn e Petrin (II)}

\begin{itemize}
\item Para resolver este problema, Levinsohn e Petrin utilizam o valor de
alguns insumos intermediários como \emph{proxies} para resolver este
problema. Em especial, considerando a seguinte função de produção:
\[
y_{it}=\beta_{0}+\beta_{k}k_{it}+\beta_{l}l_{it}^{s}+\beta l_{it}^{u}+\beta_{m}m_{it}+\beta_{f}f_{it}+\beta_{e}e_{it}+\omega_{it}+\eta_{it}
\]
\item Em que as variáveis são logs de:

\begin{itemize}
\item $y_{it}$ - Valor da Produção
\item $k_{it}$ - Valor do estoque de capital da planta da empresa
\item $l_{it}^{s}$ - Valor da mão-de-obra qualificada
\item $l_{it}^{u}$ - Valor da mão-de obra não qualificada
\item $m_{it}$ - Valor das matérias-primas
\item $f_{it}$ - Valor dos combustíveis
\item $e_{it}$ - Valor da eletricidade consumida
\end{itemize}
\end{itemize}
\end{frame}

\begin{frame}\frametitle{Levinsohn e Petrin (III)}

\begin{itemize}
\item Agora, Levinsohn e Petrin fazem a hipótese que a demanda de insumos
intermediários -- no caso, energia elétrica --
é uma função da produtividade e do estoque de capital (as duas variáveis
de estado do nosso problema dinâmico):
\[
e_{t}=e(\omega_{t},k_{t})
\]
\item Como eles mostram, a função $e$ , para qualquer valor de $k_{t}$,
é crescente em $\omega_{t}$, o que indica que podemos inverter a
função $e$:
\[
\omega_{t}=e^{-1}(e_{t},k_{t})
\]
\end{itemize}
\end{frame}

\begin{frame}\frametitle{Levinsohn e Petrin (IV)}

\begin{itemize}
\item Fazendo esta inversão, a função de produção fica assim:
\end{itemize}
\[
y_{it}=\beta_{k}k_{it}+\beta_{l}l_{it}^{s}+\beta l_{it}^{u}+\beta_{m}m_{it}+\beta_{f}f_{it}+\phi(e_{it},k_{it})+\eta_{it}
\]

\begin{itemize}
\item Em que:
\[
\phi_{it}(e_{it},k_{it})=\beta_{0}+\beta_{k}k_{it}+\omega_{t}(e_{t},k_{t})
\]
\item Ao invés de usar aproximações polinomiais para limpar os efeitos deste
negócio, Levinsohn e Petrin utilizarão um método não-paramétrico diferente
para a estimação (\emph{locally weighted least squares}). 
\end{itemize}
\end{frame}

\begin{frame}\frametitle{Levinsohn e Petrin (V)}

\begin{itemize}
\item O que eles fazem é calcular o valor esperado de cada uma das variáveis,
em função dos termos $e_{it}$ e $k_{it}$, e aí fazer a diferença:{\scriptsize{}
\begin{eqnarray*}
y_{it}-E(y_{it}|e_{it},k_{it}) & = & \beta_{s}(l_{it}^{s}-E(l_{it}^{s}|e_{it},k_{it}))+\beta_{u}(l_{it}^{u}-E(l_{it}^{u}|e_{it},k_{it}))+\\
 & + & \beta_{m}(m_{it}-E(m_{it}|e_{it},k_{it}))+\beta_{f}(f_{it}-E(f_{it}|e_{it},k_{it}))+\eta_{it}
\end{eqnarray*}
}{\scriptsize \par}
\item Para obtermos o coeficiente $\beta_{e}$, os autores oferecem dois
caminhos.

\begin{itemize}
\item Supondo o que os autores chamam de separabilidade, não precisamos
estimar os coeficientes para o insumo intermediário cujo consumo serve
de instrumento. 
\item Este coeficiente seria exatamente igual à elasticidade do produto
com relação a ele. No caso de uma função Cobb-Douglas, isto implica
que o coeficiente deste insumo é igual á participação do mesmo na
receita:
\[
\beta_{e}=s_{e}
\]
\end{itemize}
\end{itemize}
\end{frame}

\begin{frame}\frametitle{Levinsohn e Petrin (VI)}

\begin{itemize}
\item Neste caso, o processo se facilita muito. Precisamos apenas obter
um coeficiente para a variável capital. Isto será obtido por GMM. 
\item Dados os coeficientes estimados antes, podemos reescrever, para um
valor qualquer do parâmetro $\beta_{k}$:
\[
y_{it}-\beta_{l}l_{it}^{s}-\beta l_{it}^{u}-\beta_{m}m_{it}-\beta_{f}f_{it}-s_{e}e_{it}-\beta_{k}^{*}k_{it}
\]
\item Sabemos que este negócio é igual aos choques de produtividade mais
os componentes aleatórios $\omega_{it}+\eta_{it}$. 
\item Pela suposição de que a produtividade segue um processo de Markov,
temos que $\omega_{it}=E(\omega_{it}|\omega_{it-1})+\xi_{it}$. Desta
forma, temos que:
\[
y_{it}-\beta_{l}l_{it}^{s}-\beta l_{it}^{u}-\beta_{m}m_{it}-\beta_{f}f_{it}-s_{e}e_{it}-\beta_{k}^{*}k_{it}-E(\omega_{it}|\omega_{it-1})=\xi_{it}+\eta_{it}
\]
\end{itemize}
\end{frame}

\begin{frame}\frametitle{Levinsohn e Petrin (VII)}

\begin{itemize}
\item Tudo estaria tranquilo, se soubéssemos o valor de $E(\omega_{it}|\omega_{it-1})$;
no entanto, ainda não o sabemos, de forma que temos que estimar esta
parada. Considerando que o termo $\eta_{it}$ é composto pela parte
aleatória da função de produção, podemos afirmar que:
\[
E(\omega_{it}+\eta_{it}|\omega_{it-1})=E(\omega_{it}|\omega_{it-1})
\]
\item Assim, podemos usar como estimativa de $\omega_{it}+\eta_{it}$ o
seguinte:
\[
y_{it}-\beta_{l}l_{it}^{s}-\beta l_{it}^{u}-\beta_{m}m_{it}-\beta_{f}f_{it}-s_{e}e_{it}-\beta_{k}^{*}k_{it}
\]
\item Agora, para a estimativa de $\omega_{it-1}$, podemos fazer o seguinte:
\[
\hat{\omega}_{it-1}=\hat{\phi}_{it-1}-s_{e}e_{it-1}-\beta_{k}^{*}k_{it-1}
\]
\end{itemize}
\end{frame}

\begin{frame}\frametitle{Levinsohn e Petrin (VIII)}

\begin{itemize}
\item Com estes valores, podemos utilizar mínimos quadrados ponderados locais
para calcular $E(\omega_{it}|\omega_{it-1})$, que denominaremos $\hat{\Omega}$.
\item Usando esta estimativa, podemos calcular os valores de $\xi_{it}+\eta_{it}$,
para os valores candidatos de $\beta_{k}$. Finalmente, podemos calcular
a função critério:
\[
q=\left(\sum_{i}\sum_{t=t_{0}+1}^{T}(\xi_{it}+\eta_{it})k_{it}\right)^{2}
\]
\item Com isso, estimamos o coeficiente do capital.
\end{itemize}
\end{frame}

\begin{frame}\frametitle{Wooldridge}

\begin{itemize}
\item Neste paper, \citet{Wooldridge2009} se volta para a questão de como implementar
os estimadores OP e LP em um contexto de GMM.
\item Ele mostra que é possível estimar os parâmetros na forma de um sistema
de equações não lineares.
\item Esse negócio é implementável no GMM do Stata.
\end{itemize}
\end{frame}

\section{Ackerberg-Caves-Frazer}
\begin{frame}\frametitle{ACF}

\begin{itemize}
\item \citet{Ackerberg2005c} criticam OP e LP e propõem uma alternativa.
\item A principal crítica está relacionada com o que exatamente se consegue
com a utilização de medidas de insumos intermediários como proxies
para a produtividade.
\item O principal ponto pode ser visto quando a gente olha para a forma
funcional do primeiro estágio de OP e LP:
\[
y_{it}=\beta_{L}l_{it}+\phi_{t}(m_{it},k_{it})+\varepsilon_{it}
\]
\item Como a gente consegue variação em $l_{it}$ independente de $\phi_{t}$?
Porque se a gente acertasse direitinho $\phi_{t}$, provavelmente
esse negócio deveria andar muito junto.
\end{itemize}
\end{frame}

\begin{frame}\frametitle{ACF (II)}

\begin{itemize}
\item Eles mostram que a gente só consegue esta variação independente --
e a identificação de $\beta_{L}$ se a gente assumir algumas hipóteses
sobre o timing da decisão sobre a quantidade de insumos intermediários
(e trabalho também).
\item Por exemplo, caso $m_{it}$ seja escolhido conjuntamente com $l_{it}$
não temos variação independente.
\item Por outro lado, se $m_{it}$ for escolhido em um instante do tempo
diferente de $l_{it}$ mas $\omega_{it}$ mexe nesse meio tempo, vc
tem variação independente, mas não consegue recuperar o $\omega_{it}$correto
\end{itemize}
\end{frame}

\begin{frame}\frametitle{ACF (III)}

\begin{itemize}
\item ACF sugerem dois casos em que isso pode funcionar:
\end{itemize}
\begin{enumerate}
\item Erro de medida em $l_{it}$ com $m_{it}$ sendo escolhido simultaneamente
\item $l_{it}$ é escolhido depois de $m_{it}$ e $\omega$ não varia neste
meio tempo
\item O caso que eles usam é assumir que as formas escolhem $l_{it}$ no
instante $t-b$ , com $(0<b<1)$, o que implica que a empresa escolhe
a quantidade de trabalho depois do estoque de capital $k_{it}$ ter
sido determinado em $t-1$, por $k_{it-1}+i_{it-1}$, mas antes de
a quantidade de insumos $m_{it}$ forem escolhidos. 

\begin{enumerate}
\item Com esta ordem nas decisões de escolha, o $l_{it}$ pode ser uma variável
de estado e determinando a quantidade de insumos intermediários.
\end{enumerate}
\end{enumerate}
\end{frame}

\begin{frame}\frametitle{Timing}

\begin{tikzpicture}
\draw (0,0) -- (9,0);

\foreach \x in {0,5,9}
   \draw (\x cm,3pt) -- (\x cm,-3pt);

\draw (0,0) node[below=3pt] {\footnotesize{$t-1$}} node[above=18pt] {\small{$k_{it}=k_{it-1}+i_{it-1}$}};
\draw (0,0) node[above=3pt] {\small{$\mathbb{I}_{t-1}=\omega_{t-1}$}};
\draw (5,0) node[below=3pt] {\footnotesize{$t-b$}} node[above=3pt] {\small{$E(\omega_{t-b}|\mathbb{I}_{t-1})=E(\omega_{t-b}|\omega_{t-1})$}} node[above=18pt] {$l_{it}$};
\draw (9,0) node[below=3pt] {\footnotesize{$t$}} node[above=3pt] {\tiny{$E(\omega_{t}|\mathbb{I}_{t-b})=E(\omega_{t}|\omega_{t-b})$}} node[above=18pt] {\tiny{$m_{it}=h_{t}(\omega_{it},k_{it},l_{it}$)}};

\end{tikzpicture}

\end{frame}

\begin{frame}\frametitle{ACF (IV)}

\begin{itemize}
\item Eles sugerem o seguinte algoritmo:
\item Primeiro, considerando a seguinte função de produção:
\[
y_{it}=\alpha+\beta_{L}l_{it}+\beta_{K}k_{it}+\omega_{it}+\varepsilon_{it}
\]
\item Sendo que:

\begin{itemize}
\item $i_{it}$ e $k_{it}$ determinados em $t-1$
\item $l_{it}$ em $t-b,0<b<1$
\item $m_{it}$ determinado em $t$
\end{itemize}
\end{itemize}
\end{frame}

\begin{frame}\frametitle{ACF (V):}

\begin{itemize}
\item Além disso, os choques de produtividade seguem a seguinte regra:
\begin{eqnarray*}
P(\omega_{it}|I_{it-b}) & = & P(\omega_{it}|\omega_{it-b})\\
P(\omega_{it-b}|I_{it-1}) & = & P(\omega_{it-b}|\omega_{it-1})
\end{eqnarray*}
\item Com todas estas premissas, podemos escrever a escolha de $m_{it}$
como sendo uma função $f(l_{it},\omega_{it},k_{it})$, podendo ser
invertida. Ou seja, a função de produção fica:
\[
y_{it}=\alpha+\beta_{L}l_{it}+\beta_{K}k_{it}+f^{-1}(l_{it},m_{it},k_{it})+\varepsilon_{it}
\]
\end{itemize}
\end{frame}

\begin{frame}\frametitle{Procedimento ACF}

\begin{itemize}
\item Estágio 1 - Regredir $y_{it}$ em uma função não-paramétrica de $l_{it}$,
$m_{it}$ e $k_{it}$
\item Estágio 2 - Estimar $\beta_{L}$ e $\beta_{K}$ usando os seguintes
momentos:
\[
E(\xi(\beta_{K},\beta_{L})|k_{it},l_{it-1})=0
\]
\item Sendo que $\xi_{it}=\omega_{it}-E(\omega_{it}|\omega_{it-1})$ --
por exemplo:

\begin{itemize}
\item $\xi(\beta_{k},\beta_{l})$ com uma regressão não paramétrica de $\hat{\phi}_{it}-\beta_{K}k_{it}-\beta_{L}l_{it}$
em $\hat{\phi}_{it-1}-\beta_{K}k_{it-1}-\beta_{L}l_{it-1}$ 
\end{itemize}
\end{itemize}
\end{frame}

\begin{frame}\frametitle{Visão geral}

\begin{enumerate}
\item OP/LP/ACF permite um processo geral, não apenas um AR(1) para a produtividade.
\item DPD pode ter efeitos fixos de empresa enquanto OP não (as condições
de momento não seriam válidas).
\item DPD não requer a condição de monotonicidade.
\item DPD pode ter suas premissas testadas diretamente
\end{enumerate}
\end{frame}

%\begin{comment}
\begin{frame}[allowframebreaks]
\bibliographystyle{aea}
\bibliography{C:/Bibliog/library}

\end{frame}

%\end{comment}




\end{document}


