\documentclass{beamer}
\usepackage{beamerthemesplit}
\usepackage[brazil]{babel}
\usepackage{epsfig}
\usepackage[utf8x]{inputenc}
\usepackage{pgf}
%\usepackage{tikz}
%\usetikzlibrary{snakes}
\usepackage{nicefrac}
\usepackage{amsfonts}
\usepackage{amsmath}
\usepackage{amssymb}
\usepackage{amsthm}
%\usepackage{float}
\usetheme{Frankfurt}
\usepackage{epstopdf}
\usepackage{comment}
\usepackage{natbib}
\usepackage{float}
\usepackage{graphicx}
\usepackage{booktabs}
\usepackage{array}
\title{Aula 04}

\subtitle{Função de Produção}

\author{Claudio R. Lucinda}


\institute{FEA-RP/USP}

\date{}
\logo{\includegraphics[scale=.1]{logousp.png}}
\beamertemplatenavigationsymbolsempty
\begin{document}

\frame{\titlepage}
\begin{frame}\frametitle{Agenda}
  \tableofcontents[pausesections]
\end{frame}


\section{Produção e Custos -- Introdução}
\begin{frame}\frametitle{Produção e Custos}

\begin{itemize}
\item Agora vamos mudar o foco de nossa análise. Até o momento, estávamos
preocupados com a modelagem do comportamento do produtor; agora, nos
preocuparemos com o comportamento do produtor. 
\item Nesta aula, nos preocuparemos inicialmente com a derivação das principais
primitivas do comportamento do produtor: a função de produção e a
função custos, para apresentar algumas formas funcionais comuns para
a modelagem da função custos. 
\item Peculiaridades da modelagem da decisão do produtor:

\begin{itemize}
\item Viés de Seleção da Amostra: a atrição da amostra não é completamente
aleatória.
\item Problemas de Endogeneidade entre a quantidade produzida e o uso dos
insumos.
\item Ineficiência
\end{itemize}
\end{itemize}
\end{frame}

\begin{frame}\frametitle{Pontos Interessantes na Modelagem de Produção:}

\begin{itemize}
\item Em geral, na análise da produção e custos, estamos interessados nos
seguintes elementos:

\begin{itemize}
\item Escala: Verificação de se a empresa ou o setor exibem retornos constantes
de escala, crescentes ou decrescentes;
\item Substituição: O grau de substituição dos fatores de produção em resposta
a alterações na quantidade produzida;
\item Separabilidade: A capacidade de separação das relações de produção
-- ou de custos -- em componentes aninhados
ou aditivos.
\item Progresso Técnico: Mudança na forma pela qual os fatores de produção
são combinados para a produção.
\item Distribuição da renda: Como as parcelas da renda se distribuem entre
os fatores de produção;
\item Custo Marginal: Obtenção de estimativas de custos marginais para as
análises subsequentes. 
\end{itemize}
\end{itemize}
\end{frame}

\section{Formas Funcionais -- Produção e Custos}
\begin{frame}\frametitle{Formas Funcionais}

\begin{itemize}
\item Agora, começaremos a detalhar as formas funcionais mais comumente
utilizadas para a modelagem de produção e custos.
\item Em certo sentido, a modelagem que colocaremos é próxima das escolhas
feitas no contexto da modelagem de funções de utilidade indireta:
uma vez que não é de se esperar que observemos diretamente a primitiva
relevante, tentaremos aproximar qual seria a ``verdadeira'' a partir
dos dados observados.

\begin{itemize}
\item Afinal de contas, NÃO EXISTE UMA FUNÇÃO DE PRODUÇÃO ESCONDIDA EM ALGUM
ARMÁRIO EM CADA EMPRESA
\end{itemize}
\end{itemize}
\end{frame}

\begin{frame}\frametitle{Formas Funcionais Flexíveis}

\begin{itemize}
\item Todas estas funções podem ser vistas como expansões lineares em parâmetros
que podem aproximar uma função arbitrária. Esta expansão pode ser
vista na seguinte forma:
\[
f^{*}(\mathbf{x})\approx f(\mathbf{x})=\sum_{i=1}^{N}a_{i}h^{i}(\mathbf{x})
\]
\item Em que os $a_{i}$ eram parâmetros, os $h^{i}$ são funções conhecidas
e os $\mathbf{x}$ são vetores de variáveis. 
\item Se algumas condições são satisfeitas para uma dada realização do vetor
$\mathbf{x}^{*}$, podemos dizer que $f(\mathbf{x}^{*})$ é uma aproximação
da função verdadeira no ponto. 
\item Além disso, aproxima os valores da primeira e segunda derivadas da
função também. Consideramos esta uma forma funcional flexível parsimoniosa.
\end{itemize}
\end{frame}

\begin{frame}\frametitle{Forma Funcional Flexível Parsimoniosa -- Problemas}

\begin{itemize}
\item Um cuidado adicional: quando estamos estimando uma função como esta
com uma base de dados com um domínio extensivo -- ou seja,
com valores que mapeiam muito do quadrante relevante da variável $\mathbf{x}$
-- é bem provável que a função obtida não será uma aproximação
de segunda ordem da função de produção verdadeira em qualquer ponto. 
\item Como resultado, os efeitos de estática comparativa resultantes podem
ser bem diferentes dos resultados da função verdadeira. 
\item Ou seja, podemos rejeitar uma hipótese mesmo quando a função verdadeira
não rejeitaria.
\end{itemize}
\end{frame}

\begin{frame}\frametitle{Formas Funcionais -- Exemplos:}

\tiny
\begin{tabular}{>{\centering}p{3cm}>{\centering}p{5cm}>{\centering}p{3cm}}
\hline 
{\tiny{}Forma Funcional} & {\tiny{}Fórmula} & {\tiny{}Restrições}\tabularnewline
\hline 
{\tiny{}Cobb-Douglas (Cobb e Douglas 1928)} & {\tiny{}$\ln y=a_{0}+\sum_{j=1}^{J}a_{j}\ln z_{j}$} & {\tiny{}$\sum_{j=1}^{J}a_{j}=1$ para homog. lin.}\tabularnewline
{\tiny{}CES (Arrow et. al 1961)} & {\tiny{}$y^{\rho}=a_{0}+\sum_{j=1}^{J}a_{j}z_{j}^{\rho}$} & {\tiny{}$a_{0}=0$ para Homog.lin.}\tabularnewline
{\tiny{}Leontief/Linear Generalizada (Diewert 1971)} & \textrm{\tiny{}$y=a_{0}+\sum_{j=1}^{J}a_{j}\sqrt{z_{j}}+\sum_{k=1}^{J}\sum_{j=1}^{J}a_{kj}\sqrt{z_{k}z_{j}}$} & {\tiny{}$a_{i}=0,i=0,\cdots J$ para Homog. Lin.}\tabularnewline
{\tiny{}Translog (Christensen, Jorgenson e Lau (1971))} & \textrm{\tiny{}$\ln y=a_{0}+\sum_{j=1}^{J}a_{j}\ln z_{j}+\sum_{k=1}^{J}\sum_{j=1}^{J}a_{kj}\ln z_{k}\ln z_{j}$} & {\tiny{}$\sum a_{j}=1$ e $\sum a_{ij}=0$para Homog. Lin.}\tabularnewline
{\tiny{}Cobb-Douglas Generalizada (Diewert (1971))} & {\tiny{}$\ln y=a_{0}+\sum_{k=1}^{J}\sum_{j=1}^{J}a_{jk}\ln((z_{k}+z_{j})/2)$} & {\tiny{}$\sum_{k}\sum_{j}a_{jk}=1$para H. L.}\tabularnewline
{\tiny{}Quadrática (Lau (1974))} & \textrm{\tiny{}$y=a_{0}+\sum_{j=1}^{J}a_{j}z_{j}+\sum_{k=1}^{J}\sum_{j=1}^{J}a_{kj}z_{k}z_{j}$} & \tabularnewline
{\tiny{}Côncava Generalizada (McFadden (1974))} & {\tiny{}$y=\sum_{k=1}^{J}\sum_{j=1}^{J}z_{j}\phi^{kj}\left(\frac{z_{k}}{z_{j}}\right)a_{kj}$} & {\tiny{}$\phi^{kj}$ é uma função côncava conhecida}\tabularnewline
\hline 
\end{tabular}{\scriptsize \par}
\end{frame}

\section{Estimação de Funções de Produção}
\begin{frame}\frametitle{Estimação de Funções de Produção}

\begin{itemize}
\item A estimação das funções de produção começou com o trabnalho de Cobb
e Douglas (1928), que buscavam testar as implicações da teoria da
distribuição baseada na produtividade marginal dos fatores.
\item A principal crítica deste tipo de literatura é que os dados sobre
fatores de produção, quando estamos falando em quantidades agregadas,
são determinados simultaneamente aos valores do produto; 
\item desta forma, a função de produção não seria identificável. 
\item Vamos ilustrar este ponto mais detalhadamente, considerando a seguinte
equação:,
\[
q=a+\alpha z+\beta x+u
\]
\item Em que $q$ é o log da quantidade produzida, $z$ é o log do capital
(ou qualquer outra quantidade de fatores ``fixos'' de produção),
e $x$ o log de todos os insumos variáveis. 

\begin{itemize}
\item O problema, levantado já em 1944 por Marschak e Andrews, é que não
podemos tratar $x$ e $z$ como verdadeiramente exógenos e estimar
este negócio por OLS. 
\end{itemize}
\end{itemize}
\end{frame}

\begin{frame}\frametitle{Estimação de Funções de Produção (II):}
\small
\begin{itemize}
\item A demanda pelo insumo variável, supondo que as empresas escolham as
quantidades de $x$ ao observar a realização de $u$, é dada por:
\[
X=\left[\frac{p}{w}\beta e^{a+u}Z^{\alpha}\right]^{\frac{1}{1-\beta}}
\]
\item Uma vez que a escolha de $X$ depende de $u$, temos problemas de
endogeneidade.
\item Um segundo problema é o da seleção de amostra. Um exemplo clássico
é o de Dunne, Roberts e Samuelson (1988) encontrou taxas de saída
maiores do que 30\% entre intervalos de 5 e 5 anos.
\item É de se supor que o principal determinante deste padrão de saída não
é o componente aleatório ortogonal à escolha das variáveis.

\begin{itemize}
\item Pelo contrário! É de se supor que as decisões da empresa tenham papel
preponderante nas decisões de saída (i.e., falência) das empresas.
\end{itemize}
\end{itemize}
\end{frame}

\begin{frame}\frametitle{Endogeneidade da Função de Produção}

\scriptsize
\begin{itemize}
\item Dois exemplos de endogeneidade como a mencionada no slide anterior:

\begin{enumerate}
\item Vamos supor que observemos um \emph{cross section }de empresas. Algumas
delas são mais produtivas e têm melhores gestores.E por isso, elas
podem precisar de menos trabalho para produzir a mesma quantidade.
Ou seja, estas empresas vão produzir mais com menos trabalho e por
isso OLS vai subestimar $\beta_{l}$
\item Suponha que, agora observamos um painel e, em cada período a empresa
tem um choque de produtividade -- positivo por ela observado
e com este valor vai contratar mais. Ou seja, no final o aumento de
produção com o choque de produtividade vai ser devido às duas coisas
mas OLS vai atribuir TODO o aumento de produção ao aumento de trabalho,
sobrestimando $\beta_{l}$
\end{enumerate}
\item Ou seja, pode ir para qualquer direção. 
\item Usualmente, assumimos que o problema da endogeneidade é mais presente
no trabalho.
\end{itemize}
\end{frame}

\begin{frame}\frametitle{Atrição da Amostra em Funções de Produção}

\begin{itemize}
\item Pra ilustrar melhor este ponto, suponha que as empresas sejam monpólios
que são dotados exogenamente de diferentes quantidades de capital.
\item Desta forma, dependendo do valor de $u$, elas podem decidir sair
ou não. 

\begin{itemize}
\item Ou seja, se $u$ for ``muito ruim'', pode ser melhor vender o valor
residual da empresa.
\end{itemize}
\item Isto pode ser racionalizado com a seguinte regra de saída:
\[
\chi(u,Z,p,w,a,\beta,\alpha)=0\,se\,\Pi(u,Z,p,w,a,\beta,\alpha)<\Psi
\]
\item Em que $\Pi$ é a parte variável dos lucros e $\Psi$o valor residual
da empresa.
\end{itemize}
\end{frame}

\begin{frame}\frametitle{Atrição de Amostra em Funções de Produção}

\begin{itemize}
\item O ponto aqui é que esta condição gerará uma correlação entre $u$
e $Z$ condicional à empresa estar no mercado.
\item Isto ocorre porque as empresas com maiores estoques de capital devem
ter maiores lucros variáveis e, portanto, podem suportar piores choques
$u$ sem sair do mercado.

\begin{itemize}
\item Ou seja, devemos observar apenas aquelas empresas em que o $Z$ é
relativamente grande e/ou $u$ relativamente pequeno.
\item Isso implica que as empresas menores devem sair da amostra
\end{itemize}
\end{itemize}
\end{frame}

\begin{frame}\frametitle{Soluções Tradicionais para o Problema:}

\begin{itemize}
\item Existem duas formas de lidar com alguns dos problemas mencionados
aqui:

\begin{itemize}
\item Aproveitamento de amostra de dados em painel
\item Utilização de Variáveis Instrumentais
\end{itemize}
\item Vamos representar nosso modelo da seguinte forma:
\[
y_{it}=\beta_{0}+\beta_{k}k_{it}+\beta_{l}l_{it}+\omega_{it}+\eta_{it}
\]
\item Em que $\omega_{it}$ representa a parte de informação não observada
pelo econometrista que é observada pela empresa na tomada de suas
decisões, e $\eta_{it}$ representa a parte da informação não observada
pelo econometrista que também não é observada pela empresa.

\begin{itemize}
\item $\omega_{it}$: capacidade gerencial
\item $\eta_{it}$: comportamento anômalo.
\end{itemize}
\end{itemize}
\end{frame}

\begin{frame}\frametitle{Solução I -- Dados em Painel}

\begin{itemize}
\item Uma solução interessante para o problema da endogeneidade é utilizar
a informação da estrutura em painel dos dados.
\item Aqui estamos considerando que a parte $\omega_{it}$ é constante ao
longo do tempo
\item Neste caso, podemos usar os diferentes estimadores mencionados em
Wooldridge (2002), e que alguns de vocês viram no curso de Econometria
com Dados em Painel:

\begin{itemize}
\item Primeiras Diferenças: $(y_{it}-y_{it-1})=\beta_{k}(k_{it}-k_{it-1})+\beta_{l}(l_{it}-l_{it-1})+(\eta_{it}-\eta_{it-1})$
\item Efeitos Fixos: $(y_{it}-\bar{y}_{i})=\beta_{k}(k_{it}-\bar{k}_{i})+\beta_{l}(l_{it}-\bar{l}_{i})+\eta_{it}$
\end{itemize}
\end{itemize}
\end{frame}

\begin{frame}\frametitle{Dados em Painel:}

\begin{itemize}
\item Dada a hipótese que $\eta_{it}$ são independentes das escolhas de
insumos em qualquer instante do tempo, podemos estimar as duas equações
por OLS.

\begin{itemize}
\item Esta hipótese é a chamada ``exogeneidade estrita''. Em alguns casos,
podemos estimar este modelo de efeitos fixos sob a premissa de ``exogeneidade
seqüencial'', em que $\eta_{it}$ não é correlacionado com a escolha
de insumos nos instantes anteriores à $t$.
\end{itemize}
\item Esta premissa de $\omega$ constante ao longo do tempo também resolveria
o problema da atrição da amostra, caso a regra de saída dependa somente
de $\omega$, e não de $\eta_{it}$.
\item No entanto, existem algumas limitações da abordagem com dados em painel.
\end{itemize}
\end{frame}

\begin{frame}\frametitle{Dados em Painel -- Limitações}

\begin{itemize}
\item É uma premissa complicada assumir que os $\omega$ sejam constantes
ao longo do tempo, especialmente quando bases de microdados mais longas
estão disponíveis.
\item Além disso, pode haver interesse nas mudanças em $\omega$ propriamente
dito.
\item Outro problema é que, quando há erros de medida nos insumos, os estimadores
de dados em painel podem gerar estimativas piores que OLS - em especial,
$\beta_{k}$ muito baixos

\begin{itemize}
\item Griliches e Hausman (1986) mostram que quando os insumos são mais
correlacionados que os erros de medida, pode se reduzir a razão sinal/ruído
nas variáveis independentes (a parcela da variabilidade mais devida
a alterações na variável mesmo do que nos erros de medida).
\end{itemize}
\item Um terceiro problema é que, em geral, efeitos fixos dão estimativas
muito baixas para os coeficientes de retornos de escala.

\begin{itemize}
\item Além disso, os dados mudam muito se pegamos o painel inteiro ou apenas
a parte balanceada do mesmo.
\end{itemize}
\end{itemize}

\note[item]{Se $\omega$ fosse mesmo constante, e a saída da amostra dependesse
somente dele, deveríamos ter resolvido o problema de endogeneidade
induzido pela atrição da amostra e, portanto, o padrão de saída seria
muito aleatório. Neste caso, se pegássemos a parte balanceada do painel
e o painel completo, deveríamos ter resultados muito similares.}

\end{frame}

\begin{frame}\frametitle{Solução II -- Variáveis Instrumentais:}

\begin{itemize}
\item As abordagens de variáveis instrumentais se baseiam na premissa que
é possível encontrar instrumentos adequados.
\item Alguns instrumentos ``naturais''

\begin{itemize}
\item Preços dos fatores de produção: se eles forem independentes de $\omega$,
tudo bem
\end{itemize}
\item Estamos, neste caso, assumindo que não existe poder de mercado por
parte das empresas na aquisição de insumos.
\item No entanto, existem problemas com esta abordagem:

\begin{itemize}
\item Preços pagos por insumos não são reportados pelas empresas
\item Nem sempre há variação econometricamente ``boa'' nestas variáveis
\item É difícil imaginar que $\omega$ não seja afetado pelos preços dos
insumos
\item Não resolve a questão da saída
\end{itemize}
\end{itemize}
\end{frame}

\begin{frame}\frametitle{Solução III -- Painéis Dinâmicos}
\small
\begin{itemize}
\item Uma linha de ataque aos problemas mencionados anteriormente envolve
a estimação de modelos de painel dinâmico. 
\item Vamos começar supondo o seguinte modelo:
\begin{eqnarray*}
y_{it} & = & \gamma_{t}+\beta_{k}k_{it}+\beta_{l}l_{it}+f_{i}+\eta_{it}\\
\eta_{it} & = & \rho\eta_{it-1}+\epsilon_{it}\\
\epsilon_{it} & \sim & MA(0)
\end{eqnarray*}
\item Assume-se que a parte da produtividade tenha um componente aleatório
e um componente persistente -- para refletir o fato que
a produtividade apresenta forte persistência ao longo do tempo.
\item Este modelo tem uma representação dinâmica da seguinte forma:
\begin{eqnarray*}
y_{it} & = & \beta_{l}l_{it}-\rho\beta_{l}l_{it-1}+\beta_{k}k_{it}-\rho\beta_{k}k_{it-1}+\rho y_{it-1}+\\
 & + & (\gamma_{t}-\rho\gamma_{t-1})+(f_{i}(1-\rho)+\epsilon_{it})
\end{eqnarray*}
\end{itemize}
\end{frame}

\begin{frame}\frametitle{Painéis Dinâmicos}

\begin{itemize}
\item Podemos reescrever esta equação como:
\[
y_{it}=\pi_{1}l_{it}+\pi_{2}l_{it-1}+\pi_{3}k_{it}+\pi_{4}k_{it-1}+\pi_{5}y_{it-1}+\gamma_{t}^{*}+(f_{i}^{*}+\epsilon_{it})
\]
\item Sujeita a duas restrições:

\begin{itemize}
\item $\pi_{2}=-\pi_{1}\pi_{5}$
\item $\pi_{4}=-\pi_{3}\pi_{5}$
\end{itemize}
\item Arellano e Bond (1991) supõem as seguintes premissas sobre as condições
iniciais:

\begin{itemize}
\item $E(\mathbf{x_{i1}}\epsilon_{it})=0$, sendo que $\mathbf{x_{it}}=(y_{it},l_{it},k_{it})$
\end{itemize}
\item Podemos utilizar as seguintes condições de momento:
\[
m(\theta)=E(\mathbf{x_{it-s}}\Delta\epsilon_{it})=0
\]
\item Em que $s\geq2$ caso não tenhamos erros de medida.
\end{itemize}
\end{frame}

\begin{frame}\frametitle{Painéis Dinâmicos (II):}

\begin{itemize}
\item O problema é que a estimação tem propriedades ruins quando os níveis
defasados da série, os $\mathbf{x_{it-s}}$ são pouco correlacionados
com as primeiras diferenças subseqüentes $\Delta\epsilon_{it}$.
\item Causas possíveis para isso:

\begin{itemize}
\item Processo marginal de determinação de $l_{it}$ e $k_{it}$ são muito
persistentes, próximos a ter uma raiz unitária.
\end{itemize}
\item Neste casos, os $\mathbf{x_{it-s}}$ são instrumentos fracos
\end{itemize}
\end{frame}

\begin{frame}\frametitle{Problemas de GMM-Diff}

\begin{itemize}
\item Esta abordagem também tem suas limitações. Em algumas aplicações,
é comum encontrar estimativas muito baixas de $\beta_{l}$ e $\beta_{k}$
e grandes erros-padrão.
\item Geralmente, a validade das restrições sobre-identificadoras é rejeitada.
Além disso, a hipótese que o processo dos $\eta$ seja exatamente
AR(1) pode ser rejeitada, o que implica que os $\mathbf{x_{it-2}}$
não seriam instrumentos válidos.
\item Além disso, a transformação em primeira diferença pode levar ao mesmo
problema no caso de erros de medida nas variáveis
\end{itemize}
\end{frame}

\begin{frame}\frametitle{GMM -- Sistema}

\begin{itemize}
\item Supondo adicionalmente que $E(\Delta l_{it}f_{i}^{*})=E(\Delta k_{it}f_{i}^{*})=0$,
e que as condições iniciais incluam $E(\Delta y_{i2}f_{i}^{*})=0$,
podemos incluir as seguintes condições de momento na estimação:
\[
m^{2}(\theta)=E(\Delta\mathbf{x_{it-s}}(f_{i}^{*}+\epsilon_{it}))=0
\]
\item Com $s=1$ caso não haja erros de medida.
\item Este é o chamado estimador GMM em Sistema de Blundell e Bond (1998).
\item Podemos testar a adequação das restrições adicionais por meio de um
teste de diferença de Sargan:

\begin{itemize}
\item Calcular a diferença entre os valores da função objetivo e comparar
com o valor crítico de uma distribuição $\chi^{2}$, com número de
graus de liberdade igual à diferença de condições de ortogonalidade
nos dois casos.
\end{itemize}
\end{itemize}
\end{frame}

\begin{comment}
\begin{frame}[allowframebreaks]
\bibliographystyle{aea}
\bibliography{C:/Bibliog/library}

\end{frame}

\end{comment}




\end{document}


