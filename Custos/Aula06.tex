\documentclass{beamer}
\usepackage{beamerthemesplit}
\usepackage[brazil]{babel}
\usepackage{epsfig}
\usepackage[utf8x]{inputenc}
\usepackage{pgf}
\usepackage{tikz}
%\usetikzlibrary{snakes}
\usepackage{nicefrac}
\usepackage{amsfonts}
\usepackage{amsmath}
\usepackage{amssymb}
\usepackage{amsthm}
%\usepackage{float}
\usetheme{Frankfurt}
\usepackage{epstopdf}
\usepackage{comment}
\usepackage{natbib}
\usepackage{float}
\usepackage{graphicx}
\usepackage{booktabs}
\usepackage{array}
\usepackage{bookmark}

\title{Aula 06}

\subtitle{Funcoes de Producao e Custos}

\author{Claudio R. Lucinda}


\institute{FEA/USP}

\date{}
\logo{\includegraphics[scale=.1]{logousp.png}}
\beamertemplatenavigationsymbolsempty
\begin{document}

\frame{\titlepage}
\begin{frame}\frametitle{Agenda}
  \tableofcontents[pausesections]
\end{frame}

\section{Fronteiras de Producao Estocasticas}
\begin{frame}{Fronteiras de Produção Estocásticas}

\begin{itemize}
\item Em termos de uma função de produção definida em termos de uma cross-section,
uma fronteira paramétrica pode ser definida como:
\[
Y_{i}=f(X_{i},\beta)\times TE_{i}
\]
\item Em que $i$ representa os produtores, $Y_{i}$ a quantidade produzida,
$X_{i}$ a quantidade dos insumos, $\beta$ os coeficientes associados
e $TE_{i}$ a medida de eficiência técnica orientada a produto, e
definida como:
\[
TE_{i}=\frac{Y_{i}}{f(X_{i},\beta)}
\]
\end{itemize}
\end{frame}

\begin{frame}{Fronteiras Estocásticas de Produção (II)}

\begin{itemize}
\item Na abordagem de Farrell (1957), ele adotou a forma $TE_{i}=\exp(-u_{i})$.
Note que, neste momento, não fizemos nenhuma imposição de elemento
aleatório, fazendo com que esta fronteira de produção seja determinística. 
\item Com a imposição de uma forma funcional à la Cobb-Douglas, a forma
funcional fica sendo:
\[
\ln Y_{i}=\beta_{0}+\sum_{r}\beta_{r}\ln X_{ri}-u_{i}
\]
\item Aigner, Lovell e Schmidt (1977) fizeram a análise econométrica canônica
deste tipo de modelo, incorporando o termo $u_{i}$ em um termo erro
composto, $e_{i}=v_{i}-u_{i}$, em que $v_{i}$é uma parte aleatória
independente e identicamente distribuída. Neste caso, o modelo completo
fica sendo:
\[
\ln Y_{i}=\beta_{0}+\sum_{r}\beta_{r}\ln X_{ri}+v_{i}-u_{i}
\]
\end{itemize}
\end{frame}

\begin{frame}{Fronteiras Estocásticas de Produção (III)}

\begin{itemize}
\item Supondo que o o termo de ineficiência seja com uma distribuição normal
pela metade, podemos escrever a função verossimilhança, com a suposição
que os dois termos erros são distribuidos independentemente. 
\item A partir da estimativa, podemos decompor o valor esperado da ineficiência
do termo erro composto, da seguinte forma:
\[
E(u_{i}|e_{i})=\frac{\sigma\lambda}{(1+\lambda^{2})}\left[\frac{\phi(\frac{e_{i}\lambda}{\sigma})}{\Phi(-\frac{e_{i}\lambda}{\sigma})}-\frac{e_{i}\lambda}{\sigma}\right]
\]
\item Em que $\phi$ é a pdf normal, a $\Phi$ a cdf normal, $\lambda=\frac{\sigma_{u}}{\sigma_{v}}$,
e $\sigma=(\sigma_{u}^{2}+\sigma_{v}^{2})^{1/2}$. 
\item Neste caso, $TE_{i}=1-E(u_{i}|e_{i})$. Outras alternativas foram
propostas, mas agora nós mudaremos o nosso foco para incorporarmos
dados em painel. 
\end{itemize}
\end{frame}

\subsection{Dados em Painel}
\begin{frame}{FEP -- Dados em Painel}

\begin{itemize}
\item Com a dispoibilidade de dados em painel, algumas vantagens surgem.

\begin{itemize}
\item Em primeiro lugar, não necessariamente o termo erro precisa ser exógeno
em relação às quantidades dos insumos, como no caso de cross-sections. 
\item Além disso, ao adicionar informações ao longo do tempo de uma mesma
empresa, isso permite que tenhamos estimativas consistentes da ineficiência. 
\item Finalmente, a terceira vantagem está no fato que, em alguns casos
não é necessária a imposição de hipóteses distribucionais sobre o
termo erro; simplesmente as técnicas de efeitos fixos e aleatórios
nos permitem obter estimativas dos parâmetros da função de produção. 
\end{itemize}
\end{itemize}
\end{frame}

\begin{frame}{FEP -- Dados em Painel (II)}

\begin{itemize}
\item Para entender este ponto melhor, considere a função de produção como
vimos anteriormente:
\[
\ln Y_{it}=\beta_{0}+\sum_{r}\beta_{r}\ln X_{rit}+v_{it}-u_{it}
\]
\item Sendo que $u_{it}>0$. Usando a transformação within, e se supusermos
que o termo $u_{it}$ seja sistemático e não varie ao longo do tempo,
podemos estimar os coeficientes da função de produção:
\[
(\ln Y_{it}-\bar{\ln Y_{it}})=\sum_{r}\beta_{r}(\ln X_{rit}-\bar{\ln X_{rit}})+v_{i}
\]
\item Podemos definir os termos de ineficiência individual por meio das
seguintes definições:
\begin{eqnarray*}
\hat{\beta}_{0} & = & \max(\hat{\beta}_{0i})\\
u_{i} & = & \hat{\beta_{0}}-\hat{\beta_{0i}}
\end{eqnarray*}
\end{itemize}
\end{frame}

\begin{frame}{FEP -- Dados em Painel (III)}

\begin{itemize}
\item Em que os $\hat{\beta_{0i}}$ são obtidas como:
\[
\hat{\beta_{0i}}=(\ln Y_{it}-\bar{\ln Y_{it}})-\sum_{r}\beta_{r}(\ln X_{rit}-\bar{\ln X_{rit}})
\]
\item Evidentemente, este tipo de análise impede que utilizemos dados que
são constantes ao longo do período amostral. 
\item Uma solução para este problema é tratar a parte de ineficiência como
sendo um efeito aleatório, podemos estimar o modelo da seguinte forma:
\[
\ln Y_{it}=(\beta_{0}-E(u_{i}))+\sum_{r}\beta_{r}\ln X_{rit}+v_{it}+(u_{it}-E(u_{it}))
\]
\item Em que tanto $v_{it}$ e $u_{it}^{*}=u_{it}-E(u_{it})$ possuem média
zero. Este modelo pode ser estimado por Mínimos Quadrados Generalizados. 
\end{itemize}

\note[item]{Note-se que, para a estimação da eficiência técnica, não precisamos
impor nenhuma hipótese distribucional sobre o termo erro. }

\note[item]{No entanto, a estimação por efeitos aleatórios precisa, para a estimação
consistente da eficiência técnica, que a dimensão tempo e cross-section
da amostra tenda ao infinito. }

\note[item]{Além disso, esta estimação poe efeitos aleatórios pressupõe que a
ineficiência seja exógena à escolha das quantidades de insumos. }

\note[item]{Ainda sobre este problema da consistência, a imposição de constância
na eficiência técnica ao mesmo tempo que exija que a amostra tenda
ao infinito é um pouco inconsistente, demandando o desenvolvimento
de modelos em que a eficiência seja alterada ao longo do tempo.}
\end{frame}

\begin{frame}{Dados em Painel}

\begin{itemize}
\item No contexto da estimação de ineficiênca técnica, algumas vantagens
surgem da utilização de dados em painel.

\begin{itemize}
\item Em primeiro lugar, não necessariamente o termo erro precisa ser exógeno
em relação às quantidades dos insumos, como no caso de cross-sections. 
\item Além disso, ao adicionar informações ao longo do tempo de uma mesma
empresa, isso permite que tenhamos estimativas consistentes da ineficiência. 
\item Finalmente, em alguns casos não é necessária a imposição de hipóteses
distribucionais sobre o termo erro; simplesmente as técnicas de efeitos
fixos e aleatórios nos permitem obter estimativas dos parâmetros da
função de produção.
\end{itemize}
\item Note que as questões mencionadas anteriormente sobre endogeneidade
continuam valendo; apenas assume-se que ela não existam.
\end{itemize}
\end{frame}

\begin{frame}{Dados em Painel}

\begin{itemize}
\item Para entender este ponto melhor, considere a função de produção como
vimos anteriormente:
\[
\ln Y_{it}=\beta_{0}+\sum_{r}\beta_{r}\ln X_{rit}+v_{it}-u_{it}
\]
\item Sendo que $u_{it}>0$. 
\item Usando a transformação within, e se supusermos que o termo $u_{it}$
seja sistemático e constante ao longo do tempo, podemos estimar os
coeficientes da função de produção:
\[
(\ln Y_{it}-\bar{\ln Y_{it}})=\sum_{r}\beta_{r}(\ln X_{rit}-\bar{\ln X_{rit}})+v_{i}
\]
\item Podemos definir os termos de ineficiência individual por meio das
seguintes definições:
\begin{eqnarray*}
\hat{\beta}_{0} & = & \max(\hat{\beta}_{0i})\\
u_{i} & = & \hat{\beta_{0}}-\hat{\beta_{0i}}
\end{eqnarray*}
\end{itemize}
\end{frame}

\begin{frame}{Dados em Painel (III)}

\begin{itemize}
\item Em que os $\hat{\beta_{0i}}$ são obtidos como:
\[
\hat{\beta_{0i}}=(\ln Y_{it}-\bar{\ln Y_{it}})-\sum_{r}\beta_{r}(\ln X_{rit}-\bar{\ln X_{rit}})
\]
\item Evidentemente, este tipo de análise impede que utilizemos dados que
são constantes ao longo do período amostral. 
\item Uma solução para este problema é tratar a parte de ineficiência como
sendo um efeito aleatório, podemos estimar o modelo da seguinte forma:
\[
\ln Y_{it}=(\beta_{0}-E(u_{i}))+\sum_{r}\beta_{r}\ln X_{rit}+v_{it}+(u_{it}-E(u_{it}))
\]
\item Em que tanto $v_{it}$ e $u_{it}^{*}=u_{it}-E(u_{it})$ possuem média
zero. 
\end{itemize}
\end{frame}

\begin{frame}{Dados em Painel -- Problemas de Estimação}

\begin{itemize}
\item Este modelo pode ser estimado por Mínimos Quadrados Generalizados,
como o modelo de Efeitos Aleatórios.
\item Note-se que, para a estimação da eficiência técnica, não precisamos
impor nenhuma hipótese distribucional sobre o termo erro. 
\item No entanto, para a estimação consistente dos coeficientes, temos alguns
problemas: 

\begin{itemize}
\item Em primeiro lugar, a estimação por efeitos aleatórios precisa, para
a estimação consistente da eficiência técnica, que a dimensão tempo
e cross-section da amostra tenda ao infinito. 
\item Além disso, esta estimação poe efeitos aleatórios pressupõe que a
ineficiência seja exógena à escolha das quantidades de insumos. 
\item Finalmente, a imposição de constância na eficiência técnica ao mesmo
tempo que exija que a amostra tenda ao infinito é contraditório, demandando
o desenvolvimento de modelos em que a eficiência seja alterada ao
longo do tempo.
\end{itemize}
\end{itemize}
\end{frame}

\begin{frame}{Cornwell, Schmidt e Sickles}

\begin{itemize}
\item A primeira linha de ação adotada foi a de Cornwell, Schmidt e Sickles,
que remonta o modelo acima da seguinte forma:
\[
\ln Y_{it}=(\beta_{0t}-u_{it})+\sum_{r}\beta_{r}\ln X_{rit}+v_{it}
\]
\item Os autores modelam os interceptos como uma função quadrática do tempo,
em que o tempo está associado com características específicas dos
produtores, no caso $\Gamma$:
\[
u_{it}=\Gamma_{1i}+\Gamma_{2i}t+\Gamma_{3i}t^{2}
\]
\item Os autores inclusive propôem inclusive uma forma de estimativa por
Variáveis Instrumentais deste negócio.
\end{itemize}
\end{frame}

\begin{frame}{Kumbhakar}

\begin{itemize}
\item A outra grande linha de ação passa pelo método da Máxima Verossimilhança. 
\item Segundo Kumbhakar (1990), supondo que a ineficiência se altere ao
longo do tempo de acordo com a seguinte especificação:
\[
u_{it}=\delta_{t}u_{i}
\]
\item E se assumirmos uma distribuição normal truncada para o termo $u_{i}$,
podemos modelar $u_{it}$ da seguinte forma:
\[
u_{it}=[1+\exp(\gamma t+\rho t^{2})]^{-1}\times u_{i}
\]
\item Em que $\gamma$ e $\rho$ são parâmetros a serem estimados. 
\item Uma linha alternativa de ataque é a de Battese e Coelli (1992), em
que apenas um parâmetro deve ser estimado:
\[
u_{it}=[\exp(-\eta(t-T)]\times u_{i}
\]
\end{itemize}
\end{frame}

\section{Economias de Escala}
\begin{frame}{Economias de Escala}

\begin{itemize}
\item Após estimar os coeficientes das funções de produção, como estimar
as economias de escala?
\item No caso de empresas multiproduto, Panzar (1989, p. 8) dá uma medida
de \emph{economias de escala tecnológicas}, definidas da seguinte
forma:
\[
S_{t}=\frac{\sum_{i=1}^{I}\sum_{j=1}^{J}x_{i}\frac{\partial f_{j}}{\partial x_{i}}}{\sum_{j}y_{j}}
\]
\item Em que $\frac{\partial f_{j}}{\partial x_{i}}$ é a produtividade
marginal do insumo $i$ na produção do bem $j$, $x_{i}$ a quantidade
do fator $i$ de produção e $y_{j}$ a quantidade do produto $j$. 
\item Os retornos de escala tecnológicos seriam crescentes caso $S_{t}$
fosse maior do que 1, decrescentes caso fosse menor do que 1, e constantes
caso fossem iguais a 1.
\end{itemize}
\end{frame}

\section{Funcoes de Custo}
\begin{frame}{Estimação de Função Custo}

\begin{itemize}
\item Com relação à função custos, uma função muito utilizada é a Translog.
Ela possui a seguinte forma funcional:{\tiny{}
\[
\ln C=a_{0}+\sum_{i}\beta_{i}\ln q_{i}+\sum_{j}\gamma_{j}\ln r_{j}+\sum_{j}\sum_{k}\gamma_{jk}\ln r_{j}\ln r_{k}+\sum_{h}\sum_{i}\beta_{hi}\ln q_{h}\ln q_{i}+\sum_{i}\sum_{j}\phi_{ij}\ln q_{i}\ln r_{j}
\]
}{\tiny \par}
\item Em geral, se multiplica as variáveis por 0,5 quando temos os termos
quadráticos, para facilitar na hora de montar os sistemas de demanda
de fatores. 
\item Vamos mostrar adicionalmente como se constrói um sistema de demandas
por fatores de produção usando o lema de Sheppard:
\[
\frac{1}{C}\frac{\partial C}{\partial r_{i}}=\gamma_{i}\frac{1}{r_{i}}+\sum_{k}\gamma_{ik}\frac{\ln r_{k}}{r_{i}}+\sum_{l}\phi_{li}\frac{\ln q_{l}}{r_{i}}
\]
\end{itemize}
\end{frame}

\begin{frame}{Lema de Shepard}

\begin{itemize}
\item Pelo lema de Sheppard, $\frac{\partial C}{\partial r_{i}}=z_{i}$,
ou seja:
\begin{eqnarray*}
\frac{z_{i}}{C} & = & \frac{1}{r_{i}}\left[\gamma_{i}+\sum_{k}\gamma_{ik}\ln r_{k}+\sum_{l}\phi_{li}\ln q_{l}\right]\\
\frac{r_{i}z_{i}}{C} & = & \left[\gamma_{i}+\sum_{k}\gamma_{ik}\ln r_{k}+\sum_{l}\phi_{li}\ln q_{l}\right]
\end{eqnarray*}
\item O lado esquerdo da igualdade mostra a participação da remuneração
do fator de produção $i$ no total dos custos. 
\item Para $N$ fatores de produção, podemos colocar um sistema de $N-1$
equações, impondo as restrições relevantes, para aumentar a eficiência
das estimativas.
\end{itemize}
\end{frame}

\begin{frame}{Estimação de Função Custos}

\begin{itemize}
\item Podemos, por teoria de dualidade que vimos anteriormente, afirmar
que a qualquer especificação de uma função de produção corresponde
uma especificação da função custo, supondo comportamento minimizador
de custos por parte das empresas envolvidas. 
\item Nos anos 70, em geral se afirmava que, na suposição de mercados competitivos
para os fatores de produção, a questão da exogeneidade das variáveis
independentes parecia resolvida (um ponto de vista como este foi adotado
por Varian (1980, cap. 12, pp. 207-9). 
\item Evidentemente, as questões de identificação induzidas pelo atrito
da amostra não estão resolvidas
\item Desta forma, vamos nos concentrar mais nos procedimentos de análise
propriamente ditos do que nas condições de identificação.
\end{itemize}
\end{frame}

\subsection{Spady e Friedlaender}
\begin{frame}{Spady e Friedlaender}

\begin{itemize}
\item A principal questão que este e o próximo artigo irão enfrentar diz
respeito ao fato que, em muitos casos, os serviços oferecidos por
uma mesma empresa são diferenciados -- ou seja, eles dependem
não apenas do número de unidades produzidas, mas também das características
destes insumos. 
\item Por exemplo, um determinado número de minutos de serviço de telecomunucações
pode se referir a minutos em horário comercial, e o restante fora
do horário comercial. 
\item Ou ainda, em termos de Kbps trafegados em uma rede, alguns podem ser
de dados e outros de voz. 
\item Consequentemente, duas empresas de telecomunicações podem possuir
duas estruturas muito diferentes de produtos e de custos se uma se
concentra em um conjunto de serviços e outra em outro. 
\end{itemize}

\note[item]{Em geral, estas diferenças são levadas em conta expandindo o vetor
de produtos para englobar a dimensão de qualidade (ou seja, Minutos
comerciais seriam um produto e Minutos Residenciais seriam outro). }

\note[item]{O problema é quando você tem uma dimensão de características que
os produtos podem tomar que, na prática, é contínua. Neste caso, exigiríamos
demais dos dados. }

\end{frame}

\begin{frame}{Funções Hedônicas de Custo}
\small
\begin{itemize}
\item Para resolver este problema, os autores propõem tratar o produto efetivo
como sendo uma função de uma medida genérica da quantidade e das suas
qualidades. 
\item Assim, estimaríamos as chamadas ``funções hedônicas de custo''. 
\item Os autores utilizam uma função de custos separável em qualidade, da
seguinte forma:
\[
C=C[\psi(y,q),w]
\]
\item Em que $\psi(y,q)$ representa um vetor de funções que mensuram os
produtos efetivos e $w$ denota um vetor de preços de fatores de produção. 
\item Desta forma, $\psi=[\psi^{1},\psi^{2},\cdots,\psi^{n}]$ , para uma
empresa que oferece $n$ produtos físicos diferentes, é um vetor em
que cada elemento $\psi^{i}=\psi^{i}(y_{i},q_{1}^{i},\cdots,q_{r}^{i})$.
\item $y_{i}$ é a quantidade física do produto $i$ 
\item $q_{r}^{i}$ é a r-ésima qualidade do produto $i$. 
\end{itemize}
\end{frame}

\begin{frame}{Spady e Friedlaender}
\tiny
\begin{itemize}
\item Em especial, eles assumem que:
\[
\psi^{i}=y_{i}\times\phi(q_{1}^{i},\cdots,q_{r}^{i})
\]
\item Dados estes diferentes serviços, eles estimam uma função custo translog
da seguinte forma:
\begin{eqnarray*}
\ln C(\psi,w) & = & \alpha_{0}+\sum_{i}\alpha_{i}(\ln\psi_{i}-\ln\bar{\psi_{i}})+\sum_{S}\beta_{s}(\ln w_{s}-\ln\bar{w_{s}})+\\
 & + & \frac{1}{2}[\sum_{i}\sum_{j}A_{ij}(\ln\psi_{i}-\ln\bar{\psi_{i}})(\ln\psi_{j}-\ln\bar{\psi_{j}})+\\
 & + & \sum_{s}\sum_{t}B_{st}(\ln w_{s}-\ln\bar{w_{s}})(\ln w_{t}-\ln\bar{w_{t}})]+\\
 & + & \sum_{i}\sum_{s}C_{is}(\ln\psi_{i}-\ln\bar{\psi_{i}})(\ln w_{s}-\ln\bar{w_{s}})+\varepsilon
\end{eqnarray*}
\end{itemize}

\note[item]{Em que as barras representam as médias das variáveis. }

\end{frame}

\begin{frame}{Spady e Friedlaender}

\begin{itemize}
\item Além disso os autores colocam no sistema as equações que representam
as participações dos fatores no custo total:
\[
\frac{w_{s}x_{s}}{C}=\beta_{s}+\sum_{t}B_{st}(\ln w_{t}-\ln\bar{w_{t}})+\sum_{i}C_{is}(\ln\psi_{i}-\ln\bar{\psi_{i}})
\]
Até agora, o que há de novo é simplesmente a expressão das variáveis
independentes em termos de desvios em relação às suas médias. 
\item Neste ponto, os autores precisamo expor a sua escolha referente à
forma funcional de $\psi_{i}$. 
\item Eles utilizam uma aproximação translog para esta função, da seguinte
forma:
{\tiny{}
\begin{eqnarray*}
\ln\psi_{i} & = & \ln y_{i}+\sum_{h}a_{h}^{i}(\ln q_{h}^{i}-\ln\bar{q_{h}^{i}})+\\
 & + & \frac{1}{2}\sum_{h}\sum_{l}b_{hl}^{i}(\ln q_{h}^{i}-\ln\bar{q}_{h}^{i})(\ln q_{l}^{i}-\ln\bar{q}_{l}^{i})
\end{eqnarray*}}
\end{itemize}

\note[item]{Esta função é substituída no sistema original e ele é estimado por
Mínimos Quadrados Generalizados (no paper de Spady e Friedlaender,
usaram FIML). }

\end{frame}

\begin{frame}{Spady e Friedlaender}

\begin{itemize}
\item Além disso, eles impõem as condições de simetria que estão a seguir:
\begin{eqnarray*}
\sum\beta_{s} & = & 1\\
\sum_{s}B_{st} & = & 0,t=1,\cdots,m;\\
\sum_{s}C_{is} & = & 0,i=1,\cdots,n\\
B_{ts} & = & B_{st}\\
A_{ij} & = & A_{ji}
\end{eqnarray*}
\item Os custos marginais de cada serviço podem ser estimados com a seguinte
fórmula:{\scriptsize{}
\[
CMg_{i}=\frac{C}{y_{i}}\frac{\partial\ln C}{\partial\ln y_{i}}=\frac{C}{y_{i}}\left\{ \alpha_{i}+\sum_{i=1}^{M}A_{im}(\ln y_{i}-\ln\bar{y_{i}})+\sum_{j=1}^{M}C_{ij}(\ln w_{i}-\ln\bar{w}_{i})\right\} 
\]
}{\scriptsize \par}
\end{itemize}
\end{frame}

\begin{frame}{Spady e Friedlaender}

\begin{itemize}
\item Adicionalmente, podemos calcular o grau de economias de escala desfrutadas
pela empresa usando a seguinte fórmula:
\[
S=\frac{C(y,w)}{\sum_{i}y_{i}\frac{\partial C}{\partial y_{i}}}
\]
\item Se $S$ for maior do que 1, os retornos de escala são crescentes.
Se $S$ é menor do que 1, os retornos de escala são decrescentes e
se $S$ for igual a 1, os retornos de escala são constantes. 
\end{itemize}
\end{frame}

\subsection{Evans e Heckman}
\begin{frame}{Evans e Heckman}

\begin{itemize}
\item Estes autores utilizam um método para testar a sub-aditividade das
funções custos no contexto de uma função de produção multiproduto. 
\item Eles começam assumindo que a função custo da AT\&T é dada por
\[
C=f(L,T,r,m,w,t)
\]
\item Em que: 

\begin{itemize}
\item $L$ denotava a quantidade de chamadas locais produzidas, 
\item $T$ a quantidade de chamadas a longa distância produzidas, 
\item $w$ o salário médio,
\item $r$ o custo do capital, 
\item $m$ o preço das matérias-primas 
\item $t$ um índice de mudança tecnológica. 
\end{itemize}
\end{itemize}
\end{frame}

\begin{frame}{Evans e Heckman}
\tiny
\begin{itemize}
\item A idéia desta separação entre produtos é baseada que, à época, estes
eram os principais produtos oferecidos pela empresa. 
\item Inicialmente eles começam estimando uma função translog de custos
da seguinte forma:
\begin{eqnarray*}
C & = & \alpha_{0}+\sum_{i=1}^{N}\alpha_{i}\ln p_{i}+\sum_{k=1}^{M}\beta_{k}\ln q_{k}+\mu\ln t\\
 & + & \left[\sum_{i=1}^{N}\sum_{j=1}^{N}\gamma_{ij}\ln p_{i}\ln p_{j}+\sum_{k=1}^{M}\sum_{l=1}^{M}\delta_{kl}\ln q_{k}\ln q_{l}\right]+\\
 & + & \sum_{i=1}^{N}\sum_{k=1}^{M}\rho_{ik}\ln p_{i}\ln q_{k}+\sum_{i=1}^{M}\lambda_{i}\ln p_{i}\ln t+\sum_{k=1}^{N}\theta_{k}\ln q_{k}\ln t+\tau(\ln t)^{2}
\end{eqnarray*}
\item Além disso, eles colocam um sistema associado de participações dos
insumos nos custos totais. Além disso, eles investigam dois pontos
adicionais. 
\begin{itemize}
\item A função de custos é separável em termos dos produtos 
\item A função de custos é aditiva. 
\end{itemize}
\end{itemize}
\end{frame}

\begin{frame}{Evans e Heckman}
\small
\begin{itemize}
\item A primeira hipótese, separabilidade em termos dos produtos, se for
verdade, implica que a decisão de montante a ser produzido pode ser
separada da decisão da composção da quantidade. 
\item Se ela for válida, podemos agregar as quantidades de $L$ e $T$ em
um único índice, da seguinte forma:
\[
C=f(L,T,r,m,w,t)=f(A(L,T),r,m,w,t)=f(Q^{*},r,m,w,t)
\]
\item Sendo que $Q^{*}=A(L,T)$. 
\item Para que isto valha, devemos ter $\rho_{ik}\beta_{l}=\rho_{il}\beta_{k},\forall i,k\neq l$. 
\item A segunda hipótese testável, a de aditividade, é que os custos de
produzir todos os produtos em um mesmo lugar é igual à soma dos custos
de produzir em unidades separadas, ou seja:
\[
C=f(L,T,r,m,w,t)=C_{L}(L,r,m,w,t)+C_{T}(T,r,m,w,t)
\]
\item Esta aditividade pode ser testada pela seguinte hipótese, na função
translog: $\delta_{kl}=-\beta_{k}\beta_{l}$. 
\end{itemize}

\note[item]{Se esta restrição for válida, a empresa não possui nem economias
nem deseconomias de escopo. }

\end{frame}

\begin{frame}{Evans e Heckman -- Economias de Escopo:}

\begin{itemize}
\item Finalmente, eles buscam testar a existência de economias de escopo. 
\item O teste canônico de economias de escopo foi proposto por Panzar (1989)
e é da seguinte forma. 
\item Seja $\hat{T}$ um subconjunto do espaço de produtos $(\hat{T},T)$,
no ponto em que a quantidade produzida é $q$. 
\item O grau de economias de escopo, neste caso, é dado por:
\[
SC=\frac{C(q_{T})+C(q_{\hat{T}})-C(q)}{C(q)}
\]
\item Em que $y_{T}$ é a quantidade produzida do subconjunto de produtos
$T$, $q_{\hat{T}}$ a quantidade produzida do subconjunto de produtos
$\hat{T}$ . 
\item A função custo apresentaria economias de escopo se $SC$ for maior
do que zero. 
\end{itemize}
\end{frame}

\begin{frame}{Economias de Escopo II}

\begin{itemize}
\item No entanto, esta metodologia ``canônica'' apresenta problemas quando
aplicada para a função translog.
\item Com os coeficientes em questão, a gente somente poderia calcular os
valores de $C(q_{T})$ e $C(q_{\hat{T}})$ se supuséssemos os valores
de algumas quantidades iguais a zero. 
\item No entanto, como a simples inspeção da fórmula acima pode nos mostrar,
temos que este tipo de função não é definida nos pontos em que os
argumentos são iguais a zero. 
\item Evans e Heckman seguem um caminho diferente:

\begin{itemize}
\item Eles testam uma hipótese mais simples, se é mais econômica a produção
dos dois bens em conjunto dentro de uma empresa -- no caso,
a AT\&T -- ou em duas empresas separadas. 
\end{itemize}
\end{itemize}
\end{frame}

\begin{frame}{Economias de Escopo III}

\begin{itemize}
\item O estudo que eles fazem é o seguinte. Considere $Q_{t}^{*}=(Q_{1t}^{*},Q_{2t}^{*})$
como sendo o vetor de bens produzidos em um determinado ano $t$. 
\item Além disso, considere também o vetor $Q_{M}=(Q_{1M},Q_{2M})=(\min Q_{1t}^{*},\min Q_{2t}^{*})$,
que são os menores valores dos dois produtos disponíveis na amostra. 
\item Vamos considerar duas empresas hipotéticas, A e B, que possuem as
seguintes quantidades produzidas:
\begin{eqnarray*}
Q_{At} & = & (\phi Q_{1t}+Q_{1M},\omega Q_{2t}+Q_{2M})\\
Q_{Bt} & = & ((1-\phi)Q_{1t}+Q_{1M},(1-\omega)Q_{2t}+Q_{2M})
\end{eqnarray*}
\item Neste caso, temos que ,$\omega$ e $\phi$ pertencem am intervalo
$[0,1]$, pois a quantidade observada seria igual à produção das duas
empresas hipotéticas. 
\end{itemize}
\end{frame}

\begin{frame}{Economias de Escopo IV}
\small
\begin{itemize}
\item Restringindo o domínio na função aos valores observados, temos que
as duas empresas, juntas, produzem:
\begin{eqnarray*}
Q_{1t}+2Q_{1M} & = & Q_{1t}^{*}\\
Q_{2t}+2Q_{2M} & = & Q_{2t}^{*}
\end{eqnarray*}
\item Finalmente, considere as seguintes definições:
\begin{eqnarray*}
C_{At}(\phi,\omega) & = & C(Q_{At})\\
C_{Bt}(\phi,\omega) & = & C(Q_{Bt})\\
C_{t}^{*} & = & C(Q_{t}^{*})
\end{eqnarray*}
\item A sub-aditividade vai ser obtida calculando a seguinte estatística:
\[
Sub(\phi,\omega)=\frac{C_{t}^{*}-C_{At}(\phi,\omega)-C_{Bt}(\phi,\omega)}{C_{t}^{*}}
\]
\item Podemos notar que esta estatística ainda é uma função de $\phi$ e
$\omega$. 
\end{itemize}

\note[item]{Neste caso, o que os autores fazem é estimar o valor desta estatística
para todos os parâmetros possíveis $\phi$ e $\omega$, e se o máximo
deste valor for negativo (afinal de contas, este negócio é igual a
(-1) multiplicando a estatística $SC$ anterior) e significativo,
podemos dizer que existem economias de escopo e sub-aditividade da
função de custos. }
\end{frame}


%\begin{comment}
%\begin{frame}[allowframebreaks]
%\bibliographystyle{aea}
%\bibliography{C:/Bibliog/library}

%\end{frame}

%\end{comment}




\end{document}


