\documentclass{beamer}
\usepackage{beamerthemesplit}
\usepackage[brazil]{babel}
\usepackage{epsfig}
\usepackage[utf8x]{inputenc}
\usepackage{pgf}
%\usepackage{tikz}
%\usetikzlibrary{snakes}
\usepackage{nicefrac}
\usepackage{amsfonts}
\usepackage{amsmath}
\usepackage{amssymb}
\usepackage{amsthm}
%\usepackage{float}
\usetheme{Frankfurt}
\usepackage{epstopdf}
\usepackage{comment}
\usepackage{natbib}
\usepackage{float}
\usepackage{graphicx}
\usepackage{booktabs}
\usepackage{array}
%\usepackage{bookmark}
%\usepackage[normalem]{ulem}

\title{Aula 09}

\subtitle{Entrada -- Modelos de Informação Incompleta}

\author{Claudio R. Lucinda}


\institute{FEA-RP/USP}

\date{}
\logo{\includegraphics[scale=.1]{logousp.png}}
\beamertemplatenavigationsymbolsempty
\begin{document}

\frame{\titlepage}
\begin{frame}\frametitle{Agenda}
  \tableofcontents[pausesections]
\end{frame}

\section{Introducao}

\begin{frame}[fragile]\frametitle{Introdução}
    \begin{itemize}
    	\item Vamos agora falar dos modelos estáticos de informação incompleta.
    	\item Neste caso, as funções payoff são de conhecimento privado dos players -- porque contém informações específicas aos players.
    	\item Os jogadores fazem então crenças sobre os aspectos não observados do \textit{payoff} das empresas.
    	\item Esta informação incompleta pode ser modelado da forma mais simples por conta do $\varepsilon$, a parte dos lucros que também não é observada pelo pesquisador. 
    	\item Mais especificamente, vamos supor que cada jogador observa o seu $\varepsilon_{i}$, mas só sabe a distribuição dos valores dos seus competidores, ou seja $F(\varepsilon_{j\neq i})$
    \end{itemize}


\end{frame}

\begin{frame}[fragile]\frametitle{Introdução (II):}
\begin{itemize}
	\item O econometrista, neste caso, só conhece a distribuição $F(\varepsilon_{j})$, que é quase a mesma coisa que o que as empresas sabem.
	\item Cada empresa então forma crenças sobre o comportamento das oponentes, o que leva ao seguinte sistema de desigualdades (para $K=2$):
\begin{align*}
	a_{1}&=\mathbf{1}\left[X\beta +\delta p_{2} + \varepsilon_{1} \geq 0 \right]\\
	a_{2}&=\mathbf{1}\left[X\beta +\delta p_{1} + \varepsilon_{2} \geq 0 \right]
\end{align*}
	\item Sendo que $p_{i}=E_{i}(y_{-i})$ representam as crenças da empresa $i$ sobre as ações dos seus oponentes.
	\item Isso gera um negócio chamado \textit{Conditional Choice Probability} que vai ser usada para representar a estratégia de cada empresa.
\end{itemize}
    
\end{frame}

\begin{frame}[fragile]\frametitle{Equilíbrio de Nash Bayesiano}
    \begin{itemize}
    	\item Podemos usar essas CCP para caracterizar melhor o Equilíbrio de Nash Bayesiano:
		\begin{align*}
		p_{1} &= \Psi_{1}(X\beta + \delta p_{2}) \\
		p_{2} &= \Psi_{2}(X\beta + \delta p_{1})
		\end{align*}
		\item Sendo que a forma exata da função $\Psi(\cdot)$ vai depender da distribuição $F$. 
		\item Essas $\Psi$ são as funções melhor resposta.
		\item Supondo que elas sejam contínuas, a gente pode garantir que este sistema de equações não lineares tem uma solução pelo Teorema do Ponto Fixo de Brouwer.
		\item Além disso, podemos ter um jeito direto de resolver para os equilíbrios para um valor candidato dos parâmetros.
		\begin{itemize}
			\item O método das aproximações sucessivas (aka iteração de ponto fixo)
		\end{itemize}
    \end{itemize}
\end{frame}

\section{Fixed Point Iteration}

\begin{frame}[fragile]\frametitle{Iteração de Ponto Fixo}
\begin{itemize}
	\item Lógica:
	\begin{itemize}
		\item Para cada valor candidato mandado pelo otimizador, você vai resolver o sistema de equações para cada mercado.
		\item Com as probabilidades das empresas entrarem, você diz que a contribuição à verossimilhança daquele mercado é a probabilidade correspondente.
		\item Quando a gente tem os $\varepsilon$ vindo de uma distribuição de valores extremos, a função $\Psi$ tem inclusive a nossa conhecida cara Logit.
	\end{itemize}
\end{itemize}
   
\end{frame}

\begin{frame}[fragile]\frametitle{Iteração de Ponto Fixo -- Problemas Computacionais}
    \begin{itemize}
    	\item O grande problema aqui é a chamada ``praga da dimensionalidade'':
    	\item Em termos técnicos, o número mínimo de avaliações das funções relevantes e operações aritméticas necessárias para se computar uma $\epsilon$-aproximação a um ponto fixo de um sistema de $d$ cresce exponencialmente com $d$.
    	\item Em jogos estáticos com poucas empresas isso não pega tanto, mas em jogos dinâmicos e/ou com muitas empresas pode pegar.
    \end{itemize}


\end{frame}

\section{Hotz-Miller Inversion}

\begin{frame}[fragile]\frametitle{Hotz--Miller Inversion}
\begin{itemize}
	\item A grande contribuição deste artigo -- originalmente desenvolvido para jogos dinâmicos -- é que na verdade as coisas do sistema de equações do Equilíbrio de Nash Bayesiano valem para qualquer estimativa \textbf{consistente} de $p_{1}$ e $p_{2}$ do lado direito da igualdade.
	\item E você pode estimar estas probabilidades dos dados!
	\item Ou seja, você vai usar o maximizador pra achar os $\hat{\beta}$ condicional a termos 
		\begin{align*}
		p_{1} &= \Psi_{1}(X\beta + \delta \hat{p_{2}}) \\
		p_{2} &= \Psi_{2}(X\beta + \delta \hat{p_{1}})
		\end{align*}
	\item Ou seja, você calcularia uma estimativa consistente de $\hat{p_{1}}$ e $\hat{p_{2}}$ (pode ser a frequencia relativa) e manda otimizar.

\end{itemize}
\end{frame}

\begin{frame}[fragile]\frametitle{Hotz-Miller Inversion -- Vantagens e Desvantagens}
\begin{itemize}
	\item Vantagens:
	\begin{itemize}
		\item Ele é relativamente robusto à multiplicidade. Caso o mesmo equilíbrio seja sempre jogado nos dados, isso resolve o problema da coerência, pois o PBE condiciona ao equilíbrio que sempre é efetivamente jogado.
		\item Óbvio, isso pressupõe que sempre se joga o mesmo equilíbrio.
	\end{itemize}
	\item Desvantagens:
	\begin{itemize}
		\item É menos eficiente (limited information method)
		\item As estimativas das CCP tem que ser estimadas de forma consistente.
		\item Você tem que estimar isso não parametricamente. Só que isso vai implicar que as estimativas das CCP vão ser muito ruidosas, o que vai carregar pro segundo estágio.
	\end{itemize}
\end{itemize}
\end{frame}

\section{Aguirregabiria e Mira (2002, 2007)}

\begin{frame}[fragile]\frametitle{Aguirregabiria e Mira (2002, 2007)}
\begin{itemize}
	\item Aguirregabiria e Mira (2002, 2007) propõem interagir o sistema de equações do BPNE em cada passada do otimizador.
	\item Na verdade você inverteu a ordem dos passos do primeiro método.
	\item Forçando que as condições do BPNE sejam satisfeitas, você torna isso um método de informação completa.
	\item Desde que convirja.
	\item O que pega é que ele depende da iteração de melhores respostas, e pode não achar equilíbrios de Nash que não são estáveis em melhor resposta.
	\item E isso significa que o algoritmo pode não convergir (Pesendorfer e Schmidt-Dengler 2010)
\end{itemize}
\end{frame}

\section{Heterogeneidade não Observável}
\begin{frame}[fragile]\frametitle{\insertsection}
\begin{itemize}
	\item Todos esses métodos podem acomodar a chamada heterogeneidade não observada pelo econometrista (mas observada pelas empresas).
	\item Basicamente, isso seria adaptar efeitos aleatórios ou random coefficients no arcabouço anterior.
	\item Aumenta a dificuldade computacional disso. Por exemplo, conseguir convergência com métodos como os de Aguirregabiria e Mira (2002 e 2007) é bem mais difícil.
	
\end{itemize}
    


\end{frame}

\begin{comment}
\begin{frame}[allowframebreaks]
\bibliographystyle{aea}
\bibliography{C:/Bibliog/library}

\end{frame}

\end{comment}




\end{document}


