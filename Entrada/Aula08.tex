\documentclass{beamer}
\usepackage{beamerthemesplit}
\usepackage[brazil]{babel}
\usepackage{epsfig}
\usepackage[utf8x]{inputenc}
\usepackage{pgf}
%\usepackage{tikz}
%\usetikzlibrary{snakes}
\usepackage{nicefrac}
\usepackage{amsfonts}
\usepackage{amsmath}
\usepackage{amssymb}
\usepackage{amsthm}
%\usepackage{float}
\usetheme{Frankfurt}
\usepackage{epstopdf}
\usepackage{comment}
\usepackage{natbib}
\usepackage{float}
\usepackage{graphicx}
\usepackage{booktabs}
\usepackage{array}
%\usepackage{bookmark}
%\usepackage[normalem]{ulem}

\title{Aula 08}

\subtitle{Entrada}

\author{Claudio R. Lucinda}


\institute{FEA-RP/USP}

\date{}
\logo{\includegraphics[scale=.1]{logousp.png}}
\beamertemplatenavigationsymbolsempty
\begin{document}

\frame{\titlepage}
\begin{frame}\frametitle{Agenda}
  \tableofcontents[pausesections]
\end{frame}
\section{Introducao}
\begin{frame}\frametitle{Introdução}
\small
\begin{itemize}
\item Nesta aula, iremos mudar o foco de nossa análise, para investigarmos
a questão da entrada e saída das empresas em um determinado mercado. 
\item Em especial, esta discussão se filia à questão de como entender a
estrutura de mercado e a relação entre esta e a competição dentre
deste mesmo mercado. 

\begin{itemize}
\item Como o número e organização das empresas neste mercado, o tamanho
das mesmas, dos competidores em potencial e a forma das linhas de
produtos das empresas afetam a competição e os lucros das mesmas. 
\end{itemize}
\item A discussão sobre este tema variou bastante ao longo do tempo. 
\item Nos anos 50 a 70, as análises econométricas eram voltadas para como
as variáveis tais como lucro das empresas, gastos com P\&D, e preços
variavam em mercados concentrados ou pouco concentrados. 
\item A maior parte destes trabalhos assumiam que a estrutura de mercado
era exógena à decisão sobre estas variáveis. 
\end{itemize}
\end{frame}

\begin{frame}\frametitle{Introdução (II):}
\small
\begin{itemize}
\item Nos anos 80, os modelos teóricos de OI eram focados em entender como
o comportamento estratégico poderia influenciar a estrutura de mercado. 
\item Tais modelos consideravam a estrutura de mercado como o resultado
de um jogo em duas etapas, sendo que a primeira envolvia a decisão
de entrada ou saída, enquanto que a segunda etapa modelava a competição
entre as empresas que decidiram pela entrada. 
\item Do ponto de vista empírico, a maior disponibilidade de dados dos censos
industriais permitiu documentar ricos padrões de entrada e saída das
empresas, bem como mudanças na estrutura do mercado. 
\item Nesta aula, iremos descrever como a análise empírica mais moderna
sobre este tema se baseia, utilizando modelos de teoria dos jogos
para construir modelos econométricos de entrada, saída e concentração
de mercado. 
\end{itemize}
\end{frame}

\begin{frame}\frametitle{Introdução (III):}
\small
\begin{itemize}
\item Em especial, iremos discutir as previsões que estes modelos fazem
sobre os efeitos sobre a estrutura de mercado das seguintes variáveis:

\begin{itemize}
\item Tamanho e irrecuperabilidade dos custos fixos não recuperáveis;
\item Sensibilidade dos lucros das empresas à entrada e saída de competidores;
\item O grau de substitutibilidade dos produtos
\item Expectativas dos potenciais entrantes sobre a competição após a entrada
propriamente dita;
\item A existência e a eficiência dos entrantes em potencial
\item A endogeneidade dos custos fixos e dos custos irrecuperáveis.
\end{itemize}
\item Vamos começar a nossa análise com uma estrutura bem simples de estudo,
na seção seguinte, originalmente apresentada no paper de \citet{Bresnahan1991} e resumido em \citet{Berry2007a}
\end{itemize}
\end{frame}

\section{Modelos de Escolhas Estratégicas Discretas}
\begin{frame}\frametitle{Modelos de Escolhas Estratégicas Discretas}
\small
\begin{itemize}
\item Vamos começar a ilustrar como é esta metodologia com um problema bem
simples. 
\item Vamos supor que existam dados sobre as decisões de entrada das firmas
-- supostas homogêneas -- e queremos estimar
os custos fixos de produção das mesmas. 
\item Além disso, vamos supor que se observa um grande número de mercados
regionais distintos e que existem dados sobre a demanda de mercado
e os custos dos insumos das empresas. 
\item Se estes mercados fossem competitivos -- o que implicaria
que as decisões de entrada dos diferentes agentes são independentes
-- poderíamos modelar esta entrada das empresas como um
modelo de escolha discreta. 
\item No entanto, quando estamos falando de mercados oligopolizados, as
expectativas das empresas acerca do comportamento dos seus competidores
também afetam as decisões de entrada, e vice-versa. 
\end{itemize}
\end{frame}

\begin{frame}\frametitle{Modelos de Escolhas Estratégicas Discretas (II)}
\small
\begin{itemize}
\item Para que possamos saber até que ponto os maiores custos fixos afetam
a concentração de mercado ou se o comportamento dos competidores afeta
os lucros, precisamos de modelos estatísticos para decisões interdependentes.
\item Para isto, vamos começar com um modelo bem simples, para investigar
como os custos fixos afetam a decisão de entrada de duas empresas.
\item Vamos definir $a_{i}=1$ como sendo o evento de entrada no mercado
da empresa $i$, e $a_{i}=0$ o evento de não entrada da empresa $i$. 
\item Podemos representar a decisão estratégica das empresas com a ajuda
da seguinte matriz de \emph{payoffs}:
\end{itemize}
\begin{center}
\begin{tabular}{|c|c|c|c|}
\hline 
 &  & \multicolumn{2}{c|}{Empresa 1}\tabularnewline
\hline 
\hline 
 &  & $a_{1}=0$ & $a_{1}=1$\tabularnewline
\hline 
Empresa 2 & $a_{2}=0$ & $\Pi_{00}^{1}$,$\Pi_{00}^{2}$ & $\Pi_{10}^{1}$,$\Pi_{10}^{2}$\tabularnewline
\hline 
 & $a_{2}=1$ & $\Pi_{01}^{1}$,$\Pi_{01}^{2}$ & $\Pi_{11}^{1}$,$\Pi_{11}^{2}$\tabularnewline
\hline 
\end{tabular}
\par\end{center}

\end{frame}

\begin{frame}\frametitle{Escolhas Discretas}
\small
\begin{itemize}
\item Supondo que os dados observados venham de um jogo sem repetição e
simultâneo, e que possamos restringir a nossa atenção aos equilíbrios
de Nash em estratégias puras, as estratégias de equilíbrio dos agentes
podem ser representadas pelas seguintes condições:
\begin{eqnarray*}
a_{1}^{*} & = & \left\{ \begin{array}{cc}
1 & se\,\pi_{1}^{*}\geq0\\
0 & se\,\pi_{1}^{*}<0
\end{array}\right.
\end{eqnarray*}
\[
a_{2}^{*}=\left\{ \begin{array}{cc}
1 & se\,\pi_{2}^{*}\geq0\\
0 & se\,\pi_{2}^{*}<0
\end{array}\right.
\]
\item Sendo que $\pi^{*}$ representa os lucros adicionais para a empresa
decorrentes da decisão de entrar. 
\item Em especial, $\pi_{1}^{*}=(1-a_{2})(\Pi_{10}^{1}-\Pi_{00}^{1})+a_{2}(\Pi_{11}^{1}-\Pi_{01}^{1})$
e $\pi_{2}^{*}=(1-a_{1})(\Pi_{01}^{2}-\Pi_{00}^{2})+a_{1}(\Pi_{11}^{2}-\Pi_{10}^{2})$. 
\end{itemize}
\end{frame}

\begin{frame}\frametitle{Estratégias Discretas:}
\scriptsize
\begin{itemize}
\item Uma vez que as empresas somente incorrem em custos fixos
se elas decidirem pela entrada, estas diferenças de lucros apresentam
os custos fixos como termos separados. 
\item Para a estimação econométrica, precisamos especificar
a distribuição da parte variável dos lucros (o ganho de lucros além
do custo fixo que as empresas auferem no caso da entrada propriamente
dita).
\item Seguindo \citet{Bresnahan1991}, podemos escrever
estes lucros incrementais da entrada da seguinte forma:
\begin{eqnarray*}
\pi_{1}^{*} & = & X_{1}\beta_{0}^{1}+a_{2}\Delta_{1}^{1}-\varepsilon^{1}\\
\pi_{2}^{*} & = & X_{2}\beta_{0}^{2}+a_{1}\Delta_{1}^{2}-\varepsilon^{2}
\end{eqnarray*}

\item Podemos entender estas equações da seguinte forma. Normalizando
os lucros na eventualidade da não entrada em zero, temos que $\pi_{1}^{*}=\Pi_{10}^{1}+a_{2}(\Pi_{11}^{1}-\Pi_{10}^{1})$. 
\item O primeiro dos termos representa os lucros que a empresa
1 desfrutaria caso fosse monopolista neste mercado, que podemos assumir
que é uma função de elementos observáveis, alguns parâmetros e componentes
aleatórios, dando $\Pi_{10}^{1}=X_{1}\beta_{0}^{1}-\varepsilon^{1}$.
\end{itemize}
\end{frame}

\begin{frame}\frametitle{Estratégias Discretas }

\begin{itemize}
\item Além disso, o termo entre parênteses representa a diferença nos lucros
da empresa 1 decorrente da entrada da empresa 2. 
\item Este elemento podemos deixar como sendo um parâmetro a ser estimado
e denotado por $\Delta_{1}^{1}$. 
\item Estas considerações fazem com que possamos especificar um modelo econométrico
e estimar os parâmetros relevantes. 
\item No entanto, ainda temos dois problemas, um de natureza teórica e outro
de natureza econométrica. O problema de natureza econométrica é que,
dado que consideramos $\Delta_{1}^{1}$ como um parâmetro a ser estimado,
se não pusermos nenhuma restrição sobre o suporte da distribuição
do termo $\varepsilon$, não conseguiremos identificar adequadamente
os parâmetros $\beta$. 
\end{itemize}
\end{frame}

\begin{frame}\frametitle{Estratégias Discretas}
\footnotesize
\begin{itemize}
\item O problema de natureza teórica é que nem sempre os valores são tais
que garantem a existência e a unicidade do equilíbrio (ou seja, pode
haver um equilíbrio múltiplo, em que apenas uma das empresas entra).
\item A solução para estes problemas passou pelo fato de se considerar os
resultados não únicos como observacionalmente equivalentes. Esta foi
a abordagem de \citet{Bresnahan1990}. 
\item Esta restrição implica que o modelo econométrico deixa de ser um modelo
especificamente voltado para a decisão de entrada, mas sim um modelo
para a previsão do número de entrantes em um determinado mercado. 
\item No caso de um duopólio, a função de verossimilhança contém as seguintes
assertivas sobre as probabilidades de ocorrência:
\begin{eqnarray*}
\Pr(a_{1}=0,a_{2}=0) & = & \Pr(X\beta_{0}^{1}<\varepsilon^{1},X\beta_{0}^{2}<\varepsilon^{2})\\
\Pr(a_{1}=1,a_{2}=1) & = & \Pr(X\beta_{0}^{1}+\Delta_{1}^{1}>\varepsilon^{1},X\beta_{0}^{1}+\Delta_{1}^{2}>\varepsilon^{2})\\
\Pr(a_{1}=1\vee a_{2}=1) & = & 1-\Pr(a_{1}=0,a_{2}=0)-\Pr(a_{1}=1,a_{2}=1)
\end{eqnarray*}
\end{itemize}
\end{frame}

\begin{frame}\frametitle{Estratégias Discretas}
\small
\begin{itemize}
\item A distribuição dos termos $\Delta_{1}^{i}$ determinaria, então, a
forma funcional específica para estas probabilidades. 
\item Evidentemente, esta distribuição deve respeitar as restrições sobre
os \emph{payoffs} dos jogadores e permitir a identificação dos parâmetros
chave. 
\item Uma forma utilizada por Bresnahan e Reiss (1990 e 1991) foi modelar
este negócio permitindo heterogeneidades não observáveis entre os
jogadores. 
\item Por exemplo, estes autores usam $\Delta_{1}^{i}=g(Z\gamma^{i})+\eta^{i}$,
em que $g(\cdot)$ é uma função definida no ramo negativo dos números
reais e $\eta^{i}$ é uma variável aleatória com um limite superior
de zero. 
\item Primeiro colocaremos um modelo mais geral, para depois abordar especificamente
a forma pela qual estes artigos chegaram aos seus resultados, de forma
a nos familiarizar com as suas premissas.
\end{itemize}
\end{frame}

\subsection{Um Modelo Geral de firmas homgêneas}
\begin{frame}\frametitle{Um Modelo Geral de firmas homogêneas}
\footnotesize
\begin{itemize}
\item Como vimos anteriormente, para evitarmos o problema da identificação
dos modelos econométricos, bem como a questão de como modelar os equilíbrios
múltiplos, a solução é considerar os resultados múltiplos como um
único elemento e, depois, modelar a questão sobre o número de empresas
que entram.
\item Iremos, em especial, discutir o que poderemos aprender sobre as primitivas
do modelo quando observamos um vetor de quantidades $N_{1},N_{2},\cdots,N_{T}$,
que entraram em $T$ mercados diferentes. 
\item Para isto, precisamos relacionar o $N_{T}$ observado com os lucros
não observados no mercado $T$. 
\item Dados $N_{i}$ entrantes no mercado $T$ , temos que os lucros de
cada uma delas é dado por:
\[
\pi_{i}(N_{i})=V(N_{i},x_{i},\theta)-F_{i}
\]
\end{itemize}
\end{frame}

\begin{frame}\frametitle{Modelo Geral (II)}

\begin{itemize}
\item Em que $V(\cdot)$ representa os lucros variáveis totais e $F$ o
custo fixo. Vamos assumir que os custos fixos, que também não são
observados pelo econometrista são distribuídos de acordo com $\Phi(F_{i}|x_{i},\omega)$.
Com esta função lucro, podemos ligar as decisões de entrada ao número
observado de empresas. Para as $N^{*}$ empresas que entraram, temos:
\[
V(N^{*},x,\theta)-F>0
\]
\item Enquanto que para qualquer entrante potencial, temos que:
\[
V(N^{*}+1,x,\theta)-F<0
\]
\item Combinando estas duas desigualdades, temos um limite para os custos
fixos:
\[
V(N^{*},x,\theta)\geq F>V(N^{*}+1,x,\theta)
\]
\end{itemize}
\end{frame}

\begin{frame}\frametitle{Modelo Geral (III):}

\begin{itemize}
\item Estes limites permitem que possamos estimar os parâmetros da função
lucro variável e os custos fixos a partir da observação do vetor $x$
e do número de empresas:
\begin{eqnarray*}
\Pr(V(N^{*},x) & \geq & F|x)-\Pr(V(N^{*}+1,x)>F|x)=\\
 & = & \Phi(V(N^{*},x,\theta)|x)-\Phi(V(N^{*}+1,x,\theta)|x)
\end{eqnarray*}
\item Supondo amostras independente eidenticamente distribuídas, temos que
a amostra possui uma função de verossimilhança ``ordenada'' da seguinte
forma:
\[
LL(\theta|\{x,N^{*}\})=\sum_{t}(\ln(V(N_{t}^{*},x_{t}))-\ln(V(N_{t}^{*}+1,x_{t})))
\]
\end{itemize}
\end{frame}

\begin{frame}\frametitle{Modelo Geral }

\begin{itemize}
\item A pergunta, para a qual veremos algumas respostas mais adiante, é
como especificar esta função de lucros variáveis. 
\item Uma delas seria montar a função $V(\cdot)$ de tal forma que ela torne
a estimação simples e, ao mesmo tempo, atenda a restrição que a função
seja não crescente em $N$. 
\item A segunda abordagem é baseada em especificar a função $V(\cdot)$diretamente
a partir de premissas de mercado e hipóteses sobre o jogo após a entrada. 
\item Vamos ver dois exemplos deste tipo de abordagem nos papers a seguir.
\end{itemize}
\end{frame}

\subsection{Bresnahan e Reiss (1990)}
\begin{frame}\frametitle{Bresnahan e Reiss (1990)}

\begin{itemize}
\item \citet{Bresnahan1990}, fazem a suposição de que os resultados
que equivalem a equilíbrios múltiplos são observacionalmente equivalentes. 
\item Neste caso, eles se focam no número de empresas -- que,
neste caso, podem ser zero, uma ou duas empresas. A função de lucro
de cada empresa é dada por:
\[
\Pi_{i}^{N}=V_{i}^{N}\times S(Y)-F_{i}^{N}
\]
\item Sendo que a empresa $i$ pode ser monopolista $N=M$, ou duopolista
$N=D$. 
\item A função $V_{i}$ representa os ``lucros variáveis por consumidor'',
ou seja, a margem entre preço e custo marginal, multiplicada pela
função demanda individual. 
\end{itemize}
\end{frame}

\begin{frame}\frametitle{Bresnahan e Reiss (1990) (II)}

\begin{itemize}
\item O termo $S(Y)$ seria uma medida do tamanho do mercado, enquanto que
$F_{i}^{N}$ é uma medida dos custos fixos. 
\item A modelagem é refinada ao assumirmos uma parte não observável para
os lucros variáveis por consumidor, bem como para os custos fixos,
o que deixa a função lucro da seguinte forma:
\begin{eqnarray*}
\Pi_{i}^{N} & = & [\bar{V}_{i}^{N}+\eta_{i}^{N}]\times S(Y)-\bar{F_{i}^{N}}+\varepsilon_{i}^{N}\\
 & = & \bar{V}_{i}^{N}\times S(Y)-\bar{F}_{i}^{N}+\xi_{i}^{N}\\
 & = & \bar{\Pi}_{i}^{N}+\xi_{i}^{N}
\end{eqnarray*}
\item Acredito que as similaridades entre a forma colocada anteriormente
e a que desenvolvemos agora estejam bastante claras. 
\end{itemize}
\end{frame}

\begin{frame}\frametitle{Bresnahan e Reiss (1990) (III)}
\scriptsize
\begin{itemize}
\item Caso não tenhamos elementos não observáveis sobre os lucros variáveis
por consumidor, o termo $\xi_{i}^{N}$ fica igual ao termo $\varepsilon_{i}^{N}$,
de forma que podemos estimar este negócio por um modelo PROBIT ordenado. 
\item A diferença entre o modelo PROBIT multinomial (ou seja, a extensão
do modelo PROBIT para o caso de mais de dois resultados) para o modelo
PROBIT multinomial reside no fato que a ordem dos valores que a variável
representativa dos lucros possui relevância para a análise. 
\item Caso tenhamos elementos não observáveis sobre os lucros variáveis,
chegamos ao modelo PROBIT ordenado, só que o termo $\xi_{i}^{N}$
é inerentemente heterocedástico, pois o termo $\varepsilon_{i}^{N}$
ali presente está multiplicado pela função $S(Y)$. Neste caso, as
funções de probabilidade ficariam sendo:
\begin{eqnarray*}
P_{0} & = & 1-\Phi(\bar{\Pi}^{M}/\sigma_{\xi})\\
P_{2} & = & \Phi(\bar{\Pi}^{D}/\sigma_{\xi})\\
P_{1} & = & 1-P_{0}-P_{2}
\end{eqnarray*}
\end{itemize}
\end{frame}

\begin{frame}\frametitle{Bresnahan e Reiss (1990)}
\scriptsize
\begin{itemize}
\item Em que uma restrição deveria ser imposta para que $\bar{\Pi}^{M}\geq\bar{\Pi}^{D}$.
Com relação às precisas formas funcionais utilizadas, elas foram:
\[
S(Y)=TOWNPOP+\lambda_{1}OPOP10+\lambda_{2}NGRW70+\lambda_{3}PGRW70+\lambda_{4}OCTY
\]
\item Em que $TOWNPOP$ representa a população da cidade, $OPOP10$ representa
o montante de demanda das pessoas em volta da cidade, $NGRW70$ e
$PGRW70$ são as taxas de crescimento negativas e positivas no tamanho
da população entre 1970 e 1980. $OCTY$ é a fração dos residentes
da cidade que comutam para fora da cidade. 
\item Com relação à função lucros variáveis por consumidor, temos a seguinte
função:
\[
V^{N}=\theta^{M}+\theta^{D}D+Z\theta_{Z}+W\theta_{W}
\]
\item Em que os $\theta$ são parâmetros a serem estimados, $D$ é uma \emph{dummy}
com valor de um caso exista mais de uma empresa no mercado, e $W$
e $Z$ são variáveis que explicariam variações na demanda e nos custos
em cada uma das regiões. 
\end{itemize}
\end{frame}

\begin{frame}\frametitle{Bresnahan e Reiss (1990)}
\scriptsize
\begin{itemize}
\item Os autores colocam variáveis tais como renda média dos consumidores,
a mediana da idade e dos anos de estudo, bem como o salário médio
no varejo. Com relação aos custos fixos, os autores colocam a seguinte
especificação:
\[
F^{N}=\gamma^{N}+\gamma^{D}D+\gamma_{W}RETWAGE+\gamma_{L}LANDVAL
\]
\item Em que $RETAWGE$ é o salário médio do varejo na região e $LANDVAL$
é o valor médio do acre. 
\item As variáveis relevantes para o estudo são os valores de $S^{D}$e
$S^{M}$, o mínimo tamanho de mercado que é viável a sustentação de
duas e uma empresa, respectivamente. 
\item Além disso, outro elemento importante é $\frac{V^{D}}{V^{M}}$ mensura
a fração pela qual os lucros variáveis por consumidor caem com a entrada
da segunda firma. 

\begin{itemize}
\item Por exemplo, se os duopolistas vendessem produtos iguais, esta razão
deveria ser igual a 0,5. 
\item Além disso, outra coisa interessante envolve a razão $\frac{F^{D}}{F^{M}}$. 
\end{itemize}
\end{itemize}
\end{frame}

\subsection{Breshanan e Reiss (1991)}
\begin{frame}\frametitle{Bresnahan e Reiss (1991)}

\begin{itemize}
\item O segundo dos artigos em que temos a aplicação e o desenvolvimento
desta metodologia também é \citet{Bresnahan1991a}. 
\item Este artigo é uma extensão da metodologia anterior para o caso de
um oligopólio. 
\item Ele faz a mesma premissa de equivalência entre os diferentes equilíbrios
múltiplos em estratégias puras em um jogo de entrada,assumindo a seguinte
função para os lucros de uma entrante em potencial:
\[
\Pi_{N}=S(\mathbf{Y},\lambda)\times V_{N}(\mathbf{Z},\mathbf{W},\alpha,\beta)-F(\mathbf{W},\gamma)+\varepsilon
\]
\end{itemize}
\end{frame}

\begin{frame}\frametitle{Bresnahan e Reiss (1991)}

\begin{itemize}
\item A verossimilhança deste negócio é dada por:
\begin{eqnarray*}
P(N & = & 0)=P(\Pi_{1}<0)=1-\Phi(\bar{\Pi}_{1})\\
P(N=n) & = & P(\bar{\Pi}_{n}>0\wedge\bar{\Pi}_{n+1}<0)=\Phi(\bar{\Pi}_{n})-\Phi(\bar{\Pi}_{n+1})\\
P(N\geq n^{max}) & = & \Phi(\bar{\Pi}_{n^{max}})
\end{eqnarray*}
\item A forma da função de lucros variáveis por consumidor é dada por 
\[
V_{N}=\alpha_{1}+\mathbf{X}\beta+\sum_{n=2}^{N}\alpha_{n}
\]
\item Neste paper, eles calculam os ``limites de entrada'', ou seja, o
menor tamanho de mercado necessário para sustentar exatamente $N$
empresas, que é dado por $S_{N}^{*}=\frac{\bar{F}}{V_{N}}$, bem como
o limite de entrada por empresa, que seria $\frac{S_{N}^{*}}{N}$
\end{itemize}
\end{frame}



\section{Modelos de Informação Completa}

\begin{frame}[fragile]\frametitle{\insertsection}
    \begin{itemize}
        \item Após os papers de Bresnahan e Reiss, a literatura avançou em quatro caminhos diferentes para lidar com o problema apontado por B\& R:
        \begin{itemize}
            \item Agregar o problema para eliminar a questão da multiplicidade de equilíbrios (\citet{Bresnahan1991a})
            \item Colocar restrições sobre a ordem dos \textit{players} que garanta a unicidade (\citet{Berry1992})
            \item Especificar uma regra de seleção de equilíbrio na área onde não dá unicidade
            \item Usar coisas mais invocadas, tipo \textit{Bounds Approach} ou \textit{Moment Inequalities}
        \end{itemize}
        \item Hoje vamos falar mais do segundo ponto.
    \end{itemize}


\end{frame}

\subsection{Berry, 1992 -- \textit{Econometrica}}
\begin{frame}[fragile]\frametitle{Berry 1992}
\begin{itemize}
    \item O Objetivo do paper de \citet{Berry1992} é avaliar os efeitos da escala de operação sobre a propabilidade de entrada em rotas que saem deste aeroporto.
    \item Hipótese subjacente: existe um jogo em dois estágios em cada mercado $i$. 
    \begin{itemize}
        \item No primeiro estágio, cada uma das $K_{i}$ decide entrar ou não no mercado
        \item Cada configuração é um vetor $s$ com dimensão $K_{i}$, composto por uns e zeros.
    \end{itemize}
    \item A função lucro de cada empresa é dada por:
    \[
    \pi_{ik}(N)=X_{i}\beta - \delta \ln{N} + Z_{ik} \alpha +\rho u_{io} + \sigma u_{ik}
    \]
    \item $u_{io}$ é um choque de mercado e $u_{ik}$ é um choque das firmas -- observado pelas empresas, mas não pelo econometrista. Adicionalmente, $\sigma=\sqrt{1-\rho^2}$.
\end{itemize}
\end{frame}

\begin{frame}[fragile]\frametitle{Probabilidade de Observarmos $N$ empresas}
    
    \begin{itemize}
        \item O Cálculo destas probabilidades é complicado por duas razões:
        \begin{enumerate}
            \item É multidimensional
            \item E tem uma região de integração meio maluca
        \end{enumerate}
        \item A região de integração é meio maluca porque ela depende dos não observáveis de TODAS as empresas e dos parãmetros do modelo
        \[
        Pr(a_{i1}=1|\theta)=\int_{\epsilon_{i1}}\int_{\epsilon_{i2}}\cdots \int_{\epsilon_{iK}} \mathbf{1}(a_{i1}=1|\theta,\tilde{\epsilon})d(\tilde{\epsilon})
        \]
    \end{itemize}


\end{frame}


\begin{frame}[fragile]\frametitle{Soluções de Berry para o problema dos Equilíbrios -- Alternativas}

\begin{itemize}
    \item Solução 1: $\delta=0$, $\rho=0$. Neste caso, não há interação estratégica e você cai em um probit tradicional para cada empresa.
    \item Solução 2: $\rho=0$. Neste caso, há a interação estratégica, e os componentes não observáveis de cada empresa não são correlacionados. Dá pra fazer igual Bresnahan e Reiss, só que tem que se focar em mercados com poucas empresas (a integral de probabilidade fica enorme, pq vc tem que calcular as probabilidades caso a caso).
    \item Solução 3: $\rho=1$ e heterogeneidade entre as firmas apenas observada. Aí dá um Probit Ordenado, mas implica que você tenha as características da pior empresa como regressores.
    \item 
\end{itemize}
    


\end{frame}


\begin{frame}[fragile]\frametitle{Solução de Berry -- A Ordem importa}
    
    \begin{itemize}
        \item A solução que ele considera como a mais interessante é colocar restrições sobre a ordem das empresas.
        \item Ou seja, quando os parâmetros do modelo apontarem para a região onde o modelo é ambíguo, o ``first mover'' (ou a firma mais eficiente) entra e faz com que a outra não entre.
        \item Essa é uma forma de seleção de equilíbrio bastante arbitrária. Outra forma de seleção poderia ser por meio dos dados, onde vc colocaria as duas possibilidades indicadas por probabilidades $\pi$, que seria estimada.
    \end{itemize}


\end{frame}

\subsection{O Algoritmo}

\begin{frame}[fragile]\frametitle{O Algoritmo}
    
    \begin{enumerate}
        \item Antes de rodar o código:
        \begin{enumerate}
            \item Escolher um valor inicial para os parâmetros $\theta=\{\beta, \delta, \alpha\}$
            \item Sortear um vetor de $u_{io}$ e outro vetor de $\{u_{ik}\}_{k=1}^{K}$
        \end{enumerate}
        \item Enquanto roda o otimizador, para um vetor de parâmetros $\hat{\theta}$, calcular:
        \begin{enumerate}
            \item Calcular o vetor $\hat{\epsilon_{ik}}=\hat{\rho} u_{io}+ \sqrt{1-\hat{\rho}^2} u_{ik}$, para um draw
            \item Ordenar os elementos do vetor $\hat{\epsilon_{ik}}$ de acordo com a ordem que vc impôs. Seja do menor para o maior, seja pela incumbente...
            \item Calcule os $\pi_{ik}(N)=X_{i}\hat{\beta} - \hat{\delta} \ln{N} + Z_{ik} \hat{\alpha} +\hat{\rho} u_{io} + \hat{\sigma} u_{ik}$ e os ordene. 
            \item Some as firmas de $n=1,\cdots,N$ até o ponto em que:
            \[
            v(N|\hat{\theta})+\hat{\epsilon_{iN}}\geq 0, v(N+1|\hat{\theta})+\hat{\epsilon_{iN+1}}< 0
            \]
            \item Esse vai ser o $N^{*}(\hat{\theta},\hat{\epsilon})$.
        \end{enumerate}
    \end{enumerate}

\end{frame}

\begin{frame}[fragile]\frametitle{O Algoritmo II}
    
    \begin{itemize}
        \item A condição de momento é a diferença entre o número predito - a média do de antes entre todos os draws - e o observado:
        \[
        \xi=\frac{1}{d} \sum_{d} N^{*} (\hat{\theta},\hat{\epsilon})
        \]
        \item Empilhamos essas diferenças em todos os mercados e temos as condições de momento, e podemos calcular a função critério:
        \[
        Q(\theta)=(\xi \mathbf{Z})(\mathbf{Z}^T \mathbf{Z})^{-1} (\xi \mathbf{Z})^{T}
        \]
    \end{itemize}


\end{frame}

\section{Exercício Empírico II -- Jogos Discretos e abordagens recentes}

\begin{frame}[fragile]\frametitle{O Paper}
    
    \begin{itemize}
        \item O paper em anexo reúne exercícios empíricos sobre um mesmo banco de dados para vermos como as coisas mudam.
        \item Estes dados são de um paper da Econometrica (Jia (2008)) e envolve um jogo de entrada entre a WalMart e a Kmart.
        \item 2065 mercados locais relativamente isolados.
        \item Função lucro para uma firma $i$ no mercado $m$:
        \[
        \pi_{im}=\alpha_{i}^{\prime} X_{m} + \beta_{i}^{\prime} Z_{im} + \delta_{i} y_{-im} + \varepsilon_{im}
        \]
        \item Os $\varepsilon$ não são observados pelo econometrista, mas são observados pelas empresas. Ou seja, pra elas isso é um jogo de informação completa.
        \item Sendo um jogo simultâneo, o Equilíbrio de Nash resultante leva às seguintes desigualdades
    \end{itemize}


\end{frame}


\begin{frame}[fragile]\frametitle{Desigualdades de Momento}

\begin{align*}
    y_{Km}&=\mathbf{1}(\alpha_{K}^{\prime} X_{m} + \beta_{K}^{\prime} Z_{Km} + \delta_{K} y_{Wm} + \varepsilon_{Km} \geq 0) \\
    y_{Wm}&=\mathbf{1}(\alpha_{W}^{\prime} X_{m} + \beta_{W}^{\prime} Z_{Wm} + \delta_{W} y_{Km} + \varepsilon_{Wm} \geq 0) \\
\end{align*}



\end{frame}

%\begin{comment}
\begin{frame}[allowframebreaks]
\bibliographystyle{aea}
\bibliography{C:/Bibliog/library}

\end{frame}

%\end{comment}




\end{document}


